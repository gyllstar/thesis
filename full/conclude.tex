\section{Thesis Summary}
\label{sec:thesis-summary}

%This thesis examined component failures in communication networks and algorithms to make networks robust to these failures.  
This thesis presented algorithms to make communication networks robust to component failures. % in communication networks and algorithms to make networks robust to these failures.  
Three separate but related problems were considered: node (i.e., switch or router) failure in traditional networks such as the Internet or wireless sensor networks,
the failure of critical sensors that measure voltage and current throughout the smart grid, and link failures in a smart grid communication network.

Chapter \ref{ch:rollback} considered scenarios where a malicious node injects and spreads false routing state throughout a network of routers.
We presented and evaluated three new algorithms -- \seconds, \purges, and \cpr -- for recovery in such scenarios. %from false state in distance vector routing 
Among these algorithms, we found that \cpr -- a checkpoint-rollback based algorithm -- yielded the lowest message overhead and convergence time over topologies
with fixed link weights but at the cost of storage overhead at the routers.
%However, \cpr required that routing table copies are stored at each router while the other two algorithms did not. %and synhcronization using logical clocks.
For topologies where link weights could change, \purge performed best because \purge globally invalidated false routing state, helping \purge avoid the problems that 
plagued \cpr and \seconds: updating large amounts of stale state (\cprs) and the \infinity problem (\seconds).
%Unlike \cprs, \purge had no stale state to update because \purge does not rollback in time.  
%The \infinity problem resulted in high message overhead for \seconds, while \purge eliminated the \infinity problem by globally purging false state before finding new least cost paths.


Next, in Chapter \ref{ch:pmu} we studied PMUs -- critical sensors being deployed in electric power grids worldwide that provide voltage and current measurements to power grid operators -- and 
a set of placement problems that considered detecting PMU measurement errors.  We formulated four PMU placement problems that 
considered two constraints: place PMUs ``near'' each other to allow for measurement error detection and use the minimal number of PMUs to infer the state 
of the maximum number of system buses and transmission lines. Each PMU placement problem was proved to be NP-Complete. As a first step, we proposed and evaluated  
a simple greedy approximation algorithm to each placement problem.  Using simulations based on topologies generated from real portions of the North American electric power grid, we found 
our greedy algorithms consistently reached close-to-optimal performance (on average within $97\%$ of optimal).  
Additionally, our simulations showed that requiring PMUs to placed near each other (in order to detect measurement errors) resulted in only a small decrease in system observability (on average
only $5\%$ fewer buses were observed with this additional constraint), which made for a strong case for imposing this requirement.
%Additionally, results showed that imposing a requirement that PMUs be placed near each other (in order to detect measurement errors) resulted in a small marginal decrease

In our final technical chapter, we designed algorithms that provide fast recovery from link failures in a smart grid communication network. 
We proposed, designed, and evaluated solutions to all three aspects of link failure recovery: link failure detection, algorithms that pre-computed backup multicast trees, and
fast backup tree installation.  Because these algorithms required making changes to network switches, these algorithms used OpenFlow to access and modify the forwarding plane of switches. 


As an alternative to slower algorithms based on end-to-end measurements, we presented \pcnts.  \pcnt used OpenFlow primitives to detect and report link failures inside the network.  
Next, a new problem was formulated, \mcs, that considered computing backup trees that reuse edges of already installed multicast trees as a means to reduce control plane signaling.
\mc was proved to be at least NP-hard so we designed an approximation algorithm for \mcs. Lastly, we presented two algorithms, \pre and \posts, that installed backup trees 
at OpenFlow controlled switches.  As an optimization to \pre and \posts, we designed \merges, an algorithm that consolidated forwarding rules at switches where multiple trees have common children.

These algorithms were evaluated with Mininet simulations using communication networks that mirrored the structure of actual portions of the North American power grid.
\pcnt packet loss estimates were accurate when monitoring even a small number of flows over short time window: after sampling only $75$ packets, the $95\%$ confidence interval of \pcnt loss estimates 
were within $15\%$ of the true loss probability. 
\pre had a $10x$ decrease in control messages compared with \post because \pre required only a single control message to install each backup tree since all other rules were pre-installed,
whereas \post had to signal multiple switches to install each backup tree. 
However, \pres's pre-installed forwarding rules accounted for a significant portion of scarce OpenFlow switch table capacity, especially in cases with many multicast groups (up to $35\%$ of
flow table capacity of a standard OpenFlow switch). Fortunately, \merge reduced the amount of pre-installed forwarding state by a factor of $2-2.5$, to acceptable levels.


\section{Future Work}
\label{sec:thesis-future}

Our research in Chapter \ref{ch:rollback} %on recovery from false routing state injected into a network of routers 
only considered a single instance of false state where we assumed that the compromised
node falsely claimed the minimum distance to all nodes.  As future work, we are interested in exploring how our algorithms (i.e, \seconds, \purges, and \cprs) 
respond to other possible false state values. Some interesting alternatives include false state that maximizes the effect of the \infinity problem and false state that contaminates a bottleneck link.
We would also like to see how our distributed recovery algorithms compare with a Software Defined Networking (SDN) based approach to false state recovery. 
It is likely that the concerns over convergence time addressed by our distributed recovery algorithms are non-factors with an SDN approach.  With SDN, recovery paths can be
computed centrally at the controller (as we did when computing backup multicast trees in Chapter \ref{ch:reliable-mcast}), negating the need for switches to exchange messages to compute
new paths. However, new challenges are likely to emerge with an SDN-based approach. For example, in what order should routers be signaled to install new routes such that 
the \infinity problem is minimized?
%and then installed at the switches because SDN's centralized management and control enables 
%is centralized with SDN the issue of convergence time that dominated our distributed recovery algorithms disappear. 



%finding the worst possible false state a compromised node can inject. Some options include the minimum distance to all nodes (e.g., our choice for false state used in this paper), 
%state that maximizes the effect of the count-to-∞ problem, and false state that contaminates a bottleneck link.

There are several topics for future work from Chapter \ref{ch:pmu} on PMU placement. The success of the greedy PMU placement algorithms suggests that bus systems have special topological characteristics,
and investigating these properties could provide interesting insight to power grid topologies. 
%As additional item for future work, we would like to evaluate our greedy approximations 
%using the IEEE bus systems used to evaluate our greedy approximations are based on portions of the North American power grid from the 1960s
Because our brute-force optimal algorithm could only produce data points for small inputs, much could be learned by implementing  
the integer programming approach proposed by Xu and Abur \cite{Xu04} to solve \fulls.  This would provide valuable data points to measure the relative performance of {\tt greedy}.


From Chapter \ref{ch:reliable-mcast}, several problems still remain to be solved. One problem of interest is using optimization criteria different from \mcs's objective function 
to compute backup trees and then evaluate \pres, \posts, and \merge performance using these backup trees.  
For example, backup trees may be computed with the goal of protecting against the worst-case impact of a subsequent link failure
by minimizing the maximum number of multicast trees using a single link. %These backup tree help protect against the worst-case impact of a subsequent link failure. 
It is unknown how effective our installation algorithms would be given these types of backup trees. % when given backup trees other than those computed using \steiners.

Measurements using real OpenFlow hardware switches would strengthen our \pcnt processing time and backup tree installation time results, which both suffered from inaccuracies due to Mininet's
performance fidelity issues.  At the end of Section \ref{subsec:pcnt} we commented on how \pcnt can be easily extended to monitor packet loss between multiple non-adjacent switches.  We showed
that in some cases packet loss at all links connecting switches used in the same multicast tree can be estimated using only a single \pcnt session 
with measurement points at only a subset of these switches. It would be
interesting to quantify the savings (in terms of switch processing time) of this approach when compared to a naive implementation that runs separate \pcnt sessions between all adjacent switches.  Our 
\pcnt simulation results suggest that these savings could be significant. 
%Lastly, the complexity of the problem \merge addresses is an open-question: find the minimum number of forwarding rules for a set of multicast trees. 
Lastly, the problem \merge addresses -- find the minimum number of forwarding rules for a set of multicast trees -- has unknown complexity.
We conjectured that this problem is NP-hard in Section \ref{subsubsec:merge-discuss}. %3.3.5.4.

% what problems can OF programmed network solve that cannot in standard system.  no longer constrained by what protocols vendors choose to implement, 
This thesis provided some encouraging initial results of how SDN (and specifically OpenFlow) can simplify fault detection and recovery but we did so under somewhat favorable conditions.  For example,
in Chapter \ref{ch:reliable-mcast} we assumed that any non-OpenFlow switches or routers had no influence on our recovery algorithms (this is equivalent to assuming 
that all network switches support OpenFlow). 
In practice, it is likely that OpenFlow switches will coexist with existing network infrastructure (e.g., IP routers and switches), which will likely complicate matters.  One potential
issue is that many backbone IP routers use MPLS to reroute flows in response to link failures.  This would result in new paths between OpenFlow switches. 
%, using non-OpenFlow switches, between OpenFlow switches. 
In these cases, it is unclear if OpenFlow switches and the control plane need to be aware of these path changes. %  in what cases these changes can be ignored. 
Also what is the best way for the OpenFlow controller to monitor the state of non-OpenFlow switches and routers? 
Would it be sufficient to passively monitor control messages sent among IP routers?  If so, how much control state needs to be tracked and what is the cost of doing so? 
%These are just a few of the open questions regarding how OpenFlow-based fault recovery algorithms can effectively coexist with similar distributed algorithms 
%built into operational IP routers and switches. 

Our hope is that the preliminary results in Chapter \ref{ch:reliable-mcast} will encourage other researchers to develop OpenFlow-based solutions for smart grid communication.  One promising
topic, not addressed in this thesis, is traffic engineering, which figures to play an important role in smart grid data dissemination. 
We believe OpenFlow's capabilities to directly control traffic flows makes OpenFlow well-suited to designing simple and effective traffic engineering solutions for the smart grid.

More broadly, we believe OpenFlow and software-defined networking (SDN) in general is the ``right'' approach to designing a smart grid communication network.  In particular, we feel a smart grid
network should be open: practitioners and researchers should have access to operational switches and routers.  It is difficult to predict how well proposed algorithms will work in a live smart
grid network, especially considering some of the challenging QoS requirements of smart grid applications (highlighted in Chapter \ref{subsec:pmu-requirements}).  Even more challenging is anticipating
all the new applications and future uses of a smart grid network. SDN and OpenFlow allow the network and its algorithms to evolve as these new requirements and network applications emerge.  Through the 
growth of the Internet, we have learned the high cost of trying to retrofit a closed system to meet unforeseen requirements and demands.  Building a smart grid communication network using SDN
will position the network to help avoid repeating this critical flaw in the Internet design. As former president George W. Bush said, 
\begin{quote}
``There's an old saying in Tennessee —- I know it's in Texas, probably in Tennessee —- that says, `fool me once, shame on ... shame on you. Fool me ... you can't get fooled again!' ''
 \footnote{\url{http://content.time.com/time/specials/packages/article/0,28804,1870938_1870943_1870944,00.html}}
\end{quote}
