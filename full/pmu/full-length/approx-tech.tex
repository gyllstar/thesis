\section{Approximation Algorithms}
\label{sec:approx}

Because all four placement problems are NPC, we propose greedy approximation algorithms for each problem, which iteratively add 
a PMU in each step to the node that observes the maximum number of new nodes. We present two such algorithms, one which directly addresses \maxinc ({\tt greedy}) and the other 
\xvalpart ({\tt xvgreedy}). {\tt greedy} and {\tt xvgreedy} can easily be used to solve \full and \xvals, respectively, by selecting the appropriate $k$ value to ensure full observability.
We prove these algorithms have polynomial complexity (i.e., they are in $\mathcal{P}$), making them feasible tools for approximating optimal PMU placement. 

{\bf {\tt greedy} Algorithm}. We start with $\Phi = \emptyset$.  At each iteration, we add a PMU to the node that results in the observation of the maximum number of 
new nodes. The algorithm terminates when all PMUs are placed.  {\footnote {\small This is the same greedy algorithm proposed by Aazami and Stilp \cite{Aazami07}. }}
The pseudo-code for {\tt greedy} is given in Algorithm \ref{alg:greedy}.
%We refer to this algorithm as {\tt greedy} and give the pseudo-code in Algorithm \ref{alg:greedy}.

\begin{algorithm}
\caption{{\tt greedy} with input $G=(V,E)$ and $k$ PMUs}
\label{alg:greedy}

\begin{algorithmic}[1]

\STATE{$\Phi_G \leftarrow \emptyset$}
\FOR{$k$ iterations}
	\STATE{$maxObserved \leftarrow 0$}
	\FOR{{\bf each} $v \in (V - \Phi_G$)}
		\STATE{$numObserved \leftarrow 0$}
		\FOR{{\bf each} $u \in (\Phi_G \cup \{v\})$}
			\STATE{add PMU to $u$}
			\STATE{apply O1 at $u$ and update $numObserved$}
		\ENDFOR
		\REPEAT
			\STATE{$flag \leftarrow False$}
			\FOR{{\bf each} $w \in (V - (\Phi_G \cup \{v\}))$}
				\IF{$w \in (V_Z \cap \Phi_G^R)$ and $w$ has $1$ unobserved neighbor}
					\STATE{apply O2 at $w$ and update $numObserved$}
					\STATE{$flag \leftarrow True$}
				\ENDIF
			\ENDFOR
		\UNTIL{$flag = False$}
		\IF{$numObserved > maxObserved$}
			\STATE{$greedyNode \leftarrow v$}
			\STATE{$maxObserved \leftarrow numObserved$}
		\ENDIF
	\ENDFOR
	\STATE{$\Phi_G \leftarrow \Phi_G \cup \{greedyNode\}$}
\ENDFOR
\end{algorithmic}
\end{algorithm}

\begin{theorem}
For input graph $G=(V,E)$ and $k$ PMUs {\tt greedy} has $O(dkn^3)$ complexity, where $n=|V|$ and $d$ is the maximum degree node in $V$.
\label{thm:greedy-complex}
\end{theorem}
\begin{proof}
%In our proof we implicitly refer to Algorithm \ref{alg:greedy} when referring to the steps of the algorithm.
%Specifically, when we refer to the {\tt greedy} algorithm steps we are referencing the Algorithm \ref{alg:greedy} steps.
The procedure to determine the number of nodes observed by a candidate PMU placement spans steps $6 - 18$.
{\footnote {\small In this proof, step $i$ refers to the $i^{th}$ line  in Algorithm \ref{alg:greedy}.}}
First, we apply O1 at each PMU node (steps $6 -9$). O1 takes $O(d)$ time 
to be applied at a single node.  Because $|\Phi_G| \leq k$, the total time to apply O1 is $O(dk)$. 

Then, we iteratively apply O2 (steps $10-18$), terminating when no new nodes are observed.  Like O1, applying O2 at a single node takes $O(d)$ time. 
In each iteration, if possible we apply O2 at each $v \in (V_Z \cap \Phi_G^R)$ (steps $13-16$). %If at least one node is observed we execute another iteration of O2 evaluation.
It total, the $loop$ spanning steps $10-18$ repeats at most $O(n)$ times.  This occurs when only a single new node is observed in each iteration.  The $for$ loop spanning steps $12-17$
repeats $O(n)$ times. We conclude that O2 evaluation for each set of candidate PMU locations takes $O(dn^2)$ time.

In order to determine the placement of each PMU, we try all possible PMU placements among nodes without a PMU. We place the PMU at the node that observes the maximum number of new nodes.  This 
corresponds to Steps $4-23$, in which the $for$ loop iterates $O(n)$ times. 
Thus the complexity of Steps $4-23$ is $O(dn^3)$. 

Finally, the outer most $for$ loop (Steps $2-25$) iterates $k$ times: one iteration
to determine the greedy placement of each PMU.  We conclude that the complexity of {\tt greedy} is $O(dkn^3)$. \qed
\end{proof}
%At most, O2 evaluation requires $O(n)$ iterations in which we attempt to apply O2 at $O(n)$ nodes. This yields $O(n^2)$ complexity.



%After placing PMUs in candidate locations, we apply rule O1 at each PMU node.  It takes $O(d)$ time, where $d$ is the maximum degree of all $v \in V$, to apply O1 at each PMU node.  With 
%$k$ PMUs, the total time to apply O1 is $O(kd)$.


{\bf {\tt xvgreedy} Algorithm}. {\tt xvgreedy} is almost identical to {\tt greedy}, except that PMUs are added in pairs such that the selected pair observe
the maximum number of nodes under the condition that the PMU pair satisfy one of the cross-validation rules. % and observe the maximum number of new nodes.
We provide the pseudo code for {\tt xvgreedy} in Algorithm \ref{alg:xvgreedy}.
%We call this algorithm {\tt xvgreedy} and provide pseudo code for {\tt xvgreedy} in Algorithm \ref{alg:xvgreedy}.

\begin{algorithm}
\caption{{\tt xvgreedy} with input $G=(V,E)$ and $k$ PMUs}
\label{alg:xvgreedy}

\begin{algorithmic}[1]

\STATE{$\Phi_G \leftarrow \emptyset$}
\FOR{$k$ iterations}
	\STATE{$maxObserved \leftarrow 0$}
	\STATE{$C \leftarrow$ all cross-validated node pairs in $(V - \Phi_G)$}
	\FOR{{\bf each} $\{v_1,v_2\} \in C$}
		\STATE{$numObserved \leftarrow 0$}
		\FOR{{\bf each} $u \in (\Phi_G \cup \{v_1,v_2\})$}
			\STATE{add PMU to $v_1$ and $v_2$}
			\STATE{apply O1 at $u$ and update $numObserved$}
		\ENDFOR
		\REPEAT
			\STATE{$flag \leftarrow False$}
			\FOR{{\bf each} $w \in (V - (\Phi_G \cup \{v_1,v_2\}))$}
				\IF{$w \in (V_Z \cap \Phi_G^R)$ and $w$ has $1$ unobserved neighbor}
					\STATE{apply O2 at $w$ and update $numObserved$}
					\STATE{$flag \leftarrow True$}
				\ENDIF
			\ENDFOR
		\UNTIL{$flag = False$}
		\IF{$numObserved > maxObserved$}
			\STATE{$greedyNodes \leftarrow \{v_1,v_2\}$}
			\STATE{$maxObserved \leftarrow numObserved$}
		\ENDIF
	\ENDFOR
	\STATE{$\Phi_G \leftarrow \Phi_G \cup greedyNodes$}
\ENDFOR
\end{algorithmic}
\end{algorithm}

\begin{theorem}
For input graph $G=(V,E)$ and $k$ PMUs {\tt xvgreedy} has $O(kdn^3)$ complexity, where $n=|V|$ and $d$ is the maximum degree node in $V$.
\label{thm:xvgreedy-complex}
\end{theorem}
\begin{proof}
	The only difference between {\tt xvgreedy} and {\tt greedy} is that {\tt xvgreedy} only considers pairs of cross-validated nodes. For this reason, 
	step $4$ in Algorithm \ref{alg:xvgreedy} does not appear in Algorithm \ref{alg:greedy}.  We can find all pairs of cross-validated nodes in $O(d^2n)$ time.  We do so by 
	implementing a breadth-first search at each $v \in (V - \Phi_G)$ but stopping at a depth of $2$.  This takes $O(d^2)$ time for each node and since 
	$O(n)$ searches are executed, step $4$ takes $O(d^2n)$ time.

	Because all other parts of Algorithm \ref{alg:greedy} and Algorithm \ref{alg:xvgreedy} are nearly identical -- Algorithm \ref{alg:xvgreedy}
	adds PMUs in pairs while Algorithm \ref{alg:greedy} adds PMUs one-at-a-time -- we are able to directly apply
	the analysis from Theorem \ref{thm:greedy-complex} in this proof.  Therefore, we conclude the complexity of {\tt xvgreedy} is $O(k(d^2n + dn^3)) = O(dkn^3)$. \qed
\end{proof}







%\xvals's greedy algorithm can either:
%\begin{itemize}
%	\item Add pairs of PMUs at each iteration such the selected pair of nodes observes the most nodes. both nodes need to be unobserved. pros/cons
%	\begin{itemize}
%		\item ignores odd number of PMUs
%	\end{itemize}
%	\item Add a single PMU at each iteration so the PMU is cross-validated and observes the maximum nodes (among the cross-validated options). pros/cons
%	\begin{itemize}
%		\item may be better to add PMUs in pairs
%
%		\item may restrict options by having to place nodes near each other ...
%	\end{itemize}
%
%\end{itemize}

