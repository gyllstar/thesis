\section{Approximation Algorithms}
\label{sec:approx}

Because all four placement problems are NPC, we propose greedy approximation algorithms for each problem, which iteratively add 
a PMU in each step to the node that observes the maximum number of new nodes. We present two such algorithms, one which directly addresses \maxinc ({\tt greedy}) and the other 
\xvalpart ({\tt xvgreedy}).  \\
{\tt greedy} and {\tt xvgreedy} can easily be used to solve \full and \xvals, respectively, by selecting the appropriate $k$ value to ensure full observability.
%Our Technical Report \cite{Tech12} gives the pseudo code for {\tt greedy} and {\tt xvgreedy} and also includes proofs
%that these algorithms have polynomial complexity (i.e., they are in $\mathcal{P}$), 
%making them feasible tools for approximating optimal PMU placement. 
%We prove these algorithms have polynomial complexity (i.e., they are in $\mathcal{P}$), making them feasible tools for approximating optimal PMU placement. 

{\bf {\tt greedy} Algorithm}. We start with $\Phi = \emptyset$.  At each iteration, we add a PMU to the node that results in the observation of the maximum number of 
new nodes. The algorithm terminates when all PMUs are placed.  {\footnote {\small The same greedy algorithm is proposed by Aazami and Stilp \cite{Aazami07}. }}
%The pseudo-code for {\tt greedy} and a proof that {\tt greedy} has polynomial running time is given in our Technical Report \cite{Tech12}.


{\bf {\tt xvgreedy} Algorithm}. {\tt xvgreedy} is almost identical to {\tt greedy}, except that PMUs are added in pairs such that the selected pair observe
the maximum number of nodes under the condition that the PMU pair satisfy one of the cross-validation rules. % and observe the maximum number of new nodes.
%We provide the pseudo code for {\tt xvgreedy} and prove that {\tt xvgreedy} has polynomial running time in our Technical Report \cite{Tech12}. 

Our Technical Report \cite{Tech12} gives the pseudo code for {\tt greedy} and {\tt xvgreedy} and includes proofs
that these algorithms have polynomial complexity, making them feasible tools for approximating optimal PMU placement. 
%We provide the pseudo code for {\tt greedy} and {\tt xvgreedy} in our Technical Report \cite{Tech12}. Our Technical Report \cite{Tech12} also includes proofs
%that these algorithms have polynomial complexity (i.e., they are in $\mathcal{P}$), 
%making them feasible tools for approximating optimal PMU placement. 

