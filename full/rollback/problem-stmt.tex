\section{Problem Formulation}
\label{sec:problem}

We consider distance vector routing \cite{Gall87} over arbitrary network topologies. We model a network as an undirected graph, $G=(V,E)$,
with a link weight function $w: E \rightarrow \mathbb{N}$.
{\footnote {\small Recovery is simple with link state routing: each node uses its complete topology map to compute new least cost paths that avoid all compromised nodes.
Thus we do not consider link state routing in this chapter.}}
Each node, $v$, maintains the following state as part of distance vector: a vector of all adjacent nodes ($adj(v)$), a vector of least cost distances to all
nodes in $G$ (\minvvs), and a \emph{distance matrix} that contains distances to every node in the network via each adjacent node (\dmatrixvs). 

For simplicity, we present our recovery algorithms in the case of a single compromised node.
We describe the necessary extensions to handle multiple compromised nodes in Section \ref{subsec:mult}.
We assume that the identity of the compromised node is provided by a different algorithm, and thus do not consider this problem in this paper.
Examples of such algorithms include \cite{Arini,Feam,Vishal02,Pad03,Paul02}. %\cite{Arini}, \cite{Feam}, \cite{Feldmann} %\cite{Arini, Feam, Feldmann}
Specifically, we assume that at time $t_b$, this algorithm is used to notify all neighbors of the 
compromised node. Let $t'$ be the time the node was compromised.

For each of our algorithms, the goal is for all nodes to recover ``correctly'': all nodes should remove the compromised nodes as a destination and find
new least cost distances that do not use a compromised node. If the network becomes disconnected as a result of removing the compromised node, all
nodes need only compute new least cost distances to all other nodes within their connected component.

For simplicity, let \bad denote the compromised node, let \oldvector refer to $\overrightarrow{min}_{\overline{v}}$ 
before \bad was compromised, and let \badvector denote $\overrightarrow{min}_{\overline{v}}$ after \bad has been compromised.
Intuitively, \oldvector and \badvector are snapshots of the compromised node's least cost vector taken at two different timesteps: \oldvector marks the snapshot taken before \bad was compromised and 
\badvector represents a snapshot taken after \bad was compromised.


Table \ref{tab:abbrev} summarizes the notation used in this chapter. 

\begin{table}[t]
\begin{center}
\begin{tabular}{l l} 
\hline \hline
   	{\bf Abbreviation} & {\bf Meaning} \\
		  \hline 
			\minvi & node $i$'s the least cost vector \\
			\dmatrixi & node $i$' distance matrix \\
		%	\lcd & link cost change event \\
			DV & Distance Vector \\
			\hline
			$t_b$ & time the compromised node is detected \\
			$t'$ & time the compromised node was compromised \\
			\badvector & compromised node's least cost vector at and after $t$  \\
			\oldvector & compromised node's least cost vector at and before $t'$ \\
			\bad & compromised node \\ 
			$adj(v)$ & nodes adjacent to $v$ in $G'$ \\ 
			\hline \hline
			\end{tabular}
			\end{center}
\caption{Table of abbreviations.} 
			%* The distance matrix for node $v$ contains $v$'s distance to all nodes $v_d \in V$ via all $v_n: v_n \in adj(v)$.}
\label{tab:abbrev}
\end{table}
