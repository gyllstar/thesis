\unnumberedchapter{Introduction}

Communication network components (routers, links, and sensors) fail. %over the lifetime of the network 
These failures can cause widespread network service disruption and outages, and potentially critical errors for network applications.
\textit{In this thesis, we examine how networks -- traditional networks and networked cyber-physical systems, such as the electric power grid -- can be made more robust to component failures.}

%We propose on-demand recovery algorithms for distributed network algorithms that optimize for control message overhead and convergence time,
%and preplanned approaches to recovery for electric power grid applications, where reliability is key. 
%An electric power grid consists of a set of buses -– electric substations, power generation centers, or aggregation points of electrical loads -– and transmission lines connecting those buses.
%We refer to modern and future electric power grids that automate power grid operations using sensors and wide-area communication as the \emph{smart grid}.




\section{Thesis Overview}

\subsection{Component Failures in Communication Networks}

We consider three separate but related problems in this thesis: node (i.e., switch or router) failure in traditional networks such as the Internet or wireless sensor networks,
the failure of critical sensors that measure voltage and current throughout the smart grid, and link failures in a smart grid communication network.  The term \emph{smart grid} refers to 
modern and future electric power grids that automate power grid operations using sensors and wide-area communication.

For distributed network algorithms, a malicious or misconfigured node can inject and spread incorrect state throughout the distributed system. 
Such false state can degrade the performance of the network or render it unusable. For example, in 1997 a significant portion of Internet traffic was routed through a 
single misconfigured router that had spread false routing state to several Internet routers.  As a result, a large potion of the Internet became inoperable for several hours \cite{Neumann97}. 


%In a smart grid, especially, component failure can be catastrophic.
Component failure in a smart grid can be especially catastrophic.
For example, if smart grid sensors or links in its supporting communication network fail, smart grid applications can make incorrect decisions and take corresponding (incorrect) actions. 
Critical smart grid applications required to operate and manage a power grid are especially vulnerable to such failures because typically these applications have strict data delivery requirements,
needing both ultra low latency and assurance that data is received correctly. 
In the worst case, component failure can lead to a cascade of power grid failures like the August 2003 blackout in the USA \cite{2003Blackout} and the 
recent power grid failures in India that left hundreds of millions of people without power  \cite{IndiaBlackout}.






\subsection{Approaches to Making Networks More Robust to Failures}



For many distributed systems, recovery algorithms operate on-demand (as opposed to being preplanned) because algorithm and system state is typically distributed throughout the network of nodes.  
As a result, fast convergence time and low control message overhead are key requirements for efficient recovery from component failure. 
In order to make the problem of on-demand recovery in a distributed system concrete, we investigate distance vector routing as an instance of this problem where nodes must recover
from incorrectly injected state information.
Distance vector forms the basis for many routing algorithms widely used in the Internet (e.g., BGP, a path-vector algorithm) and in multi-hop wireless networks (e.g., AODV, diffusion routing).

In the first technical chapter of this thesis, we design, develop, and evaluate three different approaches for correctly recovering from the injection of false distance vector routing state (e.g., a compromised node incorrectly
claiming a distance of $0$ to all destinations). Such false state, in turn, may propagate to other routers through the normal execution of distance vector routing, causing other nodes to (incorrectly) route via the misconfigured node, making this a network-wide problem.
Recovery is correct if the routing tables of all nodes have converged to a global state where, for each node, all compromised nodes are removed as a destination and
no least cost path routes through a compromised node.  
%Recovery is correct if the routing tables of all nodes have converged to a global state meeting the following conditions.  
%For each node, all compromised nodes are removed as a destination and no least cost path routes through a compromised node.  
%Recovery is correct if the routing tables of all nodes have converged to a global state where all nodes have removed each compromised node as a destination
%and no node has a least cost path to any destination that routes through a compromised node.  

The second and third thesis chapters consider robustness from component failure  in the context of the smart grid. Because reliability is a key requirement for the smart grid, each chapter focuses 
on preplanned approaches to failure recovery.

In our second thesis chapter, we study the placement of a sensor, called a Phasor Measurement Unit (PMU), currently being deployed in electric power grids worldwide. 
PMUs provide voltage and current measurements at a sampling rate orders of magnitude higher than the status quo.  As a result, PMUs can 
both drastically improve existing power grid operations and enable an entirely new set of applications, such as the reliable integration of renewable energy resources. 
We formulate a set of problems that consider PMU measurement errors, which have been observed in practice.  Specifically, we specify four PMU placement problems
that aim to satisfy two constraints: place PMUs ``near'' each other to allow for measurement error detection and use the minimal number of PMUs to infer the state of the maximum number of system buses and transmission lines. 
For each PMU placement problem, we prove it is NP-Complete, propose a simple greedy approximation algorithm, and evaluate our greedy solutions.

In our final technical chapter, we design algorithms that provide fast recovery from link failures in a smart grid communication network. 
We propose, design, and evaluate solutions to all three aspects of link failure recovery: (a) link failure detection, (b) algorithms for pre-computing backup multicast trees, and
(c) fast backup tree installation. Because this requires modifying network switches and routers, we use OpenFlow -- an open standard that cleanly separates the control 
and data planes for use in network management and control -- to program data plane forwarding using novel control plane algorithms. 


To address (a), we design link-failure detection and reporting mechanisms that use OpenFlow to detect link failures when and where they occur \emph{inside} the network.
For part (b), we formulate a new problem, \mcs, that aims to pre-compute backup multicast trees that minimize control plane signaling overhead. We prove \mc 
is at least NP-hard and present a corresponding approximation algorithm.
Lastly, two control plane algorithms are proposed that signal data plane switches to install pre-computed backup trees. 
An optimized version of each installation algorithm is designed that finds a near minimum set of forwarding rules 
by sharing forwarding rules across multicast groups. This optimization
%by using OpenFlow to dynamically write (and delete) identifiers in packet headers to allow forwarding rules to be shared across multicast groups. This optimization
reduces backup tree install time and control state.  
We implement these algorithms using the POX open-source OpenFlow controller \cite{Pox} and evaluate them using the Mininet emulator \cite{Lantz10}, quantifying control plane signaling and 
installation time.





\section{Thesis Contributions}


The main contributions of this thesis are:
\begin{itemize}

	\item  We design, develop, and evaluate three different algorithms -- \seconds, \purges, and \cpr -- for correctly recovering from the injection of false routing state in distance vector routing.
		\second performs localized state invalidation, followed by network-wide recovery using the traditional distance vector algorithm. 
		\purge first globally invalidates false state and then uses distance vector routing to recompute distance vectors.  \cpr takes and stores local routing table 
		snapshots at each router, and then uses 
		a rollback mechanism to implement recovery. We prove the correctness of each algorithm for scenarios of single and multiple compromised nodes.

	

	\item We use simulations and analysis to evaluate \seconds, \purges, and \cpr in terms of control message overhead and convergence time. We find that \second performs poorly due to routing loops.  
	Over topologies with fixed link weights, \purge performs nearly as well as \cpr even though our simulations and analysis assume near perfect conditions for \cprs.
	Over more realistic scenarios in which link weights can change, we find that \purge yields lower message complexity and faster convergence time than \cpr and \seconds. 


	\item We define four PMU placement problems, three of which are completely new, that place PMUs at a subset of electric power grid buses. 
		Two PMU placement problems consider measurement error detection by requiring PMUs to be placed ``near'' each other to allow for their measurements to be cross-validated. 
		For each PMU placement problem, we prove it is NP-Complete and propose a simple greedy approximation algorithm. 

	
	\item We prove our greedy approximations for PMU placement are correct and give complexity bounds for each.  Through simulations over synthetic topologies generated using real portions of the 
		North American electric power grid as templates, we find that our greedy approximations yield results that are close to optimal: on average, within $97\%$ of optimal.  We also find that 
		imposing our requirement of cross-validation to ensure PMU measurement error detection comes at small marginal cost: on average, only $5\%$ fewer power grid buses are observed (covered) 
		when PMU placements require cross-validation versus placements that do not. 
	

	\item We propose, implement, and evaluate a suite of algorithms for fast recovery from link failures in a smart grid communication network: \pcnts, \steiners, \pres, \posts, and \merges.
	\pcnt uses OpenFlow to accurately detect link failures inside the network, rather than using slower end-to-end measurements. Then, we define a new problem, \mcs, that computes backup 
	multicast trees with the aim of minimizing control plane signaling overhead. %`multicast trees with the aim of maximizing the reuse of installed multicast trees. 
	This problem is shown to be at least NP-hard, motivating the design of an approximation, \steiners. 
	Next, we design two algorithms -- \pre and \post -- for fast backup tree installation. \pre pre-installs backup tree forwarding rules and
	activates these rules after a link failure is detected, while, \post installs backup trees \emph{after} a link a failure is detected. 
	Lastly, we present \merges, an algorithm that can be applied to \pre and \post to speed backup tree installations and reduce the amount of pre-installed forwarding state. 
	\merge does so using local optimization to create a near minimal set of forwarding rules by ``merging'' forwarding rules in cases where multiple multicast trees have common forwarding behavior.

	\item 
	We use Mininet \cite{Lantz10} simulations to evaluate our algorithms over communication networks based on real portions of the power grid.
	We find that \pcnt provides fast and accurate link loss estimates: after sampling only $75$ packets
	the $95\%$ confidence interval is within $15\%$ of the true loss probability.  
	Additionally, we find \pre yields faster recovery than \post (\post sends up to $10$ times more control messages than \pres) 
	but at the cost of storage overhead at each switch (pre-installed backup trees can account for as much
	as $35\%$ of the capacity of a conventional OpenFlow switch \cite{Curtis11}).
	Finally, we observe that \merge reduces control plane messaging and the amount of pre-installed forwarding state by a factor of $2$ to $2.5$ when compared to a standard multicast
	implementation, resulting in faster installation and manageable sized flow tables.


	%Lastly, we present \merges, an algorithm that can be applied to \pre and \post to boost performance; 
	%with \posts, \merge speeds backup tree installations and, when used with \pres, reduces the amount of pre-installed forwarding state. 




		

\end{itemize}




\section{Thesis Outline}

The rest of this thesis is organized as follows.  We present algorithms for recovery from false routing state in distributed routing algorithms in Chapter \ref{ch:rollback}.  
In Chapter \ref{ch:pmu}, we formulate PMU placement problems that provide measurement error detection.  Chapter \ref{ch:reliable-mcast} presents our algorithms for fast recovery 
from link failures in a smart grid communication network.  We conclude, in Chapter \ref{ch:conclusion}, with a summary and discussion of open problems emerging from this thesis.
