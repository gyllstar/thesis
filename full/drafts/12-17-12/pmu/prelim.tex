\section{Preliminaries}
\label{sec:prelim}

In this section we introduce notation and underlying assumptions (Section \ref{subsec:notation-assume}), 
and define our observability (Section \ref{subsec:observe}) and cross-validation (Section \ref{subsec:xval-rules}) rules.

\xx{Maybe merge Observability Rules and Cross-Validation Rules Sections}

\subsection{Assumptions, Notation, and Terminology}
\label{subsec:notation-assume}

We model a power grid as an undirected graph $G=(V,E)$.  Each $v \in V$ represents a bus. % A bus is either an electrical substation, a power generation center, or an aggregation of loads. 
$V=V_Z \cup V_I$, where $V_Z$ is the set of all zero-injection buses and $V_I$ is the set of all non-zero-injection buses.  A bus is zero-injection if it has no load nor generator \cite{Zhang10}.
All other buses are non-zero-injection, which we refer to as injection buses.
Each $(u,v) \in E$ is a transmission line connecting buses $u$ and $v$. 

Consistent with the conventions in \cite{Baldwin93,Brueni05,Abur06,Mili90,Xu04,Xu05}, we assume: PMUs can only be placed on buses and a PMU on a bus measures 
the voltage phasor at the bus and the current phasor of all transmission lines connected to it. For convenience, we refer to any bus with a PMU as a \emph{PMU node}. 

For  $v\in V$ define let $\Gamma(v)$ be the set of $v$'s neighbors in $G$.
A PMU placement $\Phi_G \subseteq V$ is a set of nodes at which PMUs are placed,
and $\Phi^R_G\subseteq V$ is the set of observed nodes for graph $G$ with placement $\Phi_G$ (see definition of observability below).
%$k^* = \min \{|\Phi_G|:\Phi^R_G=V\}$ denotes the minimum number of PMUs needed to observe the entire network. %Where the graph $G$ is clear from the context, we drop the $G$ subscript.


\subsection{Observability Rules}
\label{subsec:observe}

We use the simplified observability rules stated by Brueni and Heath \cite{Brueni05}:
\begin{enumerate}
	
	\item {\bf Observability Rule 1 (O1)}.  {\it If node $v$ is a PMU node, then $v \cup \Gamma(v)$ is observed. } %Formally, if $v \in \Phi_G$, then $\Gamma[v] \subseteq \Phi^R_G$. }

	\item {\bf Observability Rule 2 (O2)}. {\it If a zero-injection node, $v$, is observed and  $\Gamma(v)\backslash\{u\}$ is observed for some $u\in\Gamma(v)$, then $v \cup \Gamma(v)$ is observed.}
	%Formally, if $v \in \Phi^R_G \cap V_Z$ and $|\Gamma(v) \cap (V - \Phi^R_G)| \leq 1$, then $\Gamma[v] \subseteq \Phi^R_G$. }

\end{enumerate}
Since O2 only applies with zero-injection nodes, the number of zero-injection nodes can greatly affect system observability. 

\subsection{Cross-Validation Rules}
\label{subsec:xval-rules}

Cross-validation formalizes the intuitive notion of placing PMUs ``near'' each other to allow for measurement error detection. 
%`A PMU node is cross-validated if {\em the PMU node is within two hops of another PMU node}. 
For convenience, we say a PMU is cross-validated even though it is actually the PMU data at a node that is cross-validated.
A PMU is \emph{cross-validated} if one of the rules below is satisfied \cite{Vanfretti10}: 
%More formally, a PMU is \emph{cross-validated} if one of the rules below is satisfied \cite{Vanfretti10}: 
\begin{enumerate}
	
	\item {\bf Cross-Validation Rule 1 (XV1)}.  {\it If two PMU nodes are adjacent, then the PMUs cross-validate each other. }
	%Formally, if $u, v \in \Phi_G$, $u \in \Gamma(v)$, then the PMUs at $u$ and $v$ are cross-validated.}

	\item {\bf Cross-Validation Rule 2 (XV2)}. {\it If two PMU nodes have a common neighbor, then the PMUs cross-validate each other.}
	%Formally, if $u, v \in \Phi_G$, $u\neq v$ and $\Gamma(u)\cap\Gamma(v)\neq\emptyset$, then the PMUs at $u$ and $v$ are cross-validated.}
\end{enumerate}

XV1 derives from the fact that both PMUs are measuring the current phasor of the transmission line connecting the two PMU nodes.  XV2 is more subtle.  
Using the notation specified in XV2, when computing the voltage phasor of an element in $\Gamma(u)\cap\Gamma(v)$ the voltage equations include variables to 
account for measurement error (e.g., angle bias) \cite{Vanfretti-thesis}. 
When the PMUs are two hops from each other, there are more equations than unknowns, allowing for measurement error detection. 
Otherwise, the number of unknown variables exceeds the number of equations, which eliminates the possibility of detecting measurement errors \cite{Vanfretti-thesis}.





