\subsection{Garbage Collection}
\label{subsec:garbage}

After a link fails, primary tree forwarding rules may become stale. \mdrs's garbage collection routine identifies and deletes these stale flow table entries. 
Because garbage collection is not needed for correct data dissemination, garbage collection is run when necessary to free switch flow table space. 
%Note that this is not as simple as deleting all flow table entries of each primary tree using $l$, $T^l$, because a backup tree may be reusing one of these flow table entries. 

Garbage collection is straightforward, but more involved than simply deleting all flow table entries of each primary tree, $T^l$, using $l$.  Doing so would be problematic 
because a backup tree may be reusing one of these flow table entries.  To address this, \mdr maintains a dictionary, {\tt rule\_map}.  For each node, $v$, {\tt rule\_map} records
each flow table entry installed at $v$ and the multicast trees using the flow table entry. %each multicast tree (primary tree and backup trees) using $v$ to the flow table entry installed at $v$. 
When $l$ fails, the garbage collection routine determines the set of stale forwarding rules for each $T^l_i \in T^l$ by consulting  {\tt rule\_map}.
A stale rule exists at each $v \in V_i \setminus \hat{V}_i^l$ (i.e., nodes unique to the primary tree) and each $d \in D_i^l$ (nodes where the backup tree diverges from the primary tree).
%Based on this criteria, \mdr consults {\tt rule\_map} to find all stale flow table entries.
Finally, each stale forwarding rule is either explicitly removed (if using a hard-state signaling protocol \cite{Ji03}) 
or \mdr allows the forwarding rule to timeout (if using a soft-state signaling algorithm \cite{Clark88}).



%Under our \base multicast implementation, garbage collection is straightforward.  
%\mdr maintains a dictionary, {\tt rule\_map}, for each node, mapping each multicast tree (primary tree and backup trees) using $v$ to the flow table entry installed at $v$.
%When a link, $l$, fails the garbage collection routine determines the set of stale forwarding rules for each $T^l_i$. 
%A stale rule exists at each $v \in V_i \setminus \hat{V}_i^l$ (i.e., nodes unique to the primary tree) and each $d \in D_i^l$ (nodes where the backup tree diverges from the primary tree).
%The controller consults {\tt rule\_map} to find the forwarding rule at each node and either sends an explicit remove flow message to each of these nodes (if using a hard state signaling 
%scheme \cite{HardSoftState}) or allows the forwarding rule to time out (with a soft state algorithm \cite{HardSoftState}).
