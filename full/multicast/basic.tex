\subsection{Multicast Implementation} 
\label{subsec:basic}

In keeping with its role as a general framework that provides primitives for programmable networks, OpenFlow does not explicitly provide an implementation for multicast. 
Thus, we design our own multicast implementation called \bases.  \base assigns a multicast IP address to each multicast group and uses
this address to setup the flow tables at the multicast tree switches.  
  \footnote{Because multicast group membership is static for power grid applications (Section \ref{subsec:pmu-requirements}, 
  we simply determine the members of each multicast group by reading their static assignment from a text file.  Note that if dynamic group membership were to be required, we could 
  replace this static policy using a protocol like IGMP. }

After the controller computes a multicast tree (described in Section \ref{subsec:min-control}), $T_i = (V_i,E_i,r,S)$, \base installs a flow table entry at each switch in $V_i$. The flow table entry
matches packets using the group's multicast address (all other field are left as wildcards) and forwards a copy of each packet out the ports corresponding to the switch's 
outgoing links in $E_i$. If a switch in $E_i$ is adjacent to a downstream host, $h_j$, in the multicast group, then the flow table entry rewrites the destination layer 2 and 3 addresses of the 
packet copy sent to $h_j$ to $h_j$'s layer 2 and 3 addresses.
\footnote{Our initial plan was to use the group table abstraction described in the OpenFlow 1.1 specification \cite{OpenFlowSpec1.1} to implement multicast but, unfortunately,
as of the writing of this chapter, this feature is not yet supported by the POX controller \cite{Pox} used to implement our algorithms and 
the Mininet emulator \cite{Lantz10} used in our simulations.}

% (1) mcast group assigned mcast address, (2) After mcast tree computed for mcast group, use dst address to match packets at each switch.  packets fwded out the ports corresponding to
%  tree links, rewrite address at leaf switches.  (action applied for leaf's outport)

%those of its adjacent downstream host. (these fields were previously populated with the multicast addresses).




%Another way to implement multicast in OpenFlow is to leverage existing IP multicast protocols as detailed by Kotani et al. \cite{Kotani12}.  
%In this approach, the controller assigns a unique group ID to each multicast tree and creates a group table entry, that uses the group ID, at each switch along the multicast tree.  
%Meanwhile, the sender and its first-hop switch use IGMP to set up and manage the controller-generated group IDs. Finally, the sender embeds the group ID in each multicast packet's destination 
%field, allowing for each switch in the multicast tree to identify and forward multicast packets appropriately. 

