\section{Conclusions and Future Work}
\label{sec:conclude}

We addressed reliable multicasting of critical Smart Grid data.
%In summary, we addressed the problem of providing reliable multicast data dissemination of critical Smart Grid data. 
Towards this goal, we designed, implemented, and evaluated a suite of algorithms that collectively provide fast packet loss detection and fast rerouting using pre-computed backup multicast trees. 
Because this required making changes to network switches, we used OpenFlow to modify switch forwarding tables to execute these algorithms in the data plane.
%according to the rules generated at Using SDN abstractions

%First, we presented \pcnts, an algorithm that uses OpenFlow primitives to tag and count individual packets, allowing for accurate packet loss detection of individual links at small time scales.
First, we presented \pcnts, an algorithm that used OpenFlow primitives to accurately detect per-link packet loss inside the network rather than using slower E2E measurements. % at small time scales.
%OpenFlow packet counters and ability to mark individula packets to provide  accurate packet loss detection at small time scales. 
Next, we formulated a new problem, \mcs, that considered computing
backup trees that reuse edges of already installed multicast trees as a means to reduce control plane signaling.  \mc was proved to be at least NP-hard
so we designed an approximation algorithm called \steiners.  Lastly, we presented two algorithms, \pre and \posts, that installed backup trees at OpenFlow controlled switches. 
As an optimization to \pre and \posts, 
we introduced \merges, an algorithm that consolidated forwarding rules at switches where multiple trees had common children.
%we introduced \merges, an algorithm built on the observation that forwarding rules can be consolidated at switches where multicast trees have common children.  
%When applied to \pres, \merge reduce the amount of pre-installed control state and with \post results in fewer control messages. 


Mininet simulations were used to evaluate our algorithms over communication networks that mirrored the structure of IEEE bus systems (actual portions of the North American power grid).
We found \pcnt estimates were accurate when monitoring even a small number of flows over short time window: after sampling only $75$ packets, the $95\%$ confidence interval of \pcnt loss estimates 
were within $15\%$ of the true loss probability. By pre-installing 
backup trees, \pre resulted in up to a ten-fold decrease in control messages compared with \posts, which had to signal multiple switches to install each backup tree. 
However, in scenarios with many multicast groups, \pres's pre-installed forwarding rules accounted for a significant portion of scarce OpenFlow switch table capacity (up to $35\%$ of
a standard OpenFlow switch). Fortunately, \merge reduced the amount of pre-installed forwarding state by a factor of $2-2.5$, to acceptable levels.

As future work, several problems remain unaddressed. One problem of interest is using optimization criteria different from \mcs's objective function to compute backup trees and then evaluate
\pres, \posts, and \merge performance using these backup trees.  For example, backup trees may be computed with the goal of protecting against the worst-case impact of a subsequent link failure
by minimizing the maximum number of multicast trees using a single link. %These backup tree help protect against the worst-case impact of a subsequent link failure. 
It is unknown how effective our installation algorithms would be given these types of backup trees. % when given backup trees other than those computed using \steiners.
Measurements using real OpenFlow hardware switches would strengthen our \pcnt processing time and backup tree installation time results, which both suffered from inaccuracies due to Mininet's
performance fidelity issues.  
%Lastly, the complexity of the problem \merge addresses is an open-question: find the minimum number of forwarding rules for a set of multicast trees. 
Lastly, the problem \merge addresses -- find the minimum number of forwarding rules for a set of multicast trees -- has unknown complexity.
We conjectured that this problem is NP-hard in Section \ref{subsubsec:merge-discuss}.


%The ability to measure packet loss of individual flowsThe novelty and benefits of these algorith%m
%These algorithms are tailored to support reliable multicasting of critical Smart Grid data.  The algorithms were 
%We designed, implemented, and evaluated a suite of algorithms that collectively provide fast detection of packet loss and re-routing 

%In conclusion, we proposed a suite of algorithms that aim to make multicast communication robust to link failures by detecting link failures at small timescales, pre-computing
%backup multicast trees that aim to minimize control signaling needed to install them, and lastly installation algorithms that leverage common forwarding among multicast trees 
%to reduce the amount of installed forwarding state.  

% (a) link failures and 3 algorithm parts., (b) Algorithms benefit from OF impl, detect link failure inside the netwokr, pre-compute backup  differ from literature b/c use SDN,
% (c) results -- link failure is accuracte so can avoid overhead; less control overhead, garbage collection, and faster recoveyr when preinstall.  Merger is affective optimization.

%{\it \underline{Future Work Notes}:
%Future Work: (a) other optimization criteria for computing backup trees (e.g., Min-Flows, may even want to include problem definition in appendix), (b) real measurements 
%for \pcnt detection time and time to install backup trees (b/c of Mininet performance fidelity issues), (c) establish complexity of finding minimum number of forwarding rules,
%which we conjectured to be NP-hard.
%}
