\section{Introduction}
\label{sec:intro}


An electric power grid consists of a set of buses  -- electric substations, power generation centers, or aggregation points of electrical loads -- and transmission lines connecting those buses.
The operation of the power grid can be greatly improved by high-frequency voltage and current measurements. Phasor Measurement Units (PMUs) are  
sensors that provide such measurements. PMUs are currently being deployed in electric power grids worldwide, providing the potential to both 
(a) drastically improve existing power grid operations and applications and (b) enable an entirely new set of applications,
such as real-time visualization of electric power grid dynamics and the reliable integration of renewable energy resources. 

PMU applications have stringent and in many cases ultra-low \emph{per-packet} delay and loss requirements.  
If these per-packet delay requirements are not met, PMU applications can miss a critical power grid event (e.g., lightning strike, power link failure), potentially leading to a 
cascade of incorrect decisions and corresponding actions. For example, closed-loop control applications require delays of $8-16$ ms per-packet \cite{Bakken11}. 
If \emph{any} packet is not received within this time window, the closed-loop control application may take a wrong control action.
In the worst case, this can lead to a cascade of power grid failures similar to the August 2003 blackout in the USA 
\footnote{\url{http://en.wikipedia.org/wiki/Northeast_blackout_of_2003}} and the recent power grid failures in India \cite{IndiaBlackout}. 


As a result of this sensitivity, the communication network that disseminates PMU data must provide hard end-to-end data delivery guarantees \cite{Bakken11}. 
For this reason, the Internet's best-effort service model alone is unable to meet the stringent packet delay and loss requirements of PMU applications \cite{Birman05}. 
Instead, either a new network architecture or enhancements to the Internet architecture and its protocols are needed \cite{Bakken11,Birman05,Naspi10,Hopkinson09} to provide efficient, in-network forwarding and fast recovery from link and switch failures. 
Additionally, multicast should figure prominently in data  delivery, since PMUs disseminate  data  to applications across many locations \cite{Bakken11}.

We design algorithms for fast recovery from link failures in a Smart Grid communication network. 
Informally, a link that does not meet its packet delivery requirement (either due to excessive delay or actual packet loss) is considered failed. 
We propose, design, and evaluate solutions to all three aspects of link failure recovery: link failure detection, algorithms for pre-computing backup multicast trees, 
and fast backup tree installation. 

We make the following contributions in this paper: 
\begin{itemize}

	%\item {\bf Link-failure detection algorithm}. 
	\item {\bf Design a link-failure detection algorithm}. 
	We design a link-failure detection and reporting mechanism, \pcnts, that uses OpenFlow \cite{OpenFlow08} -- an open source framework 
	that centralizes network management and control -- to detect link failures when and where they occur, \emph{inside} the network.  In-network detection is used to reduce the 
	time between when the loss occurs and when it is detected. In contrast, most previous work \cite{Almes99,Caceres99,Friedl09} focuses on measuring end-to-end packet loss, 
	resulting in slower detection times. 

	%Using Mininet we show ...

	\item {\bf Formulate a novel optimization problem for computing backup multicast trees.} 
	Inspired by MPLS fast-reroute algorithms that quickly reroute time-critical unicast IP flows over pre-computed backup paths \cite{Rosen01}, 
	we formulate a new problem, \mcs, that pre-computes backup multicast trees, to be used after a link failure, with the aim 
	of minimizing the control overhead required to install the backup trees. 
	 This optimization criteria differs from those proposed in the literature \cite{Cui04,Fei01,Medard99,Pointurier02,Wu97} that use optimization
 	criteria specified over a \emph{single} multicast tree, while we consider conditions specified across \emph{multiple} multicast trees. 

	\item {\bf Prove \mc is at least NP-hard and propose an approximation algorithm, \steiners, for \mcs.} 
	\steiner uses an approximation to the directed Steiner Tree problem taken from the literature \cite{Charikar98} to compute backup trees and modifies link weights to 
	encourage backup trees to reuse existing forwarding rules installed in the network. Doing so reduces both the number of control messages the controller must send to install each backup tree
	and the number of forwarding rules maintained at each switch.
	%Using a copy of the network graph, \steiner modifies link weights to encourage its directed Steiner Tree computation  reuse of existing forwarding rules installed in the network,
	%thereby reducing the number of control messages the controller must send to install each backup tree.

	\item {\bf Propose \merges, an OpenFlow implementation of multicast that aims to reduce forwarding state.}  \merge uses local optimization to create a near minimal set of forwarding rules 
	by ``merging'' forwarding rules in cases where multiple multicast trees have common forwarding behavior.

	\item {\bf Design two algorithms -- \pre and \post -- for fast backup tree installation.}  \pre pre-installs backup tree forwarding rules and
	activates these rules after a link failure is detected, while, \post installs backup trees \emph{after} a link a failure is detected.  
	We show how \merge can be applied to \pre and \post to reduce the amount of \pre pre-installed forwarding state and decrease \post signaling overhead. 

	\item {\bf Provide a prototype implementation of our algorithms, \mdrs, using POX and 
	evaluate each algorithm using Mininet.} \pcnts, \steiners, \merges, \pres, and \post are implemented in POX \cite{Pox}, 
	an open-source OpenFlow controller. 
	
	We use simulations based on the Mininet emulator \cite{Lantz10} to evaluate our algorithms over synthetic graphs and actual IEEE bus systems.
	We find that \pcnt provides fast and accurate link loss estimates: after sampling only $75$ packets
	the $95\%$ confidence interval is within $15\%$ of the true loss probability.  Additionally, we find \pre yields faster recovery than \post -- 
	\post sends up to $10$ times more control messages than \pre -- but at the cost of storage overhead at each switch (in our simulations, pre-installed backup trees can account for as much
	as $35\%$ of the capacity a conventional OpenFlow switch \cite{Curtis11}).
	Lastly, we observe that \merge reduces control plane messaging and the amount of pre-installed forwarding state by a factor of $2$ to $2.5$ when compared to a standard multicast
	implementation, resulting in faster installation and manageable sized flow tables.
	% may want to better estimate how many PMUs and multicat trees is reasonable


		
\end{itemize}

The remainder of this paper is structured as follows.  In the following section (Section \ref{sec:prelim}), we provide necessary background on PMU application requirements and OpenFlow, as well
as introduce a running example later used to describe our algorithms.  
Then, we outline our algorithms in Section \ref{sec:algs}:
Section \ref{subsec:pcnt} details our link-failure detection algorithm called \pcnts; in Section \ref{subsec:min-control}, we outline our algorithms for computing backup multicast trees; and
then describe algorithms for installation backup trees in Section \ref{subsec:install-backups}; Section \ref{subsec:merge} presents \merges, a fast multicast implementation 
when applied to our backup tree installation algorithms can significantly improve performance.
Next, we briefly survey relevant literature (Section \ref{sec:related}).
Our simulation study is presented in Section \ref{sec:evaluation}. Section \ref{sec:conclude} concludes the paper with a summary and directions for future work.

