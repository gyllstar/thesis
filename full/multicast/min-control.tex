\subsection{Computing Backup Trees}
\label{subsec:min-control}

\mdr pre-computes backup trees to install after \pcnt detects a link failure.  Here we present a new problem, \mcs, and an approximate solution to \mcs, called \steiners, that aim to facilitate fast
recovery by computing backup trees that maximize the number of edges common between each backup tree and its primary tree.  This reuse of primary tree edges speeds recovery from link
failures because, in SDN, this reduces the number of new flow table entries that need to be installed in network routers in response to a link failure.

\mdr uses \steiner as a part of system initialization, where the set of backup trees are computed for each network link, $l$; \mdr computes a single backup tree for each primary tree using $l$. 
Additionally, \steiner is used after a set of backup trees, $\hat{T}^l$, are installed in response to a link failure.  For each newly installed tree $\hat{T}^l_i \in \hat{T}^l$, \mdr computes 
a backup tree for each link in $\hat{T}^l_i$.
%recovery by minimizing control overhead (i.e., number of new flow table entries) required to install backup trees.  

%In SDNs, reusing forwarding rules which under SDN reduces the number of new flow table entries required to install backup trees.  

%\mdr uses \steiner in two scenarios. One, as a part of system 
%initialization where a set of backup trees are computed for each network link, $l$; \mdr computes a single backup tree for each primary tree using $l$.  Two, \mdr triggers the execution of 
%backup tree computations after backup trees, $T_l$, are installed in response $l$'s failure. \mdr recomputes backup trees for all primary trees as opposed to only the newly install $T_l$ trees.
%The motivation for recomputing all backup trees is that the network state -- the set of installed forwarding rules -- is 
% the entire computation (as opposed to only computing backup trees for the newly installed
%Tl trees): for each network link, l, Appleseed computes a backup MT for each MT that would be affected if l failed. Appleseed recomputes all backup MTs because installing the Tl MTs changes how flows are distributed and processed inside the network and, as a result, may adversely affect flows processed by MTs other than those in Tl. In other words, backup MTs computed before l fails may become stale when the Tl MTs are installed since the algorithms used to compute them considered an old network state in their optimization. 

\subsubsection{\mcn Problem}
\label{subsubsec:min-control}

% Existing Outline:  (a) control overhead definition + example, (b) multiple PTs need entry for each tree (c) benefits of reducing control overhead, (d) problem definition, (e) NP-hard
% Revised Outline:  (a) informal control overhead + benefits, (b) control overhead definition, (c) example, (d) problem statement, (e) NP-hardness outline
%control overhead definition + example, (b) multiple PTs need entry for each tree (c) benefits of reducing control overhead, (d) problem definition, (e) NP-hard 
% make sure handle 1 message per flow assumptio

The goal of the \mc problem is to compute backup trees that maximize reuse of primary tree edges.  
Recycling primary tree edges allows the SDN controller, when generating the forwarding rules for multicasting 
packets using the backup tree, to use primary tree rules already installed in the network rather than install new ones.
%Recycling primary tree rules reduces the number rule installation messages, referred to as control messages, the controller must send network switches. 
This speeds recovery in cases where backup trees are installed \emph{after} a link failure is detected and reduces the number of flow table entries pre-installed at switches (control state) 
when backup trees are installed \emph{before} a link failure occurs.  
Reducing control state is especially important with OpenFlow because OpenFlow switches can only store a limited number of flow table entries (see Section \ref{subsec:openflow}).
\footnote{Following from our assumption that a single link fails at-a-time, \mc assumes that all other links besides the failed one, $l$, satisfy packet loss requirements. }
%\footnote{Our \mc description implies that all other links besides the failed one, $l$, satisfy packet loss requirements. This follows from our assumption that only a single link fails at-a-time.}

For the primary tree $T^l_i = (V^l_i,E^l_i,r,S)$ and its backup $\hat{T}^l_i=(\hat{V}^l_i,\hat{E}^l_i,r,S)$, we define a binary variable $c_v^l$ for all $v \in \hat{V}_i^l$. 
If $v$ has exactly the same predecessors (outgoing edges) in $T^l_i$ and $\hat{T}^l_i$, then $c_v^l$ takes value $0$.  Otherwise, $c_v^l=1$.  
%Using  $c_v^l$, define: %Define multicast recycling, for the $T^l_i$,$\hat{T}^l_i$ pair as:
For the $T^l_i$,$\hat{T}^l_i$ pair define:
\begin{eqnarray}
\label{eqn:control-overhead}
 C_i^l &=&  \sum_{\forall v \in \hat{V}_i^l} c_v^l 
\end{eqnarray}
For our purposes, $C_i^l$ is the number of new rules (i.e., non-recycled primary tree rules) needed to install $\hat{T}^l_i$.  Note that the primary tree rules not recycled by 
$\hat{T}^l_i$ should be deleted after $l$ fails and $\hat{T}^l_i$ is installed, especially considering the limited size of OpenFlow switch flow tables 
(Section \ref{subsec:openflow}).  As we describe in Section \ref{subsec:garbage}, these rules can be garbage collected in the background because
stale primary tree forwarding rules do not affect how packets are (correctly) forwarded by $\hat{T}^l_i$. 
For this reason, \mc aims to minimize the number of new rules needed to install a backup tree, rather than the number of primary tree rules that must be garbage collected after a link failure.
%For this reason, minimizing the number of newly installed rules is more critical than minimizing the primary tree rules that need to be garbage collected.

Consider the example in Figure \ref{fig:intuition-example} where $(g,l)$ fails. The green backup tree, $\hat{T}_b$, shown in Figure \ref{fig:intuition-example-t2}, has $C_b =2$ because 
a new forwarding rule is required at $b$, and $f$ to account for the new outgoing links at each node. $\hat{T}_c$, in blue, has
only one link, $(m,l)$ not in $\hat{T}_c$'s primary tree. As a result, $C_c =3$.

Our \mc problem definition below references a modified version of the Steiner tree problem, called the \arbor problem \cite{Charikar98}. As input, \arbor is given 
a directed graph $G=(V,A)$, a root vertex $r$, and a set of terminals, $S$. An arborescence is defined as a tree rooted at $r$ that has directed edges spanning $S$. 
\arbor aims to find a minimum cost arborescence, called a Steiner arborescence or directed Steiner tree. We denote $SA_i(G) = (V,E,r,S)$ as the Steiner arborescence computed over 
directed graph, $G$, rooted at $r$, and spanning $S$ such that $r, S \in T_i$. % $T_i = (V,E,r,S)$.
%and corresponds to $T_i = (V,E,r,S)$.

We formulate the \mc problem as follows: 
\begin{itemize}

	%\item  \underline{Input}: $(G,T,l)$ where $G=(V,E)$ is directed graph, $T=\{T_1,T_2, \dots T_m\}$ where each $T_i \in T$ is a primary tree, and $l \in E$.
	\item  \underline{Input}: $(G,T^l,l,\alpha)$ where $G=(V,E)$ is a directed graph, $T^l=\{T_1^l,T_2^l, \dots T_k^l\}$ where each $T_i^l \in T^l$ is a primary tree that uses $l$, 
	$l \in E$, and $\alpha \geq 1$. 
	%computed using the directed Steiner tree approximation algorithm described by Charikar et al. \cite{Charikar98}, and $l \in E$.


	\item \underline{Output}: A backup tree for each primary tree using $l$. This set of backup trees, $\hat{T}^l = \{\hat{T}^l_1,\hat{T}^l_2,\dots,\hat{T}^l_k\}$:
		\begin{equation}
		\label{eqn:mc-obj-function}
		\begin{aligned}
			& {\text{minimize}}
			& & \sum_{1 \leq i \leq k} C^l_i \\
			& \text{subject to}
			& & w(\hat{T}^l_i) \leq \alpha \cdot w(SA_i(G')), \;  \forall \hat{T}^l_i \in \hat{T}^l \\
		\end{aligned}
		\end{equation}
		where $G'=(V',E')$ such that $E' = E - \{l\}$ and $w(\hat{T}^l_i)$ is the sum of $\hat{T}^l_i$'s link weights.	
		\footnote{We assume $\hat{T}^l_i$, satisfies all per-packet delay and loss requirements if $l \notin \hat{T}^l_i$ and $w(\hat{T}^l_i) \leq \alpha \cdot w(SA_i(G'))$}

\end{itemize}   % (a) assumes all other links up, (b) C^l defined between pt and bt, point shall revisit later in approx, (c) \alpha keeps tree from growing too larger
The objective function maximizes the reuse of primary tree edges, while $\alpha$ bounds how large the backup tree can grow as consequence of minimizing $C^l_i$.  When applied to our
problem scenario this formulation reduces the number of installation rules by reusing rules already installed in the network, under the constraint that the 
backup tree does not become too large to meet the end-to-end latency requirements. 
By defining $G'$ as a copy of $G$ with the failed link removed from $G$, we are assuming that all links in $G$ besides $l$ are operational.  For
our purposes, this amounts to assuming that all non-$l$ links have packet loss rates less than their threshold. %below the satisfy their packet loss rate condition.  

Notice that we have defined $C^l_i$ in Equation \ref{eqn:control-overhead} on a per-backup tree basis where for backup tree $\hat{T}^l_i$, $C^l_i$ is a relationship defined strictly between 
$\hat{T}^l_i$ and its primary tree $T^l_i$ (there are no constraints specified across any other primary or backup tree). 
%\footnote{This simplification is reasonable for our OpenFlow-based application of \mcs. In cases where multiple backup trees require a new forwarding rule at the same switch, $v$, a separate
%forwarding rule must be installed (even if these backup trees have exactly the same predecessors at $v$) for each of these backup trees because under our multicast implementation, 
%\bases, each forwarding rule matches packets using the tree's multicast address. }
As a result, the globally optimal solution for \mc  (i.e., the optimal set of backup trees for a single link)  can be found by computing the optimal backup for each primary tree in isolation
and then taking the union of these solution We shall revisit this important property when describing our approximation algorithm for \mcs.

\begin{theorem}
\label{thm:mc-npc}
\mc is at least NP-hard.
\end{theorem}
\begin{proof}
The details of our proof can be found in Appendix \ref{sec:mc-npc}. This proof shows that \mc is NP-hard even when considering just a single backup tree.  The proof demonstrates that
in some cases an optimal solution to \mc requires a solution to \arbors, a problem known to be NP-hard.  This proves \mc is NP-hard when considering a single backup tree and therefore
the general \mc problem for $k$ backup trees must at least be NP-hard. 
\end{proof}


%to derive a globally optimal solution for \mcs.  
%Notice that we have defined the control overhead as the difference between the directed edges from the backup tree and its corresponding primary tree.  This formulation
%is convenient because for a given link, $l$, we can solve \mco seperately for each directed tree using $l$ and then simply take the union of these solutions to derive a 
%globally optimal solution for \mc (for $l$).  




\subsubsection{\steinern Approximation Algorithm}
\label{subsubsec:steiner-approx}

\steiner is a simple approximation algorithm for \mc that manipulates link weights to encourage each backup tree to reuse primary tree edges.
For each link $l$,  \steiner separately computes a backup tree for each primary tree using $l$ and then returns the union of these computed trees. 

%Our \mc approximation algorithm, \steiners, computes each backup tree for link $l$ separately and then returns the union of these computed trees. 
%\steiner manipulates link weights to encourage each backup tree to reuse primary tree edges. 

%\steiner leverages the \arbor approximation algorithm from Charikar et al. \cite{Charikar98}.  Their approximation algorithm computes a bunch: a subgraph 
%formed by taking the shortest path from the root to an intermediate vertex, $j$, and the union of shortest paths from $j$ to the terminal nodes.  
\steiner leverages the $\sqrt{s}$ \arbor approximation, where $s$ is the number of terminal nodes, from Charikar et al. \cite{Charikar98}. Their approximation algorithm computes bunches, 
where a \emph{bunch} is a subgraph formed by taking the shortest path from the root to an intermediate vertex, $i$, and the union of shortest paths from $i$ to the terminal nodes.  
The algorithm produces the bunch with best \emph{density} -- density is the average cost of connecting a terminal node with the root -- as its approximation.   The lowest density bunch can 
easily be computed in polynomial time: a brute-force approach that tries all possible nodes as the intermediate vertex yields an $O(ns^2 \log s)$ time algorithm. 
%where $s$ is the number of terminal nodes.  The authors prove this is a $\sqrt{s}$ approximation.


%\steiner leverages the \arbor approximation algorithm from Charikar et al. \cite{Charikar98}.  Their approximation algorithm recursively finds ``bunches'', where a bunch is a subgraph 
%formed by taking the shortest path from the root to an intermediate vertex, $j$, and the union of shortest paths from $j$ to a subset of the terminal nodes.  
%Th%e overall solution is the union of each bunch.  %We denote $B_i(G) = (V,E,r,S)$ as the $ith$ Steiner arborescence computed over directed graph, $G$, using this algorithm.
%\footnote{In our work, the initial set of primary trees (i.e., the primary trees used before any link failures) are computed using the Steiner arborescence approximation from \cite{Charikar98}.}

%Our approximation algorithm for \mc solves computes for each $T_i^l \in T^l$ and then returns the union of these results. 
%We leverage the Steiner arborescence approximation described in \cite{Charikar98} to approximate \mcos.  Abusing notation, we refer to the \mco primary tree as $T^l$ and its backup as $\hat{T}^l$.  

Given $(G,T^l,l,\alpha)$, for each $T_i^l \in T^l$ \steiner uses the following two-step procedure to compute $\hat{T}^l_i$:
\begin{enumerate}
	
	\item Make a copy of $G$ called $G'=(V',E')$ and remove $l$ from $E'$.  Set the link weight of each $e \in T^l_i$ to $0$ and the link weight of $e \notin T^l_i$ to $1$. 

	\item Run the \arbor approximation, using the brute-force approach described above, over $G'$ and set $\hat{T}^l_i$ to be the result.
	If $\hat{T}^l_i$ satisfies the Equation \ref{eqn:mc-obj-function} constraint, return $\hat{T}^l_i$ as the solution.  Otherwise, return False. 
	%Otherwise, compute the backup tree from scratch -- run the \arbor approximation 
	%over a copy of $G$ that uses $G$'s original link weights and has link $l$ removed -- and return this result.

\end{enumerate}
Setting the primary tree link weights to $0$ in Step (1) allows the \arbor approximation algorithm to use any primary tree edge without penalty (i.e., adding cost to the 
backup tree) and so encourages reusing primary tree edges.  If \steiner returns False in Step (2) either $\alpha$ must be made larger or a new multicast tree should be computed from
scratch that satisfies the tree-size constraint.

In Figure \ref{fig:intuition-example}, \steiner uses $f$ as the the intermediate node for $\hat{T}_b$, yielding density of $2$ (the cost of connecting terminals $p$,$q$,$r$, and $s$ to the 
root is $2$).   \steiner selects $f$ as the intermediate node by iterating over all nodes and remembering the node with the smallest density. $g$ or $l$ could have been used
as the intermediate node because both, like $f$, have density of $2$ ($f$ is selected arbitrarily using a tiebreaker).
The bunch for $\hat{T}_c$ is formed using $m$ as the intermediate node with density $0.4$: the cost of connecting $r$ and $s$ to the root is $1$ and $t$,$u$, and $v$ connect
with the root at $0$ cost. 


