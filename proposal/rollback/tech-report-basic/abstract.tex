%\section{Abstract}

\begin{abstract}

Malicious and misconfigured nodes can inject incorrect state into a distributed system, which can then be propagated system-wide as a result of normal network operation. 
Such false state can degrade the performance of a distributed system or render it unusable. For example, in the case of network routing algorithms, false state corresponding
to a node incorrectly declaring a cost of $0$ to all destinations (maliciously or due to misconfiguration) can quickly spread through the network. This causes other nodes to (incorrectly) 
route via the misconfigured node, resulting in suboptimal routing and network congestion. We propose three algorithms for efficient recovery in such scenarios, prove the correctness 
of each of these algorithms, and derive communication complexity bounds for each algorithm. Through simulation, we evaluate our algorithms -- in terms of message and time overhead -- when 
applied to removing false state in distance vector routing. Our analysis
shows that over topologies where link costs remain fixed and for the same topologies where link costs change, a recovery algorithm based on system-wide checkpoints and a rollback mechanism 
yields superior performance when using the poison reverse optimization.

\end{abstract}
%We propose algorithms that allow distributed routing algorithms to efficiently recover from false state injected into the network.
%We prove the correctness of each algorithm and evaluate them when applied to distance vector routing. In this context, we evaluate the message and time complexity of each algorithm through simulation.

%{\bf Keywords}: {\it distributed algorithms, fault tolerance, routing, security}
