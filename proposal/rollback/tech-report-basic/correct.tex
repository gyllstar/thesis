%%%%%%%%%%%%%%%% TRENDS IN RECOVERY SECTION %%%%%%%%%%
\section{Correctness of Recovery Algorithms}
\label{sec:correct}

Our correctness proofs consider the general case where multiple nodes are compromised. We use the following notation in our proofs:
\begin{itemize}
	\item We refer to the set of compromised nodes as $\overline{V}$. 
	\item $t_b$ marks the time at outside algorithm detects that all $\overline{V}$ are compromised. 
	\item $t'$ refers to the time the first $\overline{v} \in \overline{V}$ is compromised. 
	\item $t^*$ marks the time when the recovery algorithm (e.g., \seconds, \purges, or \cprs), which started executing at time $t$, completes.
	\item We use the definition of $G$ described in Section \ref{sec:algs}.
	\item We redefine $G'$ as follows. $G'=(V',E')$, where $V' = V - \overline{V}$, $E'=E - \{(\overline{v},v_i)$ $|$ $\overline{v} \in \overline{V} \wedge v_i \in adj(\overline{v}) \}$.
\end{itemize}
All important timesteps are shown in Figure \ref{fig:timeline}.


\begin{figure}
\begin{center}
\begin{picture}(200,55)
\put(0,25){\vector(1,0){220}}  % pi and to the right
\put(0,25){\vector(-1,0){20}}  % pi and to the right

%tics
\put(20,22){\line(0,1){6}} 
\put(70,22){\line(0,1){6}} 
\put(120,22){\line(0,1){6}} 
\put(180,22){\line(0,1){6}}

%\put(25,25){\vector(1,0){50}} % pi and to left
%\put(25,25){\line(1,0){50}} % pi and to left
\put(20,11){\makebox(0,0)[b]{$t'$}}
\put(70,11){\makebox(0,0)[b]{$t_b$}}
\put(120,11){\makebox(0,0)[b]{$\hat{t}$}}
\put(185,11){\makebox(0,0)[b]{$t^*$}}

\put(20,43){\makebox(0,0){{\footnotesize $\overline{v}$}}}
\put(20,35){\makebox(0,0){{\footnotesize  compromised}}}
\put(70,43){\makebox(0,0){{\footnotesize $\overline{v}$}}}
\put(70,35){\makebox(0,0){{\footnotesize detected}}}
\put(120,43){\makebox(0,0){{\footnotesize diffusing}}}
\put(120,35){\makebox(0,0){{\footnotesize comp. complete}}}
\put(185,43){\makebox(0,0){{\footnotesize recovery alg.}}}
\put(185,35){\makebox(0,0){{\footnotesize complete}}}


\end{picture}
\end{center}
\caption{Time line with important timesteps labeled.}
\label{fig:timeline}
\end{figure}

%We first state our assumptions and then define correctness. Finally, we prove the correctness of each algorithm. 
We make the following assumptions in our proofs. All the initial \dmatrix values are non-negative. Furthermore, all \minv values periodically
exchanged between neighboring nodes are non-negative. 
All $v \in V$ know their adjacent link costs. All link weights in $G$ (and therefore $G'$ as well) are non-negative and do not change.
$G$ is finite and connected. Finally, we assume reliable communication. 

\begin{define} 
An algorithm is correct if the following two conditions are satisfied. One, $\forall v \in V'$, $v$ has the least cost to all destinations $v' \in V'$.
Two, the least cost is computed in finite time.
\end{define}


\begin{theorem}
\label{thm:dv-correct}
Distance vector is correct.
\end{theorem}
\begin{proof}
Bertsekas and Gallager \cite{Gall87} prove correctness for distributed Bellman-Ford for arbitrary non-negative \dmatrix values. 
Their distributed Bellman-Ford algorithm is the same as the distance vector algorithm used in this paper. 
\end{proof}
%$Proof$. Bertsekas and Gallager \cite{Gall87} prove correctness for distributed Bellman-Ford for arbitrary nonnegative \dmatrix values. 
%Their distributed Bellman-Ford algorithm is the same as the distance vector algorithm used in this paper. \qed

\begin{corollary}
\label{cor:second-correct-single}
\second is correct when a single node is compromised.
\end{corollary}
\begin{proof}
As per the preprocessing step, each $v \in adj(\overline{v})$ initiates a diffusing computation to remove \bad as a destination. For each diffusing computation,
all nodes are guaranteed to receive a diffusing computation (by our reliable communication and finite graph assumptions). Further, each diffusing computation
terminates in finite time.  Thus, we conclude that each $v \in V'$ removes $\overline{v}$ as a destination in finite time.

After the diffusing computations to remove \bad as a destination complete, each 
$v \in adj($\bads$)$ uses distance vector to determine new least cost paths to all nodes in their connected component.
Because all \dmatrixv are non-negative for all $v \in V'$, by Theorem \ref{thm:dv-correct} we conclude \second is correct if no additional node(s) are compromised
during $[t',t^*]$. % the execution of \seconds. %trigger by the compromised of $\overline{V}$
\end{proof}

\begin{corollary}
\label{cor:second-correct-mult}
\second is correct when multiple nodes are compromised.
\end{corollary}
\begin{proof}
If multiple nodes, $\overline{V}$, are simultaneously compromised the proof is the same as that for Corollary \ref{cor:second-correct-single}, substituting $\overline{V}$ for $\overline{v}$.

Next, we prove \second is correct in the case where a set of nodes, $\overline{V}_2$, are compromised concurrent with a running execution of \second (e.g., during  $[t',t^*]$), triggered by the compromise of $\overline{V}$.
First we show that any least cost computation (e.g., one triggered by  $\overline{V}$'s compromise) to any $v \in \overline{V}_2$ is eventually terminated.  We have already proved that the diffusing computations to remove each $v \in \overline{V}_2$ as 
a destination complete in finite time.  Let $t_d$ mark the time these diffusing computations complete. For all $t \geq t_d$, any running least cost computation to a destination $v \in \overline{V}_2$ 
is terminated by the actions specified in Section \ref{subsec:mult}.  Therefore, the only remaining least cost computations are to all $v \in V'$, where 
$V' = V - \left(\overline{V} \cup \overline{V}_2 \right)$.  Because all \dmatrixi values are non-negative for all $i \in V'$, by Theorem \ref{thm:dv-correct} we conclude \second is correct.
%when additional nodes are compromised concurrent with any \second execution. 
%We have already proved (in Corollary \ref{cor:second-correct-single}) that the diffusing computations to remove $\overline{V}_2$ as a destination complete in finite time.  

Since we have proved \second is correct when multiple nodes are simultaneously compromised and when nodes are compromised concurrent with any \second execution,
we conclude that \second is correct when multiple nodes are compromised.
\end{proof}


\begin{corollary}
\label{cor:purge-correct-single}
\purge is correct when a single node is compromised.
\end{corollary}
\begin{proof}
Each $v \in adj($\bads$)$ finds every destination, $a$, to which $v$'s least cost path 
uses \bad as the first-hop node. $v$ sets its least cost to each such $a$ to $\infty$, thereby invalidating its path to $a$.  
$v$ then initiates a diffusing computation. When an arbitrary node, $i$, receives a diffusing computation message from $j$, $i$ iterates 
through each $a$ specified in the message. If
$i$ routes via $j$ to reach $a$, $i$ sets its least cost to $a$ to $\infty$, therefore invalidating any path to $a$ with $j$ and  
\bad an intermediate nodes.  

%By our assumptions, each node receives a diffusing computation message and the diffusing computation terminates in finite time. Thus, we conclude that all paths using \bad as an intermediary
By our assumptions, each node receives a diffusing computation message for each path using $\overline{v}$ as an intermediate node.  Additionally, our assumptions imply that all
diffusing computation terminate in finite time. 
Thus, we conclude that all paths using $\overline{v}$ as an intermediary node are invalidated in finite time. 

Following the preprocessing, all $v \in adj($\bads$)$ use
distance vector to determine new least cost paths. Because all \dmatrixi are non-negative for all $i \in V'$, by Theorem \ref{thm:dv-correct} we conclude that \purge is correct. 
\end{proof}

\begin{corollary}
\label{cor:purge-correct-mult}
\purge is correct when multiple nodes are compromised.
\end{corollary}
\begin{proof}
The same proof used for Corollary \ref{cor:second-correct-mult} applies for \purges.
\end{proof}

\begin{corollary}
\label{cor:cpr-correct-single}
\cpr is correct when a single node is compromised.
\end{corollary}
\begin{proof}
\cpr sets $t'$ to the time $\overline{v}$ was compromised. Then, \cpr rolls back using diffusing computations: each diffusing computation is initiated at each $v \in adj(\overline{v})$.
Each node that receives a diffusing computation message, rolls back to a snapshot with timestep less than $t'$.
By our assumptions, all nodes receive a message and the diffusing computation terminates in finite time.  Thus, we conclude
that each node $v \in V'$ rolls back to a snapshot with timestamp less than $t'$ in finite time.

\cpr then runs the preprocessing algorithm described in Section \ref{subsec:preprocess}, which removes each \bad as a destination in finite time (as shown
in Corollary \ref{cor:second-correct-single}). Because each node rolls back to a snapshot in which all least costs are non-negative and 
\cpr then uses distance vector to compute new least costs, by Theorem $1$ we conclude that \cpr is correct if no additional nodes are compromised during  $[t',t^*]$.
\end{proof}

\begin{corollary}
\label{cor:cpr-correct-mult}
\cpr is correct when multiple nodes are compromised.
\end{corollary}
\begin{proof}
If multiple nodes, $\overline{V}$ are simultaneously compromised, \cpr sets $t'$ to the time the first $\overline{v} \in \overline{V}$ is compromised.  Any nodes, $\overline{V}_2$, compromised
concurrent with $\overline{V}$ (e.g., during  $[t',t^*]$), trigger an additional \cpr execution.  The steps described in Section \ref{subsec:mult} ensure 
that all least cost computations (after rolling back) are to
destination nodes $a \in V'$.  By Theorem \ref{thm:dv-correct} we conclude \cpr is correct because all \dmatrixi are non-negative for all $i \in V'$.
\end{proof}







