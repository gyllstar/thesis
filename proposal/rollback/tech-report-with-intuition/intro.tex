\section{Introduction}
\label{subsec:rb-intro}

%Distributed systems are vulnerable to malicious and misconfigured nodes which inject false system state. 

Malicious and misconfigured nodes can degrade the performance of a distributed system by injecting incorrect state information. Such false state can then be further propagated 
through the system either directly in its original form or indirectly, e.g., as a result of diffusing computations initially using this false state. 
For example, consider distance vector routing.  If a compromised node incorrectly claims a distance of $0$ to all destinations, this false state would likely
spread throughout the network, infecting shortest paths network-wide.
%For example, with distance vector a compromised node incorrectly claiming a distance of $0$ to all destinations would likely infect shortest paths throughout the network.
In this paper, we consider the problem of removing such false state. % from a distributed system.

In order to make the false-state-removal problem concrete, we investigate distance vector routing as an instance of this problem. Distance vector forms the basis for many routing 
algorithms widely used in the Internet (e.g., BGP, a path-vector algorithm) and in multi-hop wireless networks (e.g., AODV, diffusion routing). However, distance vector is vulnerable 
to compromised nodes that can potentially flood a network with false routing information, resulting in erroneous least cost paths, packet loss, and congestion. Such scenarios have occurred
in practice. For example, in 1997 a significant portion of Internet traffic was routed through a single misconfigured router, rendering a large part of the Internet inoperable for several
hours \cite{Neumann97}. More recently \cite{Google}, a routing error forced Google to redirect its traffic through Asia, causing congestion that left many Google services unreachable. 
Distance vector currently has no mechanism to recover from such scenarios. Instead, human operators are left to manually reconfigure routers. It is in this context that we propose and
evaluate automated solutions for recovery.  We make the following contributions:

\begin{itemize}

\item We design, develop, and evaluate three different approaches -- \seconds, \purges, and \cpr -- for correctly recovering from the injection of false routing state (Section \ref{sec:idea}). 
\second performs localized state invalidation, followed by network-wide recovery using the traditional distance vector algorithm (Section \ref{subsubsec:second}).
The \purge algorithm performs global false state invalidation by using diffusing computations to invalidate distance vector 
entries (network-wide) that routed through a compromised node (Section \ref{subsubsec:purge}). Then, traditional distance vector routing is used to recompute distance vectors.
\cpr uses local snapshots of the routing table and a rollback mechanism to implement recovery (Section \ref{subsubsec:cpr}). 

\item We prove the correctness of each algorithm for scenarios of single and multiple compromised nodes (Section \ref{sec:correct}). 
A recovery algorithm is correct if the routing tables for all nodes have converged to a global state in which all nodes 
have removed each compromised node as a destination and no node bears a least cost path to any destination that routes through a compromised node.

\item We derive communication complexity bounds for each algorithm over a synchronous communication model (Section \ref{sec:analysis}).  
%In some scenarios, we find the exact number of messages each node sends in order recover. 
We find all three algorithms are bounded above by $O(mnd)$ -- where $d$ is the diamter, $n$ is the number of nodes, and $m$ the maximum out-degree of any node -- in scenarios where 
link costs remain fixed.  In scenarios where link costs can change, \cpr incurs additional overhead (not experieced by \second and \purges) because \cpr must update stale
state after rolling back.  This leads to an additional term for \cpr in its $O(mnd)$ upper bound. %\cpr has an additional term
\xxxx{shorten this bullet.}

\item Using simulations, we evaluate the efficiency of each algorithm in terms of message overhead and convergence time in scenarios with single and 
multiple compromised nodes. (Section \ref{sec:eval}).
Our simulations show that \cpr using poison reverse outperforms \second and \purge (with and without poison reverse) -- at the cost of checkpoint memory --
over topologies with fixed and changing link costs. This is because \cpr efficiently removes all false state by rolling back to a checkpoint
immediately preceding the injection of false routing state. 

\item We show through simulations that \purge using poison reverse yields performance close to \cpr with poison reverse in scenarios where link costs can change
(Section \ref{subsec:change}). \purge makes use of computations subsequent to the injection of false routing state that do not depend on false routing state, while \cpr 
must process all valid link cost changes that occurred since false routing state was injected.  Finally, our simulations show that poison reverse significantly improves 
performance for all three algorithms, especially for topologies with changing link costs (Section \ref{subsubsec:pr-fixed} and \ref{subsubsec:pr-change}). % by removing pairwise routing loops.

\end{itemize}

Recovery from false routing state is closely related to the problem of
recovering from malicious transactions \cite{Liu98,Liu00} in
distributed databases. Our problem is also similar to that of rollback
in optimistic parallel simulation \cite{Jeff}. However, we are unaware
of any existing solutions to the problem of recovering from false
routing state. A closely related problem to the one considered in this
paper is that of discovering misconfigured nodes. In
Section~\ref{sec:problem}, we discuss existing solutions to this
problem. In fact, the output of these algorithms serve as input to the
recovery algorithms proposed in this paper.




