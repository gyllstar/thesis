\section{Conclusions and Future Work}
\label{sec:future}

In this paper, we developed methods for recovery in scenarios where malicious nodes inject false state into a distributed system.  
We studied an instance of this problem in distance vector routing. 
We presented and evaluated -- through a theoretical analysis and simulation -- three new algorithms for recovery in such scenarios. %from false state in distance vector routing 
%Our results showed that all of our algorithms 
In the case of topologies with changing link costs, we found that poison reverse yields dramatic reductions in message complexity for all three algorithms. 
Among our three algorithms, our results showed that \cpr -- a checkpoint-rollback based algorithm -- using poison reverse yields the lowest message and time overhead in all scenarios. 
However, \cpr has storage overhead and requires loosely synchronized clocks. %The \infinity problem results in high message overhead for \second but poison reverse alleviates	
\purge does not have these restrictions and we showed that \purge using poison reverse is only slightly worse than \cpr with poison reverse.
Unlike \cprs, \purge has no stale state to update because \purge does not use checkpoints and rollbacks. 
%The \infinity problem results in high message overhead for \seconds, while \purge eliminates the \infinity problem by globally purging false state before finding new least cost paths.

As future work, we are interested in exploring other false state vectors.  For example, finding the worst possible false state a compromised node can inject (e.g., state that maximizes the effect of the \infinity problem).  In addition,
we would like to derive a lower bound -- for both message and time complexity -- for recovery.
%Some options include the minimum distance to all nodes, state that maximizes the effect of the \infinity problem, and false state that contaminates a bottleneck link.
%We have also started a theoretical analysis of our algorithms.



