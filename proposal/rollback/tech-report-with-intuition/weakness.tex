\section{Weaknesses}
\label{sec:weak}

Our evaluation fails to consider a lower bound -- both in terms of message and time overhead -- on recovery.   Our theoretical and simulation study both consider the relative 
performance of our recovery algorithms and the DUAL algorithm.  A lower bound would yield insights into the overall performance of all three of our recovery algorithms: without a lower
bound it is possible that all three of our recovery algorithms are inefficient.  However, we believe this is unlikely.
%A lower bound would give insight into the effectiveness of all three of our recovery algorithms and the DUAL algorithm, rather than just demonstrating their relative performance. 

Our model for compromised node behavior is simplistic (e.g., nodes falsely claim a cost of $1$ to all other nodes).  Clearly, this makes the detection of a compromised node easy.  Because
our focus is on false state recovery, rather than false state detection, we believe our simplistic model is still a meaningful context to evaluate our recovery algorithms. 
Furthermore, since we are unaware of any existing approach that explicity considers this problem, we feel it is appropriate to start with the simplest problem formulation that succeeds in revealing 
the fundamental challenges of the false-state recovery problem.  We believe we have succeeded in this aim. 

%{\bf look up infocom and IFIP weaknesses}
