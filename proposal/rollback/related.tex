%%%%%%%%%%%%%% Related Work section %%%%%%%%%%%%%%%%%%
\section{Related Work}
\label{sec:related-rollback}


%There is much research in detecting routing misconfiguration \cite{Arini}, \cite{Feam}, \cite{Feldmann} and forwarding misbehaviors \cite{Paul02}.
%This research is vital to our work because the output of these algorithms (i.e., identity of the compromised node) is the input to our recovery algorithms. 

%To the best our knowledge no existing approaches exist to address recovery from false routing state in distance vector. However,
%work on recovery from malicious database transactions, database crash recovery, distributed rollback-recovery, 
%and discrete parallel simulation are relevant to our problem.  We discuss each of these topics below.

To the best our knowledge no existing approach exists to address recovery from false routing state in distance vector routing. However,
our problem is similar to that of recovering from malicious but committed database transactions. Liu et al.
\cite{Liu98} and Ammann et al \cite{Liu00} develop algorithms to restore a database to a valid state after a malicious transaction has been identified. 
\purges's algorithm to globally invalidate
false state can be interpreted as a distributed implementation of the dependency graph approach by Liu et al. \cite{Liu00}.  Additionally,
if we treat link cost change events that occur after the compromised node has been discovered as database transactions, we face a similar design decision as in  \cite{Liu98}: 
do we wait until recovery is complete before applying link cost changes or do we allow the link cost changes to execute concurrently?

%the purge phase in effect finds all nodes that either directly or transitively use \bad as an intermediate node.

Database crash recovery \cite{Mohan92} and message passing systems \cite{Arini} both use snapshots to restore the system in the event of a failure. In both 
problem domains, the snapshot algorithms are careful to ensure snapshots are globally consistent. 
In our setting, consistent global snapshots are not required for \cprs, since  
distance vector routing only requires that all initial distance estimates be non-negative.

Garcia-Lunes-Aceves's DUAL algorithm \cite{JJ93} uses diffusing computations to coordinate least cost updates in order to prevent routing loops. 
In our case, \cpr and the prepossessing procedure (Section \ref{subsec:preprocess}) use diffusing computations for purposes other than updating least costs 
(e.g., rollback to a checkpoint in the case of \cpr
and remove \bad as a destination during preprocessing). Like DUAL, the purpose of \purges's diffusing computations is to prevent routing loops.  However, \purges's diffusing computations
do not verify that new least costs preserve loop free routing (as with DUAL) but instead globally invalidate false routing state. 

%In our case, the safe snapshot reflects the system  state before a node is compromised. 

%Elnozahy et al \cite{Elno} survey rollback-recovery techniques for message passing systems. Each algorithm they survey relies on system-wide snapshots 
%to restore system state in the event of a failure. Upon detecting a failure, a global snapshot is read from persistent storage and the system rolls back to this state. 
%In order to preserve correctness, the algorithms are careful to ensure that snapshots reflect a consistent global state.
%Similarly, our \cpr algorithm uses system snapshots to facilitate recovery. However, consistent global snapshots are not required in our setting 

Jefferson \cite{Jeff} proposes a solution to synchronize distributed systems called Time Warp. Time Warp is a form of optimistic 
concurrency control and, as such, occasionally requires rolling back to a checkpoint. Time Warp does so by ``unsending'' each message sent after the time the checkpoint was taken.
With our \cpr algorithm, a node does not need to explicitly ``unsend'' messages after rolling back.  Instead, each node sends its \minv taken at the time of the snapshot, which
implicitly undoes the effects of any messages sent after the snapshot timestamp.

%Jefferson \cite{Jeff} proposes a solution to synchronizing distributed systems through an abstraction of real time called virtual time. Virtual time is implemented
%using the time warp mechanism.  With time warp, each node is responsible for its own checkpointing and corrects errors independently of other nodes.  However, complications can 
%arise if a node, $i$, locally rolls back to a checkpoint. If $i$ has sent messages to other nodes between the timestamp of the checkpoint and the current time, actions at other nodes
%may need to be undone. Time warp does so using antimessages.  Each antimessage corresponds to a positive message.  When $i$ rolls back it sends an antimessage for each positive
%message it sent between the timestamp of the checkpoint and ``now''.  Nodes receiving any such antimessage, undoes the actions (if necessary) of the corresponding positive message.  
%This may involve rolling back and sending more antimessages.






%Time warp's relaxed synchronization criteria makes it an attractive approach to increasing throughput in parallel discrete event simulation.  Rather than
%force time-ordered event processing, time warp relaxes synchronization which provides flexibility for high degrees of parallelism.  

%However, time warp must handle out-of-order events.  If an out-of-order event/message is received, a node must rollback to a checkpoint taken before the receive time 
%of the message.  Also all messages sent after the snapshot is taken must be undone.  Time warp does so by sending antimessages.
%does so 
%Contrary to other approaches which force time-ordered event processing, time warp allows for out-of-order events processing.  Time warp can 	


%loose synchronization
%parallelizism
%rollback and undo any events 


%Distributed rollback and recovery has a rich history.  Jefferson \cite{Jefferson} proposed a solution to synchronizing 
%distributed systems through an abstraction of real time, which he terms virtual time.  Virtual time allows each node to be organized around a global virtual clock, rather
%than maintain a synchronized clock. Under this scheme, each node maintains a local virtual clock.  Messages are processed in the order of their virtual time.  Virtual time 
%can increase system throughput by not having to agree on global virtual time.  

%However, this optimistic approach to synchronization requires a rollback mechanism, in the case where a node receives a message ``in the past''. In the virtual time
%implementation, called Time Warp, each node periodically checkpoints.  When a message from the past is received a node rollback to a checkpoint marking the node's state
%just before the timestamp of the message.  In order to undo messages sent during this period, Time Warp introduces the notion of antimessages.  

%Virtual time is implemented using Time Warp

%{\bf Self-note:} {\it insert ``Virtual Time'' review}.

%Our problem is simpler than that of database crash recovery so our algorithms do not need to address many of the challenges faced in the database setting.
%The same reasoning can be applied to distributed database crash recovery as surveyed by Lin et al \cite{Lin97asurvey}.
%Although many more differnces exist than those mentioned, the key point is that the two problems have a weak connection. i
%To summarize, at a basic level both approaches leverage system snapshots for recovery but the connections do not run any deeper. 


%Before describing our recovery algorithms, we first present a useful analogy: a link cost change event in distance vector is like a distributed 
%database transaction. A database transaction is a sequence of read and writes of tuples. In a distributed database these read and writes can occur over several 
%different sites. Similarly, in DV a \lcd can cause a sequence of reads and writes. If the \lcd
%results in a new \minv, update messages are sent.  Nodes that receive these update messages  \emph{read} the new \minv values and update their 
%distance matrix. If the received \minv results in a new least cost distance, a node will \emph{write} to its \minv and send a message, 
%resulting in potentially more reads and writes. Note that we are not considering the update to the \dmatrix as a \emph{write}, only
%updates to \minv. We use this analogy to help describe \second, \purge, and \cpr.



%%%%%%%%%%%%%%%%%%%%%%%%%%%%%%%5 BEGIN COMMENT %%%%%%%%%%%%%%%%%%%%%%%%%%%%%%%%%%%%%%%%%
\begin{comment}



Superficially, principles from database crash recovery map well to our problem. However, crash recovery algorithms are more complex because they must address 
a set of issues that do not exist in our setting. 
For example, although snapshots are taken in our \cpr algorithm and with database checkpoints, the algorithms are fundamentally different because
a database must address many problems that do not exist in our setting.
%For example, system snapshots are leveraged in the \cpr algorithm to record a state of the system
%before the compromised node goes bad. %to provide allowing for efficient recovery by rolling back to such a snapshot.  
%This is similar to a database checkpoint because it too records system state. However,
A database checkpoints must record the state of 
in-memory data structure such as the Transaction Table and Dirty Page Table. 
Also database checkpoints must address the issue of either quiescing the system
to take a snapshot or take a fuzzy checkpoint that requires additional post-processing steps to provide a consistent state. With \cpr we do not face
these issues. 

Another difference is how state is stored. With \cpr we merely need to add a time dimension to each data structure, while
database transaction logs are more complex. Because the transaction log is responsible for archiving many data structures it has a more 
involved storage, indexing, and maintenance procedures. Another emphasis is that the transaction log physically reside on stable storage in order to
insure against losing all database state in the event multiple crashes. In distance vector we need not be so careful because we
can always rerun distance vector to produce valid distance values.

%Although many more differnces exist than those mentioned, the key point is that the two problems have a weak connection. i
To summarize, at a basic level both approaches leverage system snapshots for recovery but the connections do not run any deeper. The same reasoning
can be applied to distributed database crash recovery as surveyed by Lin et al \cite{Lin97asurvey}.






However, database checkpointing must address 
a whole set of concerns that do not exist in our problem setting. These concerns include but are not limited to:
%crash recovery vs. snapshot 
\begin{itemize}
	
	\item Database crash recovery needs to be precise about what transactions are active in the system to ensure an accurate snapshot. Many transactions may 
	be active when a snapshot is created.

	\item Many in memory data structures and persistent state needs to be recovered after a crash.

	\item There is a wealth of exception conditions to consider in crash recovery.  For example, the system may crash during recovery.

	\item stable storage is required to keep persistent state.  without this insurance all db content can be lost.  with DV the values can
	always be recomputed in the worst case by rerunning dv from scratch.

\end{itemize}

transaction log implementation, transaction table, dirty page table, what a checkpoint is, concurrency
{\it Comment may want to move these comments to the bullet points themselves}
A single distance vector node that wishes to archive the state of \minv and \dmatrix does not face these sorts of challenges. It only needs to record the
before and after state of these data structures when an update message is processed. Unlike the database scenario, we do not have to consider the 
case of concurrent messages.  Rather, a node need only consider the current message because messages are processed one at a time at a given node and 
messages processed by other nodes can be locally ignored. Also, no in memory data structures need to be 
restored in distance vector recovery and the only persist state recovered is contained in the snapshot itself.  
Finally, {\it not sure about exception conditions}.

We conclude that although the principle of using snapshots to improve recovery performance is consistent between the two problem domains, the 
spirit of database checkpointing is quite different from what is required in distance vector recovery. For this reason, we do not explicitly apply techniques
from database checkpointing algorithms. 

\end{comment}
%%%%%%%%%%%%%%%%%%%%%%%%%%%%%%%5 END COMMENT %%%%%%%%%%%%%%%%%%%%%%%%%%%%%%%%%%%%%%%%%

