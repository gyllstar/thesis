%\documentclass{sig-alternate-sigmod08}
\documentclass[10pt,conference]{IEEEtran} 
\usepackage{graphicx}
\usepackage{epsfig}
\usepackage{amssymb}
\usepackage{amsfonts}
\usepackage{verbatim}
\usepackage{moreverb}
\usepackage{cancel}
\usepackage{fancyhdr}
\usepackage{algorithmic}
\usepackage{algorithm}
\usepackage{timestamp}
\usepackage{epstopdf}
%\usepackage{tikz}
%\usepackage{tikz,fullpage,tikz_custom}
\usepackage{subfigure}
%\usepackage{multicol}
%\usepackage{coordsys}
%\usetikzlibrary{arrows,%
%	          	petri,%
%					topaths}%
%\usepackage{tkz-berge}
%\usepackage[position=top]{subfig}
%\usepackage{epstopdf}
%\usepackage{amsmath}


%%%%%%%%%%  NEEDS to be commente out to have the nice headers %%%%%
%\pagestyle{fancy}
\DeclareGraphicsRule{.tic}{png}{.png}{`convert #1 `dirname #1`/`basename #1 .tif`.png}

%\textwidth = 6.5 in
%\textheight = 9 in
%\oddsidemargin = 0.0 in
%\evensidemargin = 0.0 in
%\topmargin = 0.0 in
%\headheight = 0.0 in
%\headsep = 0.0 in
%\parskip = 0.2in
%\parindent = 0.0in
%\abovedisplayskip
%\belowdisplayskip
 
%Rnd{\classname}{FOO 699}
\newcommand{\doc}{Technical Report}
\newcommand{\doctitle}{Efficient Recovery from False State in Distributed Routing Algorithms}
\newcommand{\myname}{Daniel Gyllstrom}
\newcommand{\qed}{\hfill $\Box$ \hfill \\}
\newcommand{\mod}{{\rm mod\ }}

\newcommand{\minv}{$\overrightarrow{min}$ }
\newcommand{\minvi}{$\overrightarrow{min}_i$ }
\newcommand{\minvis}{$\overrightarrow{min}_i$}
\newcommand{\minvj}{$\overrightarrow{min}_j$ }
\newcommand{\minvjs}{$\overrightarrow{min}_j$}
\newcommand{\minvv}{$\overrightarrow{min}_v$ }
\newcommand{\minvvs}{$\overrightarrow{min}_v$}
\newcommand{\dmatrix}{$dmatrix$ }
\newcommand{\dmatrixs}{$dmatrix$}
\newcommand{\dmatrixi}{$dmatrix_i$ }
\newcommand{\dmatrixis}{$dmatrix_i$} 
\newcommand{\dmatrixj}{$dmatrix_j$ }
\newcommand{\dmatrixjs}{$dmatrix_j$} 
\newcommand{\dmatrixv}{$dmatrix_v$ }
\newcommand{\dmatrixvs}{$dmatrix_v$} 
%\newcommand{\dmatrixvs}{{\tt dmatrix$_v$}}
\newcommand{\dv}{{DV$^+$ }}
%\newcommand{\bad}{{$\hat{v}$ }}
\newcommand{\bad}{{$\overline{v}$ }}
\newcommand{\bads}{{$\overline{v}$}}
\newcommand{\alg}{{DV$^+$ }}
\newcommand{\block}{{\tt todo-rename }}
\newcommand{\second}{{{\tt 2}$^{{\tt nd}}$ {\tt best} }}
\newcommand{\seconds}{{{\tt 2}$^{{\tt nd}}$ {\tt best}}}
\newcommand{\infinity}{{count-to-$\infty$ }}
\newcommand{\purge}{{{\tt purge} }}
\newcommand{\purges}{{{\tt purge}}}
\newcommand{\resetall}{{{\tt reset-all} }}
\newcommand{\resetalls}{{{\tt reset-all}}}
\newcommand{\resetk}{{{\tt reset-k} }}
\newcommand{\resetks}{{{\tt reset-k}}}
%\newcommand{\badvector}{{$v_{bad-lie}$ }}
%\newcommand{\oldvector}{{$v_{bad-old}$ }}
\newcommand{\badvector}{$\overrightarrow{bad}$ }
\newcommand{\badvectors}{$\overrightarrow{bad}$}
\newcommand{\oldvector}{$\overrightarrow{old}$ }
\newcommand{\oldvectors}{$\overrightarrow{old}$}
\newcommand{\finalvector}{{$v_{final}$ }}
\newcommand{\illigit}{{illegitimate path} }
\newcommand{\illigits}{{illegitimate paths} }
\newcommand{\lcd}{$\Delta_{lc}$ }
\newcommand{\lcds}{$\Delta_{lc}$s }
\newcommand{\cpr}{{\tt cpr} }
\newcommand{\cprs}{{\tt cpr}}
\newcommand{\er}{Erd\"{o}s-R\'enyi }
\newcommand{\ers}{Erd\"{o}s-R\'enyi} 
\newcommand{\reword}{{\it  Comment: reword}. }
\newcommand{\more}{{\it  Comment: more details needed here}. }
\newcommand{\HRule}{\rule{\linewidth}{0.5mm}}

%newcommand{\block}{{\textsc{block-wait} }}


%%%%%%%%%%%%% This adds the header to each page - currently a bit broken so commented out
%\lhead{\myname{}}
%\chead{\myname{}}
%\rhead{\doc{}}
%\lfoot{}
%

\makeatletter
\newcommand{\un}[1]{%
   % \ifmmode \@@underline{#1} \else %
             $\@@underline{\hbox{#1}}$\fi}
\makeatother
\raggedbottom

\begin{document}




\title{\doctitle}


\author{
\IEEEauthorblockN{Daniel Gyllstrom, Sudarshan Vasudevan, Jim Kurose, Gerome Miklau } 
\IEEEauthorblockA{Department of Computer Science, University of Massachusetts Amherst. \{dpg, svasu, kurose, miklau\}@cs.umass.edu}} 
%\andI 
%\IEEEauthorblockN{Sudarshan Vasudevan} 
%\IEEEauthorblockA{Department of Computer Science \\ 
%University of Massachusetts Amherst \\ 
%svasu@cs.umass.edu} 
%\and
%\IEEEauthorblockN{James Kurose} 
%\IEEEauthorblockA{Department of Computer Science \\ 
%University of Massachusetts Amherst \\ 
%kurose@cs.umass.edu} }

\maketitle

%\section{Abstract}

\begin{abstract}

Malicious and misconfigured nodes can inject incorrect state into a distributed system, which can then be propagated system-wide as a result of normal network operation. 
Such false state can degrade the performance of a distributed system or render it unusable. For example, in the case of network routing algorithms, false state corresponding
to a node incorrectly declaring a cost of $0$ to all destinations (maliciously or due to misconfiguration) can quickly spread through the network. This causes other nodes to (incorrectly) 
route via the misconfigured node, resulting in suboptimal routing and network congestion. We propose three algorithms for efficient recovery in such scenarios, prove the correctness 
of each of these algorithms, and derive communication complexity bounds for each algorithm. Through simulation, we evaluate our algorithms -- in terms of message and time overhead -- when 
applied to removing false state in distance vector routing. Our analysis
shows that over topologies where link costs remain fixed and for the same topologies where link costs change, a recovery algorithm based on system-wide checkpoints and a rollback mechanism 
yields superior performance when using the poison reverse optimization.

\end{abstract}
%We propose algorithms that allow distributed routing algorithms to efficiently recover from false state injected into the network.
%We prove the correctness of each algorithm and evaluate them when applied to distance vector routing. In this context, we evaluate the message and time complexity of each algorithm through simulation.

%{\bf Keywords}: {\it distributed algorithms, fault tolerance, routing, security}


\section{Introduction}
\label{sec:intro}


Malicious and misconfigured nodes can degrade the performance of a distributed system by injecting incorrect state information. Such false state can then be further propagated 
through the system either directly in its original form or indirectly, e.g., by diffusing computations initially using this false state.  In this paper, we consider 
the problem of removing such false state from a distributed system.

In order to make the false-state-removal problem concrete, we investigate distance vector routing as an instance of this problem. Distance vector forms the basis for many routing 
algorithms widely used in the Internet (e.g., BGP, a path-vector algorithm) and in multi-hop wireless networks (e.g., AODV, diffusion routing). However, distance vector is vulnerable 
to compromised nodes that can potentially flood a network with false routing information, resulting in erroneous least cost paths, packet loss, and congestion. Such scenarios have occurred
in practice. For example, in 1997 a significant portion of Internet traffic was routed through a single misconfigured router, rendering a large part of the Internet inoperable for several
hours \cite{Neumann97}. More recently \cite{Google}, a routing error forced Google to redirect its traffic through Asia, causing congestion that left many Google services unreachable. 
Distance vector currently has no mechanism to recover from such scenarios. Instead, human operators are left to manually reconfigure routers. It is in this context that we propose and
evaluate automated solutions for recovery.

In this paper, we design, develop, and evaluate three different approaches for correctly recovering from the injection of false routing state (e.g., a compromised node incorrectly
claiming a distance of $0$ to all destinations). Such false state, in turn, may propagate to other routers through the normal execution of distance vector routing, making this
a network-wide problem. Recovery is correct if the routing tables in all nodes have converged to a global state in which all nodes have removed each compromised node as a destination,
and no node has a least cost path to any destination that routes through a compromised node.

Specifically, we develop three novel distributed recovery algorithms: \seconds, \purges, and \cprs. \second performs localized state invalidation, followed by network-wide recovery. 
Nodes directly adjacent to a compromised node locally select alternate paths that avoid the compromised node; the traditional distributed distance vector algorithm is then executed to 
remove remaining false state using these new distance vectors. The \purge algorithm performs global false state invalidation by using diffusing computations to invalidate distance vector 
entries (network-wide) that routed through a compromised node. As in \seconds, traditional distance vector routing is then used to recompute distance vectors. \cpr uses local snapshots 
and a rollback mechanism to implement recovery. Although our solutions are tailored to distance vector routing, we believe they represent approaches that are applicable to other instances
of this problem.

We prove the correctness of each algorithm and evaluate its efficiency in terms of message overhead and convergence time via simulation. Our simulations show that when considering topologies
in which link costs remain fixed, \cpr outperforms both \purge and \second (at the cost of checkpoint memory). This is because \cpr can efficiently remove all false state by simply rolling back
to a checkpoint immediately preceding the injection of false routing state. In scenarios where link costs can change, \purge outperforms \cpr and \seconds. \cpr performs poorly because, following 
rollback, it must process the valid link cost changes that occurred since the false routing state was injected;  \second and \purges, however, can make use of computations subsequent to the 
injection of false routing state that did not depend on the false routing state. We will see, however, that \second performance suffers because of the so-called \infinity problem.


Recovery from false routing state has similarities to the problem of
recovering from malicious transactions \cite{Liu98, Liu00} in
distributed databases. Our problem is also similar to that of rollback
in optimistic parallel simulation \cite{Jeff}. However, we are unaware
of any existing solutions to the problem of recovering from false
routing state. A related problem to the one considered in this
paper is that of discovering misconfigured nodes. In
Section~\ref{sec:problem}, we discuss existing solutions to this
problem. In fact, the output of these algorithms serve as input to the
recovery algorithms proposed in this paper.

This paper has six sections. In Section \ref{sec:problem} we define the problem and state our assumptions.
We present our three recovery algorithms in Section \ref{sec:algs}.  Then, in Section \ref{sec:analysis}, we present a qualitative evaluation 
of our recovery algorithms. Section \ref{sec:eval} describes our simulation study. We detail related work in Section \ref{sec:related} and finally we conclude and 
comment on directions for future work in Section \ref{sec:future}.







\section{Problem Formulation}
\label{sec:problem}

We consider distance vector routing \cite{Gall87} over arbitrary network topologies. We model a network as an undirected graph, $G=(V,E)$,
with a link weight function $w: E \rightarrow \mathbb{N}$.
{\footnote {\small Recovery is simple with link state routing: each node uses its complete topology map to compute new least cost paths that avoid all compromised nodes.
Thus we do not consider link state routing in this chapter.}}
Each node, $v$, maintains the following state as part of distance vector: a vector of all adjacent nodes ($adj(v)$), a vector of least cost distances to all
nodes in $G$ (\minvvs), and a \emph{distance matrix} that contains distances to every node in the network via each adjacent node (\dmatrixvs). 

For simplicity, we present our recovery algorithms in the case of a single compromised node.
We describe the necessary extensions to handle multiple compromised nodes in Section \ref{subsec:mult}.
We assume that the identity of the compromised node is provided by a different algorithm, and thus do not consider this problem in this paper.
Examples of such algorithms include \cite{Arini,Feam,Vishal02,Pad03,Paul02}. %\cite{Arini}, \cite{Feam}, \cite{Feldmann} %\cite{Arini, Feam, Feldmann}
Specifically, we assume that at time $t_b$, this algorithm is used to notify all neighbors of the 
compromised node. Let $t'$ be the time the node was compromised.

For each of our algorithms, the goal is for all nodes to recover ``correctly'': all nodes should remove the compromised nodes as a destination and find
new least cost distances that do not use a compromised node. If the network becomes disconnected as a result of removing the compromised node, all
nodes need only compute new least cost distances to all other nodes within their connected component.

For simplicity, let \bad denote the compromised node, let \oldvector refer to $\overrightarrow{min}_{\overline{v}}$ 
before \bad was compromised, and let \badvector denote $\overrightarrow{min}_{\overline{v}}$ after \bad has been compromised.
Intuitively, \oldvector and \badvector are snapshots of the compromised node's least cost vector taken at two different timesteps: \oldvector marks the snapshot taken before \bad was compromised and 
\badvector represents a snapshot taken after \bad was compromised.


Table \ref{tab:abbrev} summarizes the notation used in this chapter. 

\begin{table}[t]
\begin{center}
\begin{tabular}{l l} 
\hline \hline
   	{\bf Abbreviation} & {\bf Meaning} \\
		  \hline 
			\minvi & node $i$'s the least cost vector \\
			\dmatrixi & node $i$' distance matrix \\
		%	\lcd & link cost change event \\
			DV & Distance Vector \\
			\hline
			$t_b$ & time the compromised node is detected \\
			$t'$ & time the compromised node was compromised \\
			\badvector & compromised node's least cost vector at and after $t$  \\
			\oldvector & compromised node's least cost vector at and before $t'$ \\
			\bad & compromised node \\ 
			$adj(v)$ & nodes adjacent to $v$ in $G'$ \\ 
			\hline \hline
			\end{tabular}
			\end{center}
\caption{Table of abbreviations.} 
			%* The distance matrix for node $v$ contains $v$'s distance to all nodes $v_d \in V$ via all $v_n: v_n \in adj(v)$.}
\label{tab:abbrev}
\end{table}



%%%%%%%%%% Recovery Algorithms Descriptions %%%%%%%%%%%%%%%%%%
\section{Recovery Algorithms}
\label{sec:algs}

In this section we propose three new recovery algorithms: \seconds, \purges, and \cprs.  
With one exception, the input and output of each algorithm is the same. 
{\footnote {\small Additionally, as input \cpr requires that each $v \in adj($\bads$)$ is notified of the time, $t'$, in which \bad was compromised.}}

{\bf Input:}  Undirected graph, $G=(V,E)$, with weight function $w: E \rightarrow \mathbb{N}$.  $\forall v \in V$,  \minv and \dmatrix are computed
(using distance vector). Also, each $v \in adj($\bads$)$ is notified that \bad was compromised.

{\bf Output:} Undirected graph, $G'=(V',E')$, where $V' = V -\{$\bads$\}$, $E'=E - \{(\bar{v},v_i)$ $|$ $v_i \in adj(\bar{v}) \}$,
and link weight function $w:E \rightarrow \mathbb{N}$.  \minvv and \dmatrixv are computed via the algorithms discussed below $\forall  v \in V'$. 

First we describe a preprocessing procedure common to all three recovery algorithms. Then we describe each recovery algorithm. 


\subsection{Preprocessing}
\label{subsec:preprocess}
All three recovery algorithms share a common preprocessing procedure.  The procedure removes \bad as a destination and finds the node IDs in each connected component. 
This could be implemented (as we have done here) using diffusing computations \cite{Dijkstra80} initiated at each $v \in adj($\bads$)$. 
%A diffusing computation is a distributed algorithm started at a source node which grows by sending queries along a spanning tree, constructed
%simultaneously as the queries propagate through the network.  When the computation reaches the leaves of the spanning tree, replies travel back along the tree towards the
%source causing the tree to shrink. The computation eventually terminates when the source receives replies from each of its children in the tree. 
In our case, each diffusing computation message contains a vector of node IDs.  When 
a node receives a diffusing computation message, the node adds its ID to the vector and removes \bad as a destination. At the end of the diffusing computation, 
each $v \in adj($\bads$)$ has a vector that includes all nodes in $v$'s connected component. Finally, each $v \in adj($\bads$)$ broadcasts the vector of node IDs to 
all nodes in their connected component. In the case where removing \bad partitions the network, each node will only compute shortest paths to nodes in the vector. 


\subsection{The 2$^{nd}$ best Algorithm}
\label{subsec:second}
\second invalidates state locally and then uses distance vector to implement network-wide recovery.  Following the preprocessing described in Section \ref{subsec:preprocess}, 
each neighbor of the compromised node locally invalidates state by selecting the least cost pre-existing alternate path that does not use the compromised node as the first hop.
The resulting distance vectors trigger the execution of traditional distance vector to remove the remaining false state.

\second is simple and makes no synchronization assumptions. 
However, \second is vulnerable to the \infinity problem. Because each node only has local information, the new shortest paths may continue to use \bads.
We will see in our simulation study that the \infinity problem can incur significant message and time costs.



\subsection{The purge Algorithm}
\label{subsec:purge}


\purge globally invalidates all false state using a diffusing computation and then uses distance vector to compute new distance values that avoid all invalidated paths.
The diffusing computation is initiated at the neighbors of \bad because only these nodes are 
aware if \bad is used an intermediary node. The diffusing computations spread from \bads's neighbors to the network edge, invalidating false state at each node along the way. 
Then ACKs travel back from the network edge to the neighbors of \bads, indicating that the diffusing computation is complete. 
Next, \purge uses distance vector to recompute least cost paths invalidated by the diffusing computations.

%Note that a consequence of the diffusing computation is that not only is all \badvector state deleted, but all \oldvector state as well.  
%Consider the case when \bad is detected before node $i$ receives \badvectors.
%It is possible that $i$ uses \oldvector to reach a destination, $d$. In this case, the diffusing computation will set $i$'s distance to $d$ to $\infty$.


\subsection{The cpr Algorithm}
\label{subsec:cpr}

\cprs {\footnote {\small The name is an abbreviation for {\bf C}heck{\bf P}oint and {\bf R}ollback. }} is our third and final recovery algorithm. 
Unlike \second and \purges, \cpr requires that clocks across different nodes be loosely synchronized i.e., the maximum clock offset between
any two nodes is bounded. Here we present \cpr assuming all clocks are perfectly synchronized. 
%Extensions to handle loosely synchronized clocks should be clear. 
%Accordingly, we assume that all neighbors of \bads, are notified of the time, $t'$, at which \bad was compromised.

For each node, $i \in G$, \cpr adds a time dimension to \minvi and \dmatrixis, which \cpr then uses to locally archive a complete history of values.  
Once the compromised node is discovered, the archive allows each node to rollback to a system snapshot from a time before \bad was compromised. 
\cpr does so using diffusing computations.  Then, \cpr removes all \badvector and \oldvector state while updating stale distance values resulting from link cost changes that
occurred between the time the snapshot was taken and the ``current'' time.
This last step is executed by initiating a distance vector computation from the neigbhors of the compromised node.
%To resolve these issues, each neighbor, $i$, of \bads, sets its distance to \bad to $\infty$ and then selects new least cost values that avoid the compromised node, triggering
%the execution of distance vector to update the remaining distance vectors. 


%{\bf Step 1: Create a \minv and \dmatrix archive.} 
%	We define a  \emph{snapshot} of a data structure to be a copy of all current distance values along with a timestamp.
%	{\footnote {\small In practice, we only archive distance values that have changed. Thus each distance value is associated with its own timestamp.}}
%	The timestamp marks the time at which that set of distance values start being used. 
%	\minv and \dmatrix are the only data structures that need to be archived. This approach is similar to ones used in temporal databases 
%  \cite{Lomet06,Jensen91}.


{%\bf Step 2: Rolling back to a valid snapshot.} 
%Rollback is implemented using diffusing computations initiated at the neighbors of the compromised node. T
%he diffusing computation ensures that all nodes roll back to their most recent snapshot taken before $t'$.
%(Note that this rollback algorithm ensures that no reinstated distance value uses \badvector because every node rolls back to a snapshot with a timestamp less that $t'$. )


%{\bf Step 3: Steps after rollback.} After Step 2 completes, the algorithm in Section \ref{subsec:preprocess} is executed.
%There are two issues to address.
%First, some nodes may be using \oldvectors.  Second, some nodes may have stale state as a result of link cost changes that occurred during $[t',t]$ and 
%consequently are not reflected in the snapshot. 
%To resolve these issues, each neighbor, $i$, of \bads, sets its distance to \bad to $\infty$ and then selects new least cost values that avoid the compromised node, triggering
%the execution of distance vector to update the remaining distance vectors.  









\input{trends}

\section{Evaluation}
\label{sec:eval}

In this section, we use simulations to characterize the performance of each of our three recovery algorithms in terms of message and time overhead. 
Our goal is to illustrate the relative performance of our recovery algorithms over different topology types (e.g., \er graphs, Internet-like graphs) and
different network conditions (e.g., fixed link costs, changing link costs).
We evaluate recovery after a single compromised node has distributed false routing state.
%We count the number of messages required for nodes to find valid routing paths and count the number timesteps (epochs) for each algorithm to converge.

We build a custom simulator with a synchronous communication model: nodes send and receive messages at fixed epochs.  In each epoch, a node receives a
message from all its neighbors and performs its local computation.  In
the next epoch, the node sends a message (if needed).   All algorithms
are deterministic under this communication model.
The synchronous communication model, although simple, yields
interesting insights into the performance of each of the recovery
algorithms. Evaluation of our algorithms using a more general
asynchronous communication model is currently under
investigation. However, we believe an asynchronous implementation
will demonstrate similar trends.  

We simulate the following scenario:

\begin{enumerate}
	\item Before $t'$, $\forall v \in V$ \minvv and \dmatrixv are correctly computed.

	\item At time $t'$, \bad is compromised and advertises a \badvector (a vector with a cost of $1$ to \emph{every} node in the network) to its neighboring nodes.

	\item \badvector spreads for a specified number of hops (this varies by experiment).  Variable $k$ refers to the number of hops that \badvector has spread.

	\item At time $t$, some node $v \in V$ notifies all $v \in adj($\bads$)$ that \bad was compromised. 
	{\footnote { \small For \cpr this node also indicates the time, $t'$, \bad was compromised.}} 

\end{enumerate}
The message and time overhead are measured in step (4) above. The pre-computation common to all three recovery algorithms, described in Section \ref{subsec:preprocess},
is not counted towards message and time overhead.


\subsection{Fixed Link Weight Experiments}
\label{subsec:fixed}

In the next three experiments, we evaluate our recovery algorithms over different topology types in the case of fixed link costs.

We start with a simplified experiment setting to measure the 
 message and time overhead incurred by each of the recovery
 algorithms. In particular, we consider 
Erd\"{o}s-R\'enyi graphs with parameters $n$ and $p$. Further, the
link weight of each edge in the graph is set to 50.
$n$ is the number of graph nodes and $p$ is the probability that link $(i,j)$ exists where $i,j \in V$. We iterate over different values of $k$. For each $k$, we 
generate an Erd\"{o}s-R\'enyi graph, $G = (V,E)$, with parameters $n$ and $p$. Then we select a $v \in V$ uniformly at random and simulate the scenario described above, 
using $v$ as the compromised node. In total we sample $20$ unique nodes for each $G$.
We set $n=100$, $p=\{0.05,0.15,0.25, 0.25\}$, and let $k=\{1,2,
... 10\}$. Each data point is an average over $600$ runs ($20$ runs over 
$30$ topologies).  We then plot the $90 \%$ confidence interval.




\purge and \second message overhead increases with larger $k$. Larger $k$ implies more paths to repair, and therefore increased messaging.
For values of $k$ greater than a graph's diameter, the message overhead remains constant, as expected. 


\subsubsection{Experiment 2 - \er Graphs with Fixed but Randomly Chosen Link Weights}
\label{subsec:expt2}


The experimental setup is identical to Experiment 1 with one exception: link weights are selected uniformly at random between $[1,n]$ (rather than using 
fixed link weight of $50$).

Figure \ref{fig:msg-rand} show the message overhead for different $k$ for $p=.05$. We omit the figures for the other $p$ values because they follow the 
same trend as $p=.05$ \cite{Tech}.
In striking contrast to Experiment 1, \purge outperforms \second for all values of $k$. 
{\footnote {\small In some cases (e.g., $k=1$ for $p=.15$ and $p=.50$) \second performs better than \purges.  In these cases, \second has few routing loops \cite{Tech}.
}}
\second performs poorly because the \infinity problem: Table \ref{tab:loop1} shows the large average number of pairwise routing loops in this experiment, 
an indicator of the occurrence of \infinity problem.
No routing loops are found with \purges. \cpr performs well for the same reasons described in Section \ref{subsubsec:expt1}.  




\begin{figure*}[t]
\centering

%\subfigure[{Message overhead for \er graph with link weights selected uniformly random from $[1,100]$. $p=.05$ and diameter=$6.14$.}]{ \includegraphics[scale=.35]{figs/rollback-msg-rand5.pdf}}
%\subfigure[{Message overhead for $p=0.05$ \er with link weights selected uniformly random with different $\lambda$ values. $z$ refers to the number of timesteps \cpr must rollback.}]
%	{\includegraphics[scale=.35]{figs/rollback-p05-k4.pdf}} 

\subfigure[{sim 1.}]{ \includegraphics[scale=.55]{figs/rollback-msg-rand5.pdf}}
\subfigure[{sim 2.}] {\includegraphics[scale=.55]{figs/rollback-p05-k4.pdf}} 


\caption{Rollback simulations} 
\label{fig:rollback-simulations}
\end{figure*} 


%\begin{figure}[t]
%\centering
%\includegraphics[scale=.35]{figs/rollback-msg-rand5.pdf}
%\caption{Message overhead for \er graph with link weights selected uniformly random from $[1,100]$. $p=.05$ and diameter=$6.14$.}
%\label{fig:msg-rand}
%\end{figure}


In addition, we counted the number of epochs in which at least one pairwise routing loop existed.  For \second (across all topologies), on average, all but the last three 
timesteps had at least one routing loop.  This suggests that the \infinity problem dominates the cost for \seconds. 




\subsection{Link Weight Change Experiments}
\label{subsec:change}

So far, we have evaluated our algorithms over different topologies with fixed link costs. We found that \cpr outperforms the other algorithms because \cpr removes false
routing state with a single diffusing computation, rather than using an iterative process like \second and \purges.  In the next 
two experiments we evaluate our algorithms over graphs with changing link costs. We introduce link cost changes between the time \bad is compromised and when \bad is discovered 
(e.g. during $[t',t]$). 
In particular, there are $\lambda$ link cost changes per timestep, where $\lambda$ is deterministic. 
To create a link cost change event, we modify links uniformly at random (except for all $(v,\bar{v})$ links). % where $v \in V'$ and $v \in adj(\bar{v})$.
The new link cost is selected uniformly at random from $[1,n]$. 


\subsubsection{Experiment 5}

In this experiment we study the trade-off between message overhead and storage overhead for \cprs. To this end, we vary the frequency at which \cpr checkpoints and fix 
the interval $[t',t]$. Otherwise, our experimental setup is the same as Experiment 4.

Due to space constraints, we only display a single plot. Figure \ref{fig:lc-fixk} shows the results for an \er graph with link weights selected uniformly at random between $[1,n]$,
$n=100$, $p=.05$, $\lambda=4$ and $k=2$. We plot message overhead against the number of timesteps \cpr must rollback, $z$. \cprs's message overhead increases with larger $z$ 
because as $z$ increases there are more link cost change events to process. \second and \purge have constant message overhead because they operate independent of $z$.

We conclude that as the frequency of \cpr snapshots decreases, \cpr incurs higher message overhead.  Therefore, when choosing the frequency of checkpoints,
the trade-off between storage and message overhead must be carefully considered. 


%\begin{figure}[t]
%\centering
%\includegraphics[scale=.35]{figs/rollback-p05-k4.pdf}
%\caption{Message overhead for $p=0.05$ \er with link weights selected uniformly random with different $\lambda$ values. $z$ refers to the number of timesteps \cpr must rollback.}
%\label{fig:lc-fixk}
%\end{figure} 






\subsection{Summary}
\label{subsec:discuss}

Our results show that for graphs with fixed link costs, \cpr yields the lowest message and time overhead. \cpr benefits from removing false state with a single
diffusing computation. However, \cpr has storage overhead, requires loosely synchronized clocks, and requires the time \bad was compromised be identified.

\seconds's performance is determined by the \infinity problem. In this case of \er graphs with fixed unit link weights, the \infinity problem was minimal, 
helping \second perform better than \purges. \purge avoids the \infinity problem by first globally invalidating false state.  Therefore in cases where the \infinity problem is 
significant, \purge outperforms \seconds.

When considering graphs with changing link costs, \cprs's performance suffers because it must process all valid link cost changes that occurred since \bad was compromised.
Meanwhile, \second and \purge make use of computations that followed the injection of false state, that do not depend on false routing state. However, \seconds's performance degrades 
because of the \infinity problem.  \purge eliminates the \infinity problem and therefore yields the best performance over topologies with changing link costs.

Finally, we found that an additional challenge with \cpr is setting the parameter which determines the checkpoint frequency.
More frequent checkpointing yields lower message and time overhead at the cost of more storage overhead. Ultimately, application-specific factors must be considered
when setting this parameter. 



\section{Related Work}
\label{sec:related}

The PMUP problem -- find the minimum number and placement of PMUs to allow a bus system to be fully observable -- is well-studied \cite{Baldwin93,Brueni05,Haynes02, Mili90, Xu04}. 
Although similar, the \maxinc problem differs from the PMUP problem: \maxinc considers the more general case in which a constant number of PMUs are given and the task is to 
place the PMUs such that the maximum number of nodes are observed.
Haynes et al. \cite{Haynes02} and Brueni and Heath \cite{Brueni05} both prove PMUP is NP-Complete.  We leverage these proofs in our NP-Completeness proofs.  

The power systems literature generally ignores the fact that PMUP is NP-Complete because, in practice, power system graphs are small enough to allow for an exact solution to be found.
Xu and Abur \cite{Xu04,Xu05} use integer programming to find the optimal PMU placement when a subset of buses are zero injection. O2 can only be applied to zero injection buses.  As a result,
the PMUP problem is simplified when only some buses are zero injection.
Baldwin et al. \cite{Baldwin93} and Mili et al. \cite{Mili90} use simulated annealing to determine PMU placement. 
%The work of Xu and Abur \cite{Xu04} and Phadke et al. are representive of the power systems approach to the problem: formulate the problem as integer 
%program and use an integer programming solver to find the optimal PMU placement.  

Aazami and Stilp \cite{Aazami07} investigate approximation algorithms for the PMUP problem.  They derive a hardness approximation threshold of $2^{\log^{1 -\epsilon}n}$ for PMUP.  
Also they prove that in the worst case, the same greedy algorithm presented in Section \ref{sec:approx} does no better $\Theta(n)$ of the optimal solution.  
%We leverage this approximation result in proving the approximation ratios of our heuristic-based algorithms.

Chen and Abur \cite{Abur06} and Vanfretti et al. \cite{Vanfretti10} both study the problem of bad PMU data. Chen and Abur \cite{Abur06} formulate their problem differently than \xval and \xvalparts.  
They consider graphs that are already fully observable and then add PMUs to the system to make all existing PMU measurements non-critical 
(a critical measurement is one in which the removal of a PMU makes the system
no longer fully observable). Vanfretti et al. \cite{Vanfretti10} define the cross-validation rules used in this paper.  They also derive a
lower bound on the number of PMUs needed to ensure all PMUs are cross-validated and the system is fully observable. 





For each algorithm proposed in Chapter \ref{ch:reliable-mcast} -- \mdrs, \fls, \pcnts, \mfs, \mds, and \mc -- we plan to supplement its basic description with a detailed 
specification of the algorithm steps, implement the algorithm in OpenFlow, and evaluate its implementation.  We estimate this requires six months of steady work to complete.  
The author believes this six month goal will be met without issue as the birth and care of a newborn child -- the author expects his first born child in January 2013 --
should introduce no delays nor complications.


\section{Acknowledgments}
The authors greatly appreciate discussions with Dr. Brian DeCleene of BAE Systems, who initially suggested this problem area.


\bibliographystyle{plain}
\bibliography{infocom}

%%%%%%%%%%% Appendix %%%%%%%%%%%%%%%%%%
\section{Appendix}
\label{sec:appendix}

%%%%%%%%%%%%%%% 2nd Best ALG PART 1 %%%%%%%%%%%%%%%%%%%%%%%%%%%%%%
\begin{algorithm}
\caption{\seconds}
\label{alg:second}
%\textsc{centralized-dv}($G$)

\begin{algorithmic}[1]
%\STATE{$t_i \leftarrow t_0$}

\STATE{$flag \leftarrow$ \textsc{false}}
\STATE{set distance to \bad to $\infty$ in \minvi and \dmatrixi}
\FOR{{\bf each} destination $d$ }
	\IF{route via \bad to reach $d$}
		\STATE{select new shortest distance to $d$ which does not route via \bads. Update \minvi and \dmatrixi with this value.}
		\STATE{$flag \leftarrow$ \textsc{true}} 
	\ENDIF
\ENDFOR
	\IF{$flag$ $=$ \textsc{true}}
		\STATE{send \minvi to each  $j \in adj(i)$ where $j \neq$ \bads }
	\ENDIF

\end{algorithmic}
\end{algorithm}




%%%%%%%%%%%%%%% Purge's Purge Phase ALG PART 1 %%%%%%%%%%%%%%%%%%%%%%%%%%%%%%
\begin{algorithm}
\caption{\purges's purge phase}
\label{alg:purge}
%\textsc{centralized-dv}($G$)

\begin{algorithmic}[1]
\STATE{set distance to \bad to $\infty$}
\FOR{{\bf each} destination $d$}
	\IF{route via \bad to reach $d$}
		\STATE{$S \leftarrow S \cup \{d\}$} 
	\ENDIF
\ENDFOR
	
\IF{$S$ is not empty}
	\STATE{send $S$ to each $j \in adj(i)$ where $j \neq$ \bads }
\ENDIF

\end{algorithmic}
\end{algorithm}


%%%%%%%%%%%%%%% Purge's Purge Phase ALG PART 2 %%%%%%%%%%%%%%%%%%%%%%%%%%%%%%
\begin{algorithm}
\caption{\purges's purge phase}
\label{alg:purge2}
%\textsc{centralized-dv}($G$)

\begin{algorithmic}[1]

\FOR{{\bf each} $d \in msg.dests$}
	\IF{route via message source to $d$}
		\STATE{$S \leftarrow S \cup \{d\}$} 
	\ENDIF
\ENDFOR
\IF{$S$ is not empty}
	\STATE{send $S$ to each $j \in adj(i)$ where $j \neq$ \bads }
\ELSE
	\STATE{send $ACK$ to message source}
\ENDIF

\end{algorithmic}
\end{algorithm}



%%%%%%%%%%%%%%% Purge's Discover Phase ALG PART 1%%%%%%%%%%%%%%%%%%%%%%%%%%%%%%
\begin{algorithm}
\caption{\purges's discovery phase}
\label{alg:discover}

\begin{algorithmic}[1]
\STATE{$flag \leftarrow$ \textsc{false}}
\FOR{{\bf each} destination $d$}
	\IF{\minvis$[d] = \infty$}
		\STATE{find shortest distance in \dmatrixi and set in \minvis}
		\STATE{$flag \leftarrow$ \textsc{true}} 
	\ENDIF
\ENDFOR
\IF{$flag$ $=$ \textsc{true}}
	\STATE{send \minvi to each  $j \in adj(i)$ where $j \neq$ \bads }
\ENDIF


\end{algorithmic}
\end{algorithm}

%%%%%%%%%%%%%%% Purge's Discover Phase ALG PART 2 %%%%%%%%%%%%%%%%%%%%%%%%%%%%%%
\begin{algorithm}
\caption{\purges's discovery phase}
\label{alg:discover2}

\begin{algorithmic}[1]
\IF{first round of sending}
	\FOR{{\bf each} destination $d$}
		\STATE{update \minvi with minumum distance in \dmatrixi to $d$}
	\ENDFOR
\ENDIF
\STATE{run distance vector} 

\end{algorithmic}
\end{algorithm}


%%%%%%%%%%%%%%% CPR ALG PART 1 %%%%%%%%%%%%%%%%%%%%%%%%%%%%%%
\begin{algorithm}
\caption{\cpr steps after rollback}
\label{alg:cpr}

\begin{algorithmic}[1]

\STATE{$flag \leftarrow$ \textsc{false}}
\FOR{{\bf each} destination $d$}
	\IF{\minvis$[d] = \infty$}
		\STATE{find shortest distance in \dmatrixi and set in \minvis}
		\STATE{$flag \leftarrow$ \textsc{true}} 
	\ENDIF
\ENDFOR
\IF{$flag$ $=$ \textsc{true} or adjacent link weight changed during $[t',t]$}
	\STATE{send \minvi to each  $j \in adj(i)$ where $j \neq$ \bads }
\ENDIF



\end{algorithmic}
\end{algorithm}


%%%%%%%%%%%%%%% CPR ALG PART 2 %%%%%%%%%%%%%%%%%%%%%%%%%%%%%%
\begin{comment}
\begin{algorithm}
\caption{\cpr steps after rollback}
\label{alg:cpr}

\begin{algorithmic}[1]
	\IF{first round of sending}
		\STATE{update \minvi with most recent link weights of adjacent link}
	\ENDIF
	\STATE{run distance vector} 
\ENDIF


\end{algorithmic}
\end{algorithm}

\end{comment}







%\input{todo}

\end{document}
