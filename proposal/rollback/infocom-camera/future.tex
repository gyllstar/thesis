\section{Conclusions and Future Work}
\label{sec:future}

In this paper, we developed methods for recovery in scenarios where a malicious node injects false state into a distributed system.  
We studied an instance of this problem in distance vector routing.
We presented and evaluated three new algorithms for recovery in such scenarios. %from false state in distance vector routing 
Among our three algorithms, our results show that \cpr -- a checkpoint-rollback based algorithm -- yields the lowest message and time overhead over topologies
with fixed link costs.  However, \cpr has storage overhead and requires loosely synchronized clocks.
In the case of topologies with changing link costs, \purge performs best by avoiding the problems that plague \cpr and \seconds.
Unlike \cprs, \purge has no stale state to update because \purge does not rollback in time.  
The \infinity problem results in high message overhead for \seconds, while \purge eliminates the \infinity problem by globally purging false state before finding new least cost paths.

As future work, we are interested in finding the worst possible false state a compromised node can inject.  Some options include the minimum distance to all nodes (e.g., 
our choice for false state used in this paper), state that maximizes the effect of the \infinity problem, and false state that contaminates a bottleneck link. 
We also would like to evaluate the effects of multiple compromised nodes on our recovery algorithms. 

