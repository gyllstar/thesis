%%%%%%%%%%%%%%%% TRENDS IN RECOVERY SECTION %%%%%%%%%%
\section{Analysis of Algorithms}
\label{sec:analysis}

In Section \ref{subsec:correct}, we prove the correctness of our three
recovery algorithms.  Then, we prove specific properties of these
recovery algorithms in Section \ref{subsec:trends}, which help better
understand our simulation results.

\subsection{Correctness of Recovery Algorithms}
\label{subsec:correct}

%We first state our assumptions and then define correctness. Finally, we prove the correctness of each algorithm. 
We make the following assumptions in our proofs. All the initial \dmatrix values are nonnegative. Furthermore, all \minv values periodically
exchanged between neighboring nodes are nonnegative. 
All $v \in V$ know their adjacent link costs. All link weights in $G$ (and therefore $G'$ as well) are nonnegative and do not change.
{\footnote {\small We use the definition of $G$ and $G'$ described in Section \ref{sec:algs}.}}
$G$ is finite and connected. 

Finally, we assume reliable communication.


{\bf Definition 1.} An algorithm is \emph{correct} if the following two conditions are satisfied. One, $\forall v \in V$, $v$ has the least cost and knows next-hop to all destinations $v' \in V$.
Two, the least cost is computed in finite time.

%{\it Definition 2.} Let $D_i(t)$ be $i$'s least cost to node $d$ at time $t$, such that $d \neq i$. A distance value is \emph{incorrect} for node $i$ if $D_i(t') \neq D_i(t'')$ where 
%$t'$ is the current time and $t''$ marks the time the $D_i$ is first correct.

%{\it Definition 3.} An algorithm is \emph{recovery correct} if it is correct over $G'$.

{\bf Theorem 1}. {\it Distance vector is correct.}

$Proof$. Bertsekas and Gallager \cite{Gall87} prove correctness for distributed Bellman-Ford for arbitrary nonnegative \dmatrix values. 
Their distributed Bellman-Ford algorithm is the same as the distance vector algorithm used in this paper. \qed

{\bf Corollary 1}. {\it \second is correct.}

$Proof$. As per the preprocessing step, each node receiving a diffusing computation message removes \bad as a destination. 
Each node is guaranteed to receive a diffusing computation message (by our reliable communication and finite graph assumptions). Further, the diffusing computation
terminates in finite time.  Thus, we conclude that each $v \in V'$ removes \bad in finite time.

Following the diffusing computation, each $v \in adj($\bads$)$ uses distance vector to determine new least cost paths.
Because all \dmatrixv are nonnegative for all $v \in V'$, by Theorem $1$ we conclude \second is correct. \qed

{\bf Corollary 2}. {\it \purge is correct.}

$Proof$. The diffusing computation starts with each $v \in adj($\bads$)$ finding every destination, $d$, to which $v$'s least cost path 
uses \bad as the first-hop node. $v$ sets its least cost to each such $d$ to $\infty$, thereby invalidating its path to $d$.  
$v$ then initiates a diffusing computation. When an arbitrary node, $i$, receives a diffusing computation message from $j$, $i$ iterates 
through each $d$ specified in the message. If
$i$ routes via $j$ to reach $d$, $i$ sets its least cost to $d$ to $\infty$, therefore invalidating any path to $d$ with $j$ and  
\bad an intermediate nodes.  

By our assumptions, each node receives a diffusing computation message and the diffusing computation terminates in finite time. Thus, we conclude that all paths using \bad as an intermediary
node are invalidated in finite time. Following the preprocessing, each $v \in adj($\bads$)$ uses distance vector to determine new least cost paths.
Because all \dmatrixv are nonnegative for all $v \in V'$, by Theorem $1$ we conclude that \purge is correct. \qed

{\bf Corollary 3}. {\it \cpr is correct.}

$Proof$. \cpr rolls back using a diffusing computation. Each node that receives a diffusing computation message, rolls back to a snapshot with timestep less than $t'$.
By our assumptions, all nodes receive a message and the diffusing computation terminates in finite time.  Thus, we conclude
that each node $v \in V'$ rolls back to a snapshot with timestamp less than $t'$ in finite time.

\cpr then runs the preprocessing algorithm described in Section \ref{subsec:preprocess}, which removes \bad as a destination in finite time (as shown
in Corollary $1$). Because each node rolls back to a snapshot in which all least costs are nonnegative and 
\cpr then uses distance vector to compute new least costs, by Theorem $1$ we conclude that \cpr is correct. \qed

\subsection{Properties of Recovery Algorithms}
\label{subsec:trends}

In this section we formally characterize how \minv values change during recovery. The properties established in this section will aid
in understanding the simulation results presented in Section \ref{sec:eval}. Our proofs assume that link costs remain fixed during recovery (i.e., during $[t',t]$).
We prove properties about \minv in order provide a precise characterization of recovery trends.  
In particular, our proofs establish that:
%We do not characterize system behavior for \second
%and \cpr in this section becuase their behavior is relatively simple.  Instead, we wait until the evaluation section (Section \ref{sec:eval}) 
%to present their recovery trends. 

\begin{itemize}
	\item The least cost between two nodes at the start of recovery is less than or equal to the least cost when recovery has completed. (Theorem $2$)
	%That is, once the effects of the compromised node have been removed from the network, path costs cannot decrease. 

	\item Before recovery begins, if the least cost between two nodes is less than its cost when recovery is complete, the path must 
  be using \badvector or \oldvector either directly or transitively. (Corollary $4$)

	\item During \second and \cpr recovery, if the least cost between two nodes is less than its distance when recovery is complete, the path must 
  be using \badvector or \oldvector either directly or transitively. (Corollary $5$)

	%\item During recovery, if the least cost between two nodes exceeds the cost when recovery is complete then the node has yet to learn about the path used 
  	%when recovery is complete. (Claim $2$)

\end{itemize}

The first two statements apply to any recovery algorithm because they make no claims about \minv values during recovery. 
%These results are only relevant to \second and \purge recovery. \minv values during \cpr recovery do not follow these rules because it is possible
%that when rolling back link costs have different values such that the conditions for our claims below do not hold.

{\bf Notation}. We use the definition of $G$ and $G'$ described in Section \ref{sec:algs}. 
%$G=(V,E)$, with weight function $w: E \rightarrow \mathbb{N}$ and
%$G$ is undirected. $G'=(V',E')$, where $V' = V -$ \bad, $E'=E - \{(\bar{v},v_i)$ $|$ $v_i =adj(\bar{v}) \}$, edge weight function $w: E \rightarrow \mathbb{N}$, and $G'$ is undirected. 
Let $n,d \in V'$. Let $p_s(n,d)$ be the least cost path from node $n$ to $d$ at the start of recovery and $\delta_s(n,d)$ the
cost of this path; $p_i(n,d)$ is a path from $n$ to $d$ used during the recovery 
and $\delta_i(n,d)$ the cost of this path 
{\footnote {\small $p_i(n,d)$ and $\delta_i(n,d)$ can change over time during recovery.}}; and $p_f(n,d)$ the least cost path from $n$ to $d$ 
when recovery is finished and has cost $\delta_f(n,d)$.  %Using this notation we derive the following relationships. 
%Note by path we refer both to the nodes used to get from $n$ to $d$ and the cost of this path.  


{\bf Theorem 2.} $\forall n,d\in V'$, $ \delta_s(n,d) \leq \delta_f(n,d)$.

$Proof$: Assume $\exists n_i,d_i\in V'$ such that $\delta_s(n_i,d_i) > \delta_f(n_i,d_i)$.  The paths available at the start of recovery are a superset of 
those available when recovery is complete. This means $p_f(n_i,d_i)$ is available before recovery begins. Distance vector would use this path rather
than $p_s(n_i,d_i)$, implying that $\delta_s(n_i,d_i)=\delta_f(n_i,d_i)$, a contradiction. \qed

{\bf Corollary 4.} {\it $\forall n,d\in V'$, if $\delta_s(n,d) < \delta_f(n,d)$, then $p_s(n,d)$ is using \badvector or \oldvector either directly or transitively.}

$Proof$:  Assume $\exists n_i,d_i\in V$ such that a path $p_{s}(n_i,d_i)$ with cost $\delta_{s}(n_i,d_i)$ is used before recovery begins where
$\delta_{s}(n_i,d_i) < \delta_f(n_i,d_i)$ and  $p_{s}(n_i,d_i)$ does not use \badvector or \oldvectors.  The only paths available before recovery begins, which do not 
exist when recovery completes, are ones using \badvector or \oldvectors. Therefore, $p_{s}(n_i,d_i)$ must be available after recovery completes since we have assumed that
$p_{s}(n_i,d_i)$ does not use \badvector or \oldvector. Distance vector would use  $p_{s}(n_i,d_i)$ instead of $p_f(n_i,d_i)$ because 
$\delta_{s}(n_i,d_i) < \delta_f(n_i,d_i)$.  However this would 
imply that $\delta_{s}(n_i,d_i)= \delta_f(n_i,d_i)$, a contradiction. \qed

{\bf Corollary 5.} {\it For \second and \cprs. $\forall n,d\in V'$, if $\delta_i(n,d) < \delta_f(n,d)$ 
  then $p_i(n,d)$  must be using \badvector or \oldvector either directly or transitively} 
 {\footnote {\small Corollary 5 does not apply to \purge recovery because the  $\delta_i(n,d) < \delta_f(n,d)$ condition is not always satisfied.}} 

$Proof$: We can use the same proof for Corollary 4 if we substitute $\delta_i(n,d)$ for $\delta_s(n,d)$ and $p_i(n,d)$ for $p_s(n,d)$. \qed


%%%%%%%%%%%%%%%%%%%%% Begin COMMENT %%%%%%%%%%%%%%%%%%%%%%%%
\begin{comment}
Corollary 2 does not apply to \purge recovery because the  $\delta_i(n,d) < \delta_f(n,d)$ may not necessarily be satisfied. This is the case
because \infinity messages are not possible with \purge.

%Corollary 2 does not apply to \purge and \cpr recovery because the  $\delta_i(n,d) < \delta_f(n,d)$ may not necessarily be satisfied. Count-to-infinity
%messages are not possible with \purge so this condition is never met. With \cpr, it is possible that the state of a \minv used after rolling
%back does not satsify this condition because the \minv may reflect old link cost values that are less than their current values.
%These results establish that distance values before and during recovery are less than or equal to their final value. 
These results are useful in understanding recovery using \second and \cpr in cases where link costs remain fixed. With either algorithm,
our proofs imply that the general trend is for all \minv to increase from their value at the start of recovery up 
to their value when recovery has completed. Specifically, as a manifestation of the \infinity problem, paths using \badvector or \oldvector 
(with \cpr only \oldvector can be used as discussed in Section \ref{subsec:cpr}) are shared,
causing least cost paths to count up to their final value.  
There is one exception to this trend.  A node may learn of a new least 
cost path that uses \badvector or \oldvector with a cost less than any of its current least cost path, causing its least cost path to decrease. 

By Claim 1, \purge recovery also starts with \minv values less than or equal to their final value. However, for \purge \minv values actually 
count down from the $\infty$ costs set after the purge phase. 

%\cpr recovery trends are more difficult to characterize because link cost values can change between $t'$ and $t$.

We leverage these characterizations to explain our simulation results in the following section.
\end{comment}
%%%%%%%%%%%%%%%%%%%%% END COMMENT %%%%%%%%%%%%%%%%%%%%%%%%







