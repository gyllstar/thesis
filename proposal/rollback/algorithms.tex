%%%%%%%%%% Recovery Algorithms Descriptions %%%%%%%%%%%%%%%%%%
\section{Recovery Algorithms}
\label{sec:algs}

In this section we propose three recovery algorithms: \seconds, \purges, and \cprs.  
With one exception, the input and output of each algorithm is the same: 
{\footnote {\small Additionally, as input \cpr requires that each $v \in adj($\bads$)$ is notified of the time, $t'$, in which \bad was compromised.}}
\begin{itemize}
	\item \underline{Input}:  Undirected graph, $G=(V,E)$, with weight function $w: E \rightarrow \mathbb{N}$.  $\forall v \in V$,  \minv and \dmatrix are computed
(using distance vector). Also, each $v \in adj($\bads$)$ is notified that \bad was compromised.

	\item \underline{Output}: Undirected graph, $G'=(V',E')$, where $V' = V -\{$\bads$\}$, $E'=E - \{(\bar{v},v_i)$ $|$ $v_i \in adj(\bar{v}) \}$,
and link weight function $w:E \rightarrow \mathbb{N}$.  \minvv and \dmatrixv are computed via the algorithms discussed below $\forall  v \in V'$. 
\end{itemize}
Before we describe each recovery algorithm, we outline a preprocessing procedure common to all three recovery algorithms. %Then we describe each recovery algorithm. 


\subsection{Preprocessing Algorithm}
\label{subsec:preprocess}
All three recovery algorithms share a common preprocessing procedure.  The procedure removes \bad as a destination and finds the node IDs in each connected component. 
This could be implemented (as we have done here) using diffusing computations \cite{Dijkstra80} initiated at each $v \in adj($\bads$)$. 
%A diffusing computation is a distributed algorithm started at a source node which grows by sending queries along a spanning tree, constructed
%simultaneously as the queries propagate through the network.  When the computation reaches the leaves of the spanning tree, replies travel back along the tree towards the
%source causing the tree to shrink. The computation eventually terminates when the source receives replies from each of its children in the tree. 
In our case, each diffusing computation message contains a vector of node IDs.  When 
a node receives a diffusing computation message, the node adds its ID to the vector and removes \bad as a destination. At the end of the diffusing computation, 
each $v \in adj($\bads$)$ has a vector that includes all nodes in $v$'s connected component. Finally, each $v \in adj($\bads$)$ broadcasts the vector of node IDs to 
all nodes in their connected component. In the case where removing \bad partitions the network, each node will only compute shortest paths to nodes in the vector. 


\subsection{The 2$^{nd}$ best Algorithm}
\label{subsec:second}
\second invalidates state locally and then uses distance vector to implement network-wide recovery.  Following the preprocessing described in Section \ref{subsec:preprocess}, 
each neighbor of the compromised node locally invalidates state by selecting the least cost pre-existing alternate path that does not use the compromised node as the first hop.
The resulting distance vectors trigger the execution of traditional distance vector to remove the remaining false state.

\second is simple and makes no synchronization assumptions. 
However, \second is vulnerable to the \infinity problem. Because each node only has local information, the newly selected shortest paths may continue to use \bads.
We will see in our simulation study that the \infinity problem can incur significant message and time costs.



\subsection{The purge Algorithm}
\label{subsec:purge}


\purge globally invalidates all false state using a diffusing computation and then uses distance vector to compute new distance values that avoid all invalidated paths.
The diffusing computation is initiated at the neighbors of \bad because only these nodes are 
aware if \bad is used an intermediary node. The diffusing computations spread from \bads's neighbors to the network edge, invalidating false state at each node along the way. 
Then ACKs travel back from the network edge to the neighbors of \bads, indicating that the diffusing computation is complete. 
Next, \purge uses distance vector to recompute least cost paths invalidated by the diffusing computations.

%Note that a consequence of the diffusing computation is that not only is all \badvector state deleted, but all \oldvector state as well.  
%Consider the case when \bad is detected before node $i$ receives \badvectors.
%It is possible that $i$ uses \oldvector to reach a destination, $d$. In this case, the diffusing computation will set $i$'s distance to $d$ to $\infty$.


\subsection{The cpr Algorithm}
\label{subsec:cpr}

\cprs {\footnote {\small The name is an abbreviation for {\bf C}heck{\bf P}oint and {\bf R}ollback. }} is our third and final recovery algorithm. 
Unlike \second and \purges, \cpr requires that clocks across different nodes be loosely synchronized i.e., the maximum clock offset between
any two nodes is bounded. Here we present \cpr assuming all clocks are perfectly synchronized. 
%Extensions to handle loosely synchronized clocks should be clear. 
%Accordingly, we assume that all neighbors of \bads, are notified of the time, $t'$, at which \bad was compromised.

For each node, $i \in G$, \cpr adds a time dimension to \minvi and \dmatrixis, which \cpr then uses to locally archive a complete history of values.  
Once the compromised node is discovered, the archive allows each node to rollback to a system snapshot from a time before \bad was compromised. 
\cpr does so using diffusing computations.  Then, \cpr removes all \badvector and \oldvector state while updating stale distance values resulting from link cost changes that
occurred between the time the snapshot was taken and the ``current'' time.
This last step is executed by initiating a distance vector computation from the neighbors of the compromised node.
%To resolve these issues, each neighbor, $i$, of \bads, sets its distance to \bad to $\infty$ and then selects new least cost values that avoid the compromised node, triggering
%the execution of distance vector to update the remaining distance vectors. 


%{\bf Step 1: Create a \minv and \dmatrix archive.} 
%	We define a  \emph{snapshot} of a data structure to be a copy of all current distance values along with a timestamp.
%	{\footnote {\small In practice, we only archive distance values that have changed. Thus each distance value is associated with its own timestamp.}}
%	The timestamp marks the time at which that set of distance values start being used. 
%	\minv and \dmatrix are the only data structures that need to be archived. This approach is similar to ones used in temporal databases 
%  \cite{Lomet06,Jensen91}.


{%\bf Step 2: Rolling back to a valid snapshot.} 
%Rollback is implemented using diffusing computations initiated at the neighbors of the compromised node. T
%he diffusing computation ensures that all nodes roll back to their most recent snapshot taken before $t'$.
%(Note that this rollback algorithm ensures that no reinstated distance value uses \badvector because every node rolls back to a snapshot with a timestamp less that $t'$. )


%{\bf Step 3: Steps after rollback.} After Step 2 completes, the algorithm in Section \ref{subsec:preprocess} is executed.
%There are two issues to address.
%First, some nodes may be using \oldvectors.  Second, some nodes may have stale state as a result of link cost changes that occurred during $[t',t]$ and 
%consequently are not reflected in the snapshot. 
%To resolve these issues, each neighbor, $i$, of \bads, sets its distance to \bad to $\infty$ and then selects new least cost values that avoid the compromised node, triggering
%the execution of distance vector to update the remaining distance vectors.  






