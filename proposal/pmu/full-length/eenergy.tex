\documentclass{sig-alternate-eEnergy_v2} 
%\documentclass{acm-alternate} 
%\documentclass{acm} 
%\documentclass[conference]{acm} 
%\documentclass[conference]{IEEEtran} 
%\documentclass[10pt,conference]{IEEEtran} 
\usepackage{graphicx}
\usepackage{subfigure}
\usepackage{epsfig}
\usepackage{amssymb,amsfonts,amsmath,amsthm}
\usepackage{verbatim}
\usepackage{moreverb}
\usepackage{rotating}
\usepackage{algorithmic}
\usepackage{epstopdf}
\usepackage{color}

%\usepackage{epstopdf}
%\usepackage{amsmath}
\DeclareGraphicsRule{.tic}{png}{.png}{`convert #1 `dirname #1`/`basename #1 .tif`.png}


\newtheorem{theorem}{Theorem}[section]
\setcounter{theorem}{0}
\newtheorem{algorithm}[theorem]{Algorithm}
\newtheorem{claim}[theorem]{Claim}
\newtheorem{conjecture}[theorem]{Conjecture}
\newtheorem{corollary}[theorem]{Corollary}
\newtheorem{fact}[theorem]{Fact}
\newtheorem{lemma}[theorem]{Lemma}
\newtheorem{meta-proposition}[theorem]{Meta-Proposition}
\newtheorem{note}[theorem]{Note}
\newtheorem{observation}[theorem]{Observation}
\newtheorem{proposition}[theorem]{Proposition}
\newtheorem{proviso}[theorem]{Proviso}         
\newtheorem{question}[theorem]{Question}         
\newtheorem{remark}[theorem]{Remark}         
\newtheorem{define}[theorem]{Definition}         

%\newcommand{\HRule}{\rule{\linewidth}{0.5mm}}
%\setlength\fboxsep{0pt}
%\setlength\fboxrule{0.5pt}


\newcommand{\sat}{{\textsc{P3SAT} }}
\newcommand{\sats}{{\textsc{P3SAT}}}
\newcommand{\full}{{\textsc{FullObserve} }}
\newcommand{\fulls}{{\textsc{FullObserve}}}
\newcommand{\maxinc}{{\textsc{MaxObserve} }}
\newcommand{\maxincs}{{\textsc{MaxObserve}}}
\newcommand{\xval}{{\textsc{FullObserve-XV} }}
\newcommand{\xvals}{{\textsc{FullObserve-XV}}}
\newcommand{\xvalpart}{{\textsc{MaxObserve-XV} }}
\newcommand{\xvalparts}{{\textsc{MaxObserve-XV}}}

\newcommand{\xxx}[1]{\textit{\textcolor{red}{[[#1]]}}} % this is an elisha comment
\newcommand{\yyy}[1]{\textcolor{blue}{#1}} % this is an elisha comment


\makeatletter
\newcommand{\un}[1]{%
    \ifmmode \@@underline{#1} \else %
             $\@@underline{\hbox{#1}}$\fi}
\makeatother
\raggedbottom

\newcommand{\doctitle}{On the Impact of PMU Placement on Observability and Cross-Validation} 



\begin{document}


\title{\doctitle}


\author{
\alignauthor Daniel Gyllstrom, Elisha Rosensweig, and Jim Kurose \\
\affaddr{Department of Computer Science. University of Massachusetts Amherst USA}  \\ 
\email{\{dpg, elisha, kurose\}@cs.umass.edu} 
}

\maketitle

%\section{Abstract}

\begin{abstract}

Malicious and misconfigured nodes can inject incorrect state into a distributed system, which can then be propagated system-wide as a result of normal network operation. 
Such false state can degrade the performance of a distributed system or render it unusable. For example, in the case of network routing algorithms, false state corresponding
to a node incorrectly declaring a cost of $0$ to all destinations (maliciously or due to misconfiguration) can quickly spread through the network. This causes other nodes to (incorrectly) 
route via the misconfigured node, resulting in suboptimal routing and network congestion. We propose three algorithms for efficient recovery in such scenarios, prove the correctness 
of each of these algorithms, and derive communication complexity bounds for each algorithm. Through simulation, we evaluate our algorithms -- in terms of message and time overhead -- when 
applied to removing false state in distance vector routing. Our analysis
shows that over topologies where link costs remain fixed and for the same topologies where link costs change, a recovery algorithm based on system-wide checkpoints and a rollback mechanism 
yields superior performance when using the poison reverse optimization.

\end{abstract}
%We propose algorithms that allow distributed routing algorithms to efficiently recover from false state injected into the network.
%We prove the correctness of each algorithm and evaluate them when applied to distance vector routing. In this context, we evaluate the message and time complexity of each algorithm through simulation.

%{\bf Keywords}: {\it distributed algorithms, fault tolerance, routing, security}


\category{C.2.0}{General}{Security and protection}
%\category{F.2.0}{General}{Redundant Design}{Security and protection}
%\category{B.4.5}{Reliability, Testing, and Fault-Tolerance}{Redundant Design}

\terms{Reliability}

\keywords{smart grid, energy, PMU, NP-Complete, cross-validation}


\unnumberedchapter{Introduction}

Communication network components (routers, links, and sensors) fail. %over the lifetime of the network 
These failures can cause widespread network service disruption and outages, and potentially critical errors for network applications.
\textit{In this thesis, we examine how networks -- traditional networks and networked cyber-physical systems, such as the smart grid -- can be made more robust to component failure.}

We propose on-demand recovery algorithms for distributed network algorithms that optimize for control message overhead and convergence time,
and preplanned approaches to recovery for electric power grid applications, where reliability is key. 
An electric power grid consists of a set of buses -– electric substations, power generation centers, or aggregation points of electrical loads -– and transmission lines connecting those buses.
We refer to modern and future electric power grids that automate power grid operations using sensors and wide-area communication as the \emph{smart grid}.




\section{Thesis Overview}

\subsection{Component Failure in Communication Networks}

In this thesis, we consider three separate but related problems: node (i.e., switch or router) failure in traditional networks such as the Internet or wireless sensor networks,
the failure of critical sensors that measure voltage and current throughout the smart grid, and link failures in a smart grid communication network.
For distributed network algorithms, a malicious or misconfigured node can inject and spread incorrect state throughout the distributed system. 
Such false state can degrade the performance of the network or render it unusable. For example, in 1997 a significant portion of Internet traffic was routed through a 
single misconfigured router that had spread false routing state to several Internet routers.  As a result, a large potion of the Internet became inoperable for several hours \cite{Neumann97}. 


%In a smart grid, especially, component failure can be catastrophic.
Component failure in a smart grid can be especially catastrophic.
For example, if smart grid sensors or links in its supporting communication network fail, smart grid applications can make incorrect decisions and take corresponding (incorrect) actions. 
Critical smart grid applications required to operate and manage a power grid are especially vulnerable to such failures because typically these applications have strict data delivery requirements,
needing both ultra low latency and assurance that data is received correctly. 
In the worst case, component failure can lead to a cascade of power grid failures like the August 2003 blackout in the USA \cite{2003Blackout} and the 
recent power grid failures in India \cite{IndiaBlackout}.






\subsection{Approaches to Making Networks More Robust to Failures}



For many distributed systems, recovery algorithms operate on-demand (as opposed to being preplanned) because algorithm and system state is typically distributed throughout the network of nodes.  
As a result, fast convergence time and low control message overhead are key requirements for efficient recovery from component failure. 
In order to make the problem of on-demand recovery in a distributed system concrete, we investigate distance vector routing as an instance of this problem where nodes must recover
from incorrectly injected state information.
Distance vector forms the basis for many routing algorithms widely used in the Internet (e.g., BGP, a path-vector algorithm) and in multi-hop wireless networks (e.g., AODV, diffusion routing).

In the first technical chapter of this thesis, we design, develop, and evaluate three different approaches for correctly recovering from the injection of false distance vector routing state (e.g., a compromised node incorrectly
claiming a distance of $0$ to all destinations). Such false state, in turn, may propagate to other routers through the normal execution of distance vector routing, causing other nodes to (incorrectly) route via the misconfigured node, making this a network-wide problem.
Recovery is correct if the routing tables of all nodes have converged to a global state where, for each node, all compromised nodes are removed as a destination and
no least cost path routes through a compromised node.  
%Recovery is correct if the routing tables of all nodes have converged to a global state meeting the following conditions.  
%For each node, all compromised nodes are removed as a destination and no least cost path routes through a compromised node.  
%Recovery is correct if the routing tables of all nodes have converged to a global state where all nodes have removed each compromised node as a destination
%and no node has a least cost path to any destination that routes through a compromised node.  

The second and third thesis chapters consider robustness from component failure  in the context of the smart grid. Because reliability is a key requirement for the smart grid, we focus on 
preplanned approaches to failure recovery.

In our second thesis chapter, we study a sensor, called a Phasor Measurement Unit (PMU), currently being deployed in electric power grids worldwide. 
PMUs provide voltage and current measurements at a sampling rate orders of magnitude higher than the status quo.  As a result, PMUs can 
both drastically improve existing power grid operations and enable an entirely new set of applications, such as the reliable integration of renewable energy resources. 
We formulate a set of problems that consider PMU measurement errors, which have been observed in practice.  Specifically, we specify four PMU placement problems
that aim to satisfy two constraints: place PMUs ``near'' each other to allow for measurement error detection and use the minimal number of PMUs to infer the state of the maximum number of system buses and transmission lines. 
For each PMU placement problem, we prove it is NP-Complete, propose a simple greedy approximation algorithm, and evaluate our greedy solutions.

In our final technical chapter, we design algorithms that provide fast recovery from link failures in a smart grid communication network. 
We propose, design, and evaluate solutions to all three aspects of link failure recovery: (a) link failure detection, (b) algorithms for pre-computing backup multicast trees, and
(c) fast backup tree installation. Because this requires modifying network switches and routers, we use OpenFlow -- an open standard that cleanly separates the control 
and data planes for use in network management and control -- to program data plane forwarding.  This allows our algorithms to execute in the data plane.


To address (a), we design link-failure detection and reporting mechanisms that use OpenFlow to detect link failures when and where they occur \emph{inside} the network.
For part (b), we formulate a new problem, \mcs, that pre-computes backup multicast trees with the aim of minimizing control plane signaling overhead. We prove \mc 
is at least NP-hard and present a corresponding approximation algorithm.
Lastly, two control plane algorithms are proposed that signal data plane switches to install pre-computed backup trees. 
An optimized version of each installation algorithm is designed that finds a near minimum set of forwarding rules 
by sharing forwarding rules across multicast groups. This optimization
%by using OpenFlow to dynamically write (and delete) identifiers in packet headers to allow forwarding rules to be shared across multicast groups. This optimization
reduces backup tree install time and control state.  
We implement these algorithms using the POX open-source OpenFlow controller and evaluate them using the Mininet emulator. 





\section{Thesis Contributions}


The main contributions of this thesis are:
\begin{itemize}

	\item  We design, develop, and evaluate three different algorithms -- \seconds, \purges, and \cpr -- for correctly recovering from the injection of false routing state in distance vector routing.
		\second performs localized state invalidation, followed by network-wide recovery using the traditional distance vector algorithm. 
		\purge first globally invalidates false state and then uses distance vector routing to recompute distance vectors.  \cpr takes and stores local routing table 
		snapshots at each router, and then uses 
		a rollback mechanism to implement recovery. We prove the correctness of each algorithm for scenarios of single and multiple compromised nodes.

	

	\item We use simulations and analysis to evaluate \seconds, \purges, and \cpr in terms of control message overhead and convergence time. We find that \second performs poorly due to routing loops.  
	Over topologies with fixed link costs, \purge performs nearly as well as \cpr even though our simulations and analysis assume near perfect conditions for \cprs.
	Over more realistic scenarios in which link weights can change, we find that \purge yields lower message complexity and faster convergence time than \cpr and \seconds. 


	\item We define four PMU placement problems, three of which are completely new, that place PMUs at a subset of electric power grid buses. 
		Two PMU placement problems consider measurement error detection by requiring PMUs to be placed ``near'' each other to allow for their measurements to be cross-validated. 
		For each PMU placement problem, we prove it is NP-Complete and propose a simple greedy approximation algorithm. 

	
	\item We prove our greedy approximations for PMU placement are correct and give complexity bounds for each.  Through simulations over synthetic topologies generated using real portions of the 
		North American electric power grid as templates, we find that our greedy approximations yield results that are close to optimal: on average, within $97\%$ of optimal.  We also find that 
		imposing our requirement of cross-validation to ensure PMU measurement error detection comes at small marginal cost: on average, only $5\%$ fewer power grid buses are observed (covered) 
		when PMU placements require cross-validation versus placements that do not. 
	

	\item We propose, implement, and evaluate a suite of algorithms for fast recovery from link failures in a smart grid communication network: \pcnts, \steiners, \pres, \posts, and \merges.
	\pcnt uses OpenFlow to accurately detect link failures inside the network, rather than using slower end-to-end measurements. Then, we define a new problem, \mcs, that computes backup 
	multicast trees with the aim of minimizing control plane signaling overhead. %`multicast trees with the aim of maximizing the reuse of installed multicast trees. 
	This problem is shown to be at least NP-hard, motivating the design of an approximation, \steiners. 
	Next, we design two algorithms -- \pre and \post -- for fast backup tree installation. \pre pre-installs backup tree forwarding rules and
	activates these rules after a link failure is detected, while, \post installs backup trees \emph{after} a link a failure is detected. 
	Lastly, we present \merges, an algorithm that can be applied to \pre and \post to speed backup tree installations and reduce the amount of pre-installed forwarding state. 
	\merge does so using local optimization to create a near minimal set of forwarding rules by ``merging'' forwarding rules in cases where multiple multicast trees have common forwarding behavior.

	\item 
	We use Mininet \cite{Lantz10} simulations to evaluate our algorithms over communication networks based on real portions of the power grid.
	We find that \pcnt provides fast and accurate link loss estimates: after sampling only $75$ packets
	the $95\%$ confidence interval is within $15\%$ of the true loss probability.  
	Additionally, we find \pre yields faster recovery than \post ( 
	\post sends up to $10$ times more control messages than \pres) but at the cost of storage overhead at each switch (pre-installed backup trees can account for as much
	as $35\%$ of the capacity of a conventional OpenFlow switch \cite{Curtis11}).
	Finally, we observe that \merge reduces control plane messaging and the amount of pre-installed forwarding state by a factor of $2$ to $2.5$ when compared to a standard multicast
	implementation, resulting in faster installation and manageable sized flow tables.


	%Lastly, we present \merges, an algorithm that can be applied to \pre and \post to boost performance; 
	%with \posts, \merge speeds backup tree installations and, when used with \pres, reduces the amount of pre-installed forwarding state. 




		

\end{itemize}




\section{Thesis Outline}

The rest of this thesis is organized as follows.  We present algorithms for recovery from false routing state in distributed routing algorithms in Chapter \ref{ch:rollback}.  
In Chapter \ref{ch:pmu}, we formulate PMU placement problems that provide measurement error detection.  Chapter \ref{ch:reliable-mcast} presents our algorithms for fast recovery 
from link failures in a smart grid communication network.  We conclude, in Chapter \ref{ch:conclusion}, with a summary and discussion of open problems emerging from this thesis.


%\input{intuition}

\section{Preliminaries}
\label{sec:prelim}

In this section we introduce notation and underlying assumptions (Section \ref{subsec:notation-assume}), 
and define our observability (Section \ref{subsec:observe}) and cross-validation (Section \ref{subsec:xval-rules}) rules.

%\xx{Maybe merge Observability Rules and Cross-Validation Rules Sections}

\subsection{Assumptions, Notation, and Terminology}
\label{subsec:notation-assume}


We model a power grid as an undirected graph $G=(V,E)$.  Each $v \in V$ represents a bus. % A bus is either an electrical substation, a power generation center, or an aggregation of loads. 
$V=V_Z \cup V_I$, where $V_Z$ is the set of all zero-injection buses and $V_I$ is the set of all non-zero-injection buses.  A bus is zero-injection if it has no load nor generator \cite{Zhang10}.
All other buses are non-zero-injection, which we refer to as injection buses. Each $(u,v) \in E$ is a transmission line connecting buses $u$ and $v$. 

Consistent with the conventions in \cite{Baldwin93,Brueni05,Abur06,Mili90,Xu04,Xu05}, we make the following assumptions about PMU placements and buses. 
First, a PMU can only be placed on a bus.  Second, a PMU on a bus measures the voltage phasor at the bus and the current phasor of all transmission lines connected to it.

Using the same notation as Brueni and Heath \cite{Brueni05}, we define two $\Gamma$ functions. For $v\in V$ let $\Gamma(v)$ be the set of $v$'s neighbors in $G$, and $\Gamma[v] = \Gamma(v)\cup \{v\}$. 
A PMU placement $\Phi_G \subseteq V$ is a set of nodes at which PMUs are placed,
and $\Phi^R_G\subseteq V$ is the set of observed nodes for graph $G$ with placement $\Phi_G$ (see definition of observability below).
$k^* = \min \{|\Phi_G|:\Phi^R_G=V\}$ denotes the minimum number of PMUs needed to observe the entire network. Where the graph $G$ is clear from the context, we drop the $G$ subscript.

For convenience, we refer to any node with a PMU as a \emph{PMU node}. Additionally, for a given PMU placement we shall say that a set $W\subseteq V$ is observed if all nodes in the set are observed, and if $W=V$ we refer to the graph as \emph{fully observed}. 

\subsection{Observability Rules}
\label{subsec:observe}

We use the simplified observability rules stated by Brueni and Heath \cite{Brueni05}. For completeness, we restate the rules here:
\begin{enumerate}
	
	\item {\bf Observability Rule 1 (O1)}.  {\it If node $v$ is a PMU node, then $v \cup \Gamma(v)$ is observed. } %Formally, if $v \in \Phi_G$, then $\Gamma[v] \subseteq \Phi^R_G$. }

	\item {\bf Observability Rule 2 (O2)}. {\it If a zero-injection node, $v$, is observed and  $\Gamma(v)\backslash\{u\}$ is observed for some $u\in\Gamma(v)$, then $v \cup \Gamma(v)$ is observed.}
	%Formally, if $v \in \Phi^R_G \cap V_Z$ and $|\Gamma(v) \cap (V - \Phi^R_G)| \leq 1$, then $\Gamma[v] \subseteq \Phi^R_G$. }

\end{enumerate}

Consider the example in Figure \ref{fig:pmu-example}, where the shaded nodes are PMU nodes and $g$ is the only zero-injection node. 
Nodes $a-d$ are observed by applying O1 at the PMU at $a$, and nodes $a,b,f$ and $g$ are observed by applying O1 at $b$. 
$e$ cannot be observed via $c$ because $c$ does not have a PMU (O1 does not apply) and is an injection node (O2 does not apply). 
Similarly, $j$ is not observed via $f$. Finally, although $g \in V_Z$, O2 cannot be applied at $g$ because $g$ has two unobserved neighbors $i,h$, so they remain unobserved.

Since O2 only applies with zero-injection nodes, the number of zero-injection nodes can greatly affect system observability. 
For example, consider the case where $c$ and $f$ are {\em zero-injection} nodes. $a-d$, $g$ and $f$ are still observed as before, as O1 makes not conditions on the node type. Additionally, 
since now $c,f \in V_Z$ and each has a single unobserved neighbor,  we can apply O2 at each of them to observe $e,j$, respectively. % making $e$ and $j$, respectively, observed.   
We evaluate the effect of increasing the number of zero-injection nodes on observability in our simulations (Section \ref{sec:simulations}).


\begin{figure}[t]
\centering
\includegraphics[scale=.6]{figs/pmu-example.pdf}
%\includegraphics[scale=0.51]{figs/example2.pdf}
\caption{Example power system graph. PMU nodes ($a,b$) are indicated with darker shading. Injection nodes have solid borders while zero-injection nodes  ($g$) have dashed borders.}
\label{fig:pmu-example}
\end{figure}

\subsection{Cross-Validation Rules}
\label{subsec:xval-rules}

Cross-validation formalizes the intuitive notion of placing PMUs ``near'' each other to allow for measurement error detection. 
From Vanfretti et al. \cite{Vanfretti10}, PMU measurements can be cross-validated when: (1) a 
voltage phasor of a non-PMU bus can be computed by PMU data from two different buses or (2) the current phasor of a transmission line can be computed from PMU data from two different buses. 
{\footnote {Vanfretti et al. \cite{Vanfretti10} use the term ``redundancy'' instead of cross-validation. }  


%`A PMU node is cross-validated if {\em the PMU node is within two hops of another PMU node}. 
For convenience, we say a PMU is cross-validated even though it is actually the PMU data at a node that is cross-validated.
A PMU is \emph{cross-validated} if one of the rules below is satisfied \cite{Vanfretti10}: 
%More formally, a PMU is \emph{cross-validated} if one of the rules below is satisfied \cite{Vanfretti10}: 
\begin{enumerate}
	
	\item {\bf Cross-Validation Rule 1 (XV1)}.  {\it If two PMU nodes are adjacent, then the PMUs cross-validate each other. }
	%Formally, if $u, v \in \Phi_G$, $u \in \Gamma(v)$, then the PMUs at $u$ and $v$ are cross-validated.}

	\item {\bf Cross-Validation Rule 2 (XV2)}. {\it If two PMU nodes have a common neighbor, then the PMUs cross-validate each other.}
	%Formally, if $u, v \in \Phi_G$, $u\neq v$ and $\Gamma(u)\cap\Gamma(v)\neq\emptyset$, then the PMUs at $u$ and $v$ are cross-validated.}
\end{enumerate}
In short, the cross-validation rules require that {\em the PMU is within two hops of another PMU}.
For example, in Figure \ref{fig:pmu-example}, the PMUs at $a$ and $b$ cross-validate each other by XV1.

XV1 derives from the fact that both PMUs are measuring the current phasor of the transmission line connecting the two PMU nodes.  XV2 is more subtle.  
Using the notation specified in XV2, when computing the voltage phasor of an element in $\Gamma(u)\cap\Gamma(v)$ the voltage equations include variables to 
account for measurement error (e.g., angle bias) \cite{Vanfretti-thesis}. 
When the PMUs are two hops from each other (i.e., have a common neighbor), there are more equations than unknowns, allowing for measurement error detection. 
Otherwise, the number of unknown variables exceeds the number of equations, which eliminates the possibility of detecting measurement errors \cite{Vanfretti-thesis}.







%\input{problem-stmt-long}
\section{Four NP-Complete PMU Placement Problems}
\label{sec:problem-analysis}

In this section we first define four PMU placement problems and then provide a high-level description of the proof strategy we use to prove each problem is NP-Complete.
\yyn{In all four problems defined in this paper, we are only concerned with computing the voltage phasors of each bus (i.e., observing the buses). 
Using the values of the voltage phasors, Ohm's Law can be easily applied to compute the current phasors of each transmission line.}
Also, we consider networks with both injection and zero-injection buses.


\subsection{Problem Statements}

Here we briefly define each of our four PMU placement problems: \fulls, \maxinc, \xvalparts, and \xvals. \\ 
{\bf \full Decision Problem:} \\
\indent \underline{Instance}: Graph $G=(V,E)$ where $V=V_Z \cup V_I$, $V_Z \neq \emptyset$, $k$ PMUs such that $k \geq 1$. \\
\indent \underline{Question}: Is there a $\Phi_G$ such that $|\Phi_G| \leq k$ and $\Phi^R_G = V$?  \\
{\bf \maxinc Decision Problem:} \\
\indent \underline{Instance}: Graph $G=(V,E)$ where $V=V_Z \cup V_I$, $k$ PMUs such that $1 \leq k < k^*$. \\
\indent \underline{Question}: For a given $m< |V|$, is there a $\Phi_G$ such that $|\Phi_G| \leq k$ and $m \leq |\Phi^R_G| < |V|$? \\
{\bf \xval Decision Problem:} \\
\indent \underline{Instance}: Graph $G=(V,E)$ where $V=V_Z \cup V_I$, $k$ PMUs such that $k \geq 1$. \\
\indent \underline{Question}: Is there a $\Phi_G$ such that $|\Phi_G| \leq k$ and $\Phi^R_G = V$ under the condition that each $v \in \Phi_G$ is cross-validated? \\
{\bf \xvalpart Decision Problem:} \\
\indent \underline{Instance}: Graph $G=(V,E)$ where $V=V_Z \cup V_I$, $k$ PMUs such that $1 \leq k < k^*$, and some $m<|V|$. \\
\indent \underline{Question}: Is there a $\Phi_G$ such that $|\Phi_G| \leq k$ and $m \leq|\Phi^R_G| < |V|$ under the condition that each $v \in \Phi_G$ is cross-validated?


\begin{theorem}
\maxincs, \xvalparts, \full and \xval are all NP-Complete.
\label{thm:pmu-npc}
\end{theorem}



\subsection{Overview of NPC Proof Strategy}
\label{subsec:proofstrat}
In this section, we outline the proof strategy we used in each of NP-Completeness proofs.  Due to space constraints we omit the actual NPC proofs. 
Our proofs follow a similar structure proposed by Brueni and Heath \cite{Brueni05}. The authors prove NP-Completeness by reduction from planar 3-SAT (\sats).
A 3-SAT formula, $\phi$, is a boolean formula in conjunctive normal form (CNF) such 
that each clause contains at most $3$ literals. For any 3-SAT formula $\phi$ with the sets of variables $\{v_1,v_2, \dots , v_r\}$ and clauses $\{c_1,c_2, \dots , c_s \}$, $G(\phi)$ 
is the bipartite graph $G(\phi)=(V(\phi),E(\phi))$ defined as follows:
\begin{eqnarray*}
 V(\phi) &= &\{v_i\; \vert\; 1 \leq i \leq r \} \cup \{c_j \;\vert\; 1 \leq j \leq s \} \\
 E(\phi) &=& \{ (v_i,c_j)\;\vert\; v_i \in c_j\;\; or \;\; \overline{v_i} \in c_j\}.
\end{eqnarray*}
Note that edges pass only between $v_i$ and $c_j$ nodes, and so the graph is bipartite.  \sat is a 3-SAT formula such that $G(\phi)$ is planar \cite{Lich82}. 
For example, \sat formula
	 $\varphi = (\overline{v_1} \vee v_2 \vee v_3) \wedge (\overline{v_1} \vee \overline{v_4} \vee v_5) \wedge (\overline{v_2} \vee \overline{v_3} \vee \overline{v_5}) 
	 \wedge (v_3 \vee \overline{v_4}) \wedge  (\overline{v_3} \vee v_4 \vee \overline{v_5})$
has graph $G(\varphi)$ shown in Figure \ref{fig:gvarphi}. 
Discovering a satisfying assignment for  \sat is an NPC problem, and so it can be used in a reduction to prove the complexity of the problems we address here. 

Following the approach in \cite{Brueni05}, for \sat formula, $\phi$, we replace each variable node and each clause node in $G(\phi)$ with a specially constructed set of nodes,
termed a {\em gadget}. In this work, all variable gadgets will have the same structure, and all clause gadgets have the same structure (that is different from the variable gadget structure), 
and we denote the resulting graph as $H(\phi)$. In $H(\phi)$, each {\em variable} gadget has a subset of nodes that semantically represent assigning ``True" to that variable, and a subset of 
nodes that represent assigning it ``False". When a PMU is placed at one of these nodes, this is interpreted as assigning a truth value to the \sat variable corresponding with that gadget. 
Thus, we use the PMU placement to determine a consistent truth value for each \sat variable. Also, clause gadgets are connected to variable gadgets at either ``True" or ``False" (but never both) 
nodes, in such a way that the clause is satisfied if and only if {\em at least one} of those nodes has a PMU.

%While we assume $G(\phi)$ is planar, we make no such claim regarding $H(\phi)$, though in practice all graphs used in our proofs are indeed planar. The proof of NPC rests on the fact that 
%solving the underlying $\phi$ formula is NPC.

While the structure of our proofs is adapted from \cite{Brueni05}, the variable and clause gadgets we use to correspond to the \sat formula are novel, thus leading to a 
different set of proofs. Our work here demonstrates how the work in \cite{Brueni05} can be extended, using new variable and clause gadgets, to address a wide array of PMU placement problems.

\begin{figure}[t]

\subfigure[$G(\varphi)=(V(\varphi),E(\varphi))$ formed from $\varphi$ in Equation (\ref{eqn:varphi}).]{\label{fig:gvarphi}\includegraphics[scale=0.53]{figs/gvarphi.pdf}}
\subfigure[Graph formed using variable gadget from ...]{\label{fig:varphi2}\includegraphics[scale=0.53]{figs/proof1-inject-example.pdf}}

\caption{Example } 
\end{figure}


\begin{figure}[t]
\centering
\includegraphics[scale=0.53]{figs/proof1-inject-example.pdf}
%\includegraphics[scale=0.51]{figs/example2.pdf}
\caption{Graph $G=(V,E)=H_1(\varphi)$ formed from $\varphi$ formula in Theorem \ref{thm:npc-full} proof. Nodes with a dashed border are zero-injection nodes.}
\label{fig:proof1-inject-example}
\end{figure}

\begin{figure}[t]
    \fbox{\subfigure[Variable gadget $V_i$ used in Theorem \ref{thm:npc-full} and Theorem \ref{thm:npc-maxinc}.]
	{\label{fig:diamond-gadget}\includegraphics[scale=0.39]{figs/diamond-gadget.pdf}}}
    \fbox{\subfigure[Clause gadget $C_j$ used in Theorem \ref{thm:npc-maxinc}.]
	{\label{fig:line-gadget}\includegraphics[scale=0.39]{figs/line-gadget.pdf}}}
	\caption{Gadgets used in Theorem \ref{thm:npc-full} and Theorem \ref{thm:npc-maxinc}. $Z_i$ in Figure (a) is the only zero-injection node. The dashed edges in Figure (a) are connections to clause gadgets.  Likewise, the dashed edges in Figure (b) are connections to variable gadgets. }
  \label{fig:pmu-gadgets}
\end{figure}


%\begin{figure}[t]
%\centering
%\includegraphics[scale=0.53]{figs/gvarphi.pdf}
%\caption{$G(\varphi)=(V(\varphi),E(\varphi))$ formed from $\varphi$ in Equation (\ref{eqn:varphi}). }
%\label{fig:gvarphi}
%\end{figure}





\section{Approximation Algorithms}
\label{sec:approx}

Because all four placement problems are NPC, we propose greedy approximation algorithms for each problem, which iteratively add 
a PMU in each step to the node that observes the maximum number of new nodes. We present two such algorithms, one that directly addresses \maxinc ({\tt greedy}) and the other 
\xvalpart ({\tt xvgreedy}). {\tt greedy} and {\tt xvgreedy} can easily be used to solve \full and \xvals, respectively, by selecting the appropriate $k$ value to ensure full observability.

{\bf {\tt greedy} Algorithm}. We start with $\Phi = \emptyset$.  At each iteration, we add a PMU to the node that results in the observation of the maximum number of 
new nodes. The algorithm terminates when all PMUs are placed.  {\footnote {\small The same greedy algorithm is proposed by Aazami and Stilp \cite{Aazami07}. }}


{\bf {\tt xvgreedy} Algorithm}. {\tt xvgreedy} is almost identical to {\tt greedy}, except that PMUs are added in pairs such that the selected pair observe
the maximum number of nodes under the condition that the PMU pair satisfy one of the cross-validation rules. % and observe the maximum number of new nodes.
%We provide the pseudo code for {\tt xvgreedy} and prove that {\tt xvgreedy} has polynomial running time in our Technical Report \cite{Tech12}. 

%Our Technical Report \cite{Tech12} gives the pseudo code for {\tt greedy} and {\tt xvgreedy} and includes proofs
%that these algorithms have polynomial complexity, making them feasible tools for approximating optimal PMU placement. 

\xxn{Aazami and Stilp prove {\tt greedy} has a $\Theta(n)$ approximation ratio under the assumption that all nodes are zero-injection.}



\subsection{The Observability Rules are Not Submodular Function}

Intuitively, submodular functions are set functions with diminishing marginal returns: the value that a single element makes in the function when added to the input set decreases
as the size of the input set increases. More formally, let $X$ be a ground set such that $|X|=n$. We define a set function on $X$ as $f: 2^X \rightarrow \mathbb{R}$.
We define a function as submodular using the definition from \footnote{\url{http://theory.stanford.edu/~shaddin/papers/submodular\_survey.pdf}}.
A set function is $f: 2^X \rightarrow \mathbb{R}$ is \emph{submodular} if, for all $A,B \subseteq X$ with $A \subseteq B$, and for each $j \in X$,
\begin{eqnarray}
f(A \cup \{j\}) - f(A) &\geq& f(B \cup \{j\}) - f(B)
\end{eqnarray}

For the PMU placement problems, we define $f(X)$ on graph, $G=(V,E)$, as the number of nodes observed by placing a PMU at each $x \in X$.  Using the example graph from
Figure \ref{fig:submodular-counter}}, we show $f$ is not submodular.  Let $A=\{a\}$ and $B=\{a,b\}$. Then,
%In this example we let $A=\{a\}$ and $B=\{a,b\}$. Then, %We show $f$ is not a submodular function
\begin{eqnarray*}
f(A \cup \{c\}) - f(A) &\stackrel{?}{\geq}& f(B \cup \{c\}) - f(B) \\
f(A \cup \{c\}) - 2 &\stackrel{?}{\geq}& f(B \cup \{c\}) - 3 \\
3-2 &\stackrel{?}{\geq}& 8 - 3 \\
1 &\stackrel{?}{\geq}& 5
\end{eqnarray*}
As a result, we conclude that $f$ is not submodular and therefore that our observability rules are not submodular functions.

Note that in this example, O2 prevented us from meeting the criteria for submodular functions.  For PMU placement $B \cup \{c\}$, we are able to apply O2 at $e$, resulting in the observation of the
chain of nodes at the top of the graph.  However, we were unable to apply O2 for PMU placement $A \cup \{c\}$.
  
%It follows that $f(A)=2$ and $f(B)=3$. 

\begin{figure}[t]
\centering
\includegraphics[scale=.75]{figs/submodular-counterexample.pdf}
\caption{Example showing the observability rules are not submodular functions.}
\label{fig:submodular-counter}
\end{figure}

\xxn{Maybe we can show it is a submodular function for certain distributions of zero-injection nodes.}


%\subsection{Example Showing Greedy is not a Submodular Function}
%
%Submodular definition from \footnote{\url{http://theory.stanford.edu/~jvondrak/CS369P-files/lec16.pdf}}.  Denote $f_A(i) = f(A+i) - f(A)$ the marginal value of $i$ with respect
%to $A$.  $f$ is submodular if for all $A \subseteq B \subseteq N$ and $i \in N \setminus B$, 
%$$ f_A(i) \geq f_B(i) $$
%
%
%Execution of {\tt greedy} using Figure \ref{fig:submodular-counter}:
%\begin{enumerate}
%	\item Add PMU to $d$.  As a result, $5$ nodes, $\{a,b,c,d,h\}$, are observed by applying O1 at $d$,  O2 cannot be applied.
%	
%	\item Add PMU to $f$.  As a result, $6$ nodes become observed $\{e,f,g,i,j,k\}$ by applying O1 at $f$ and then repeatedly applying O2 (first at $h$, then at $j$, and finally at $k$)
%
%\end{enumerate}
%
%
%\begin{figure*}[t]
%  \begin{center}
%    \subfigure[The original graph (without any PMUs).]{\label{fig:submodular-counter-step0}
%		\includegraphics[scale=0.59]{figs/submodular-counterexample-step0.pdf}}
%    \subfigure[Step 1: Placing a PMU at $d$ results in the observation of $5$ nodes, $\{a,b,c,d,h\}$.]{\label{fig:submodular-counter-step1}
%		\includegraphics[scale=0.59]{figs/submodular-counterexample-step1.pdf}}
%    \subfigure[Step 2: Placing a PMU at $f$ results in the observation of $6$ nodes, $\{e,f,g,i,j,k\}$.]{\label{fig:submodular-counter-step2}
%		\includegraphics[scale=0.59]{figs/submodular-counterexample-step2.pdf}}
%  \end{center}
%	\caption{Example showing that {\tt greedy} is not a submodular function.} 
%	\label{fig:submodular-counter-old}
%\end{figure*}
%
%\end{comment}


%List of references explaining submodular functions.
%\footnote{Nice definition and examples. \url{http://theory.stanford.edu/~jvondrak/CS369P-files/lec16.pdf}}
%\footnote{Survey paper. \url{http://theory.stanford.edu/~shaddin/papers/submodular_survey.pdf}}
%\footnote{\url{http://en.wikipedia.org/wiki/Submodular_set_function}}
%\footnote{One of the earlier papers on submodular functions. \url{http://www.cs.toronto.edu/~eidan/papers/submod-max.pdf}}


 


\section{Simulations}
\label{sec:simulations}

\textbf{Topologies.} We evaluate our approximation algorithms using simulations over IEEE topologies as well as synthetic ones. 
As is standard practice in the literature \cite{Baldwin93,Abur06,Mili90,Xu04}, we use IEEE bus systems $14$, $30$, $57$, and $118$
{\footnote {\small http://www.ee.washington.edu/research/pstca/}}.  The bus system number indicates the number of nodes in the graph (e.g., bus system $57$ has $57$ nodes).
Synthetic graphs are then generated based on each of these topologies, and are used to quantify the performance of our greedy approximations.  
We use synthetic topologies in order to establish the statistical significance of our results.
%For each algorithm, we determine the number of nodes that are observed by placing $k$ PMUs on IEEE bus systems $14$, $30$, $57$, and $118$.
%{\footnote {\small http://www.ee.washington.edu/research/pstca/}} as well as synthetic graphs generated by using these IEEE graphs as templates.

Since observability is determined by the connectivity of the graph, we use the {\em degree distribution} of IEEE topologies as the template for generating our synthetic graphs.
%In Section \ref{subsec:ieee}, we compare the results we get for the synthetic graphs with those of the IEEE topologies.
A synthetic topology is generated from a given IEEE graph by randomly ``swapping'' edges in the IEEE graph. Specifically, we select a random $v \in V$ and then pick a random $u \in \Gamma(v)$. 
Let $u$ have degree $d_u$.  Next, we select a random $w \notin \Gamma(v)$ with degree $d_w = d_u -1$. % {\footnote {\small Here ``random'' means uniformly at random.}
Finally, we remove edge $(v,u)$ and add $(v,w)$, thereby preserving the node degree distribution.
We continue this swapping procedure until the original graph and generated graph share {\em no edges}, and then return the resulting graph.
%Note that this edge-swapping procedure ensures that the degree distribution of each generated graph is identical to the degree distribution of the original bus system.

\textbf{Evaluation Methods.}
We are interested in evaluating how close our algorithms are to the optimal PMU placement. 
%Ideally, we would like to compare the performance of our greedy algorithms with the optimal PMU placement.
Thus, when computationally possible (for a given $k$) we use brute-force algorithms to iterate over all possible placements of $k$ PMUs in a given graph and select the best PMU placement. 
When the brute-force algorithm is computationally infeasible, we present only the performance of the greedy algorithm.
{\footnote {\small Because of the computational cost of the brute-force algorithm for larger topologies (e.g., IEEE bus system $300$ and other more complicated systems),
we have no reference in which to measure the effectiveness of our greedy approximations for larger scale systems. 
Therefore, we do not present results for topologies larger than IEEE bus $118$. }}
In what follows, the output of the brute-force algorithm is denoted {\tt optimal}, and when we require cross-validation it is denoted {\tt xvoptimal}.
%For \xval and \xvalparts, we do the same while ignoring PMU placements that do not meet cross-validation requirements. We use {\tt xvoptimal} to refer to this algorithm.
%For \xval and \xvalparts, we ignore PMU placements which do not meet the cross-validation rules described in Section \ref{subsec:xval}.

%For large graphs, the exponential runtime of this brute force algorithm makes computation infeasible. Thus, we have no optimal results for IEEE bus systems $57,118$ for large $k$  and their corresponding synthetic graphs. For this reason, the {\tt optimal} and {\tt xvoptimal} curves stop abruptly in the corresponding plots. %Figure \ref{fig:bus57}|, \ref{fig:bus118}, Figure \ref{fig:xvbus118}, and Figure \ref{fig:all118}.

We present three different simulations in Section \ref{subsec:synth}-\ref{subsec:ieee}. 
%In Sections \ref{subsec:synth}-\ref{subsec:ieee} we investigate the performance of our algorithms as well as the network PMU requirements.  
In Section \ref{subsec:synth} we consider performance as a function of the number of PMUs, and in Section \ref{subsec:zero} we investigate the performance impact of the number of zero-injection nodes in the network. These two sections use synthetic graphs. We conclude in Section \ref{subsec:ieee}, where we compare these results to the performance over the actual IEEE graphs.

\subsection{Simulation 1: Impact of Number of PMUs}
\label{subsec:synth}

In the first simulation scenario we vary the number of PMUs and determine the number of observed nodes in the synthetic graph.  %We do so for both with and without cross-validation.
Each data point is generated as follows. For a given number of PMUs, $k$, we generate a graph, place $k$ PMUs on the graph, and then determine the number of observed nodes. 
We continue this procedure until $[0.9(\overline{x}),1.1(\overline{x})]$ -- where $\overline{x}$ is the mean number of observed nodes using $k$ PMUs -- falls within the $90\%$ confidence interval.

In addition to generating a topology, for each synthetic graph we determined the members of $V_I, V_Z$. These nodes are specified for the original graphs in the IEEE bus system database. Thus, 
we randomly map each node in the IEEE graph to a node in the synthetic graph with the same degree, and then match their membership to either $V_I$ or $V_Z$.

We present here results for solving \maxinc and \xvalparts.  The number of nodes observed given $k$, using {\tt greedy} and {\tt optimal}, are shown in Figure \ref{fig:maxinc-res}, and Figure \ref{fig:xv-res} shows this number for {\tt xvgreedy} and {\tt xvoptimal}. In both sets of plots we show $90\%$
confidence intervals. We omit results for graphs based on IEEE bus $14$ because the same trends are observed.

Our greedy algorithms perform well. On average, {\tt greedy} is within $98.6\%$ of {\tt optimal},
is never below $94\%$ of {\tt optimal}, and in most cases gives the optimal result.
Likewise, {\tt xvgreedy} is never less than $94 \%$ of {\tt xvoptimal} and on average is within $97\%$ of {\tt xvoptimal}. In about about half the cases {\tt xvgreedy} gives the optimal result.
These results suggest that despite the complexity of the problems, a greedy approach can return high-quality results. Note, however, that these statistics do not include performance over
large topologies (i.e., IEEE graphs $57, 118$) when $k$ is large.  It is an open question whether the greedy algorithms used here would do well for larger graphs.
%Typically, greedy algorithms fail because they commit to a choice too early
%and do not reconsider earlier decisions.  Our results suggest that this is not the case for our PMU placement problems.

%However, these statistics do not include a comparison of greedy versus optimal for large $k$ values in the graphs generated from IEEE graphs $57$ and $118$.
%This is the case because the exponential running time of {\tt optimal} and {\tt xvoptimal} for these inputs made computing a result infeasible.
%Therefore, it is unknown if {\tt greedy} and {\tt xvgreedy} yields results as close to optimal for these inputs.

Surprisingly, when comparing our results with and without the cross-validation requirement, we find that the cross-validation constraints have little effect on the number of observed nodes for the same $k$. Our experiments show that on average {\tt xvoptimal} observed only $5\%$ fewer nodes than {\tt optimal}.  Similarly, on average {\tt xvgreedy} observes
 $5.7\%$ fewer nodes than {\tt greedy}. This suggests that the cost of imposing the cross-validation requirement is low, with the clear gain of ensuring PMU correctness across the network.

%when compared to our greedy solution for the corresponding buses. 

%\begin{figure*}[t]
%  \begin{center}
%    \subfigure[Graphs based on IEEE Bus $30$]{\label{fig:bus30}\includegraphics[scale=0.45]{figs/allbus30.pdf}}
%    \subfigure[Graphs based on IEEE Bus $57$]{\label{fig:bus57}\includegraphics[scale=0.45]{figs/allbus57.pdf}}
%    \subfigure[Graphs based on IEEE Bus $118$]{\label{fig:bus118}\includegraphics[scale=0.45]{figs/allbus118.pdf}}
%  \end{center}
%	\caption{Mean number of observed nodes over synthetic graphs -- using {\tt greedy}, {\tt xvgreedy}, {\tt optimal}, and {\tt xvoptimal} -- 
%	when varying number of PMUs. The $90\%$ confidence interval is shown.}
%  \label{fig:maxinc-res}
%\end{figure*}

\begin{figure*}[t]
  \begin{center}
    \subfigure[Graphs based on IEEE bus $30$]{\label{fig:bus30}\includegraphics[scale=0.45]{figs/bus30.pdf}}
    \subfigure[Graphs based on IEEE bus $57$]{\label{fig:bus57}\includegraphics[scale=0.45]{figs/bus57.pdf}}
    \subfigure[Graphs based on IEEE bus $118$]{\label{fig:bus118}\includegraphics[scale=0.45]{figs/bus118.pdf}}
  \end{center}
	\caption{Mean number of observed nodes over synthetic graphs -- using {\tt greedy} and {\tt optimal} -- when varying number of PMUs. The $90\%$ confidence interval is shown.}
  \label{fig:maxinc-res}
\end{figure*}

\begin{figure*}[t]
  \begin{center}
    \subfigure[Graphs based on IEEE bus $30$]{\label{fig:xvbus30}\includegraphics[scale=0.45]{figs/xvbus30.pdf}}
    \subfigure[Graphs based on IEEE bus $57$]{\label{fig:xvbus57}\includegraphics[scale=0.45]{figs/xvbus57.pdf}}
    \subfigure[Graphs based on IEEE bus $118$]{\label{fig:xvbus118}\includegraphics[scale=0.45]{figs/xvbus118.pdf}}
  \end{center}
	\caption{Over synthetic graphs, mean number of observed nodes -- using {\tt xvgreedy} and {\tt xvoptimal} -- as a function of number of PMUs. The $90\%$ confidence interval is shown.}
  \label{fig:xv-res}
\end{figure*}




%\begin{figure}[t]
%\centering
%\includegraphics[scale=0.50]{figs/all118.pdf}
%\caption{Simulation results for \maxinc and \xvalpart using IEEE bus $118$.}
%\label{fig:all118}
%\end{figure}

\subsection{Simulation 2: Impact of Number of Zero-Injection Nodes}
\label{subsec:zero}

Next, we examine the impact of the number of zero-injection nodes ($|V_Z|$) on algorithm performance. 
For each synthetic graph, we run our algorithms for increasing values of $|V_Z|$ and determine the minimum number of PMUs needed to observe all nodes in the graph ($k^*$).
For each $z:=|V_Z|$, we select $z$ nodes uniformly at random to be zero-injection, and the rest are in $V_I$. Because we compute $k^*$ here, we solve \full and \xvals, rather than
\maxinc and \xvalpart as in Simulation 1.

We generate each data point using a similar procedure to the one described in Section \ref{subsec:synth}.  
For each $z$, we generate a graph and determine $k^*$. %the minimum number of PMUs needed to observe all nodes in the graph ($k^*$).  
We then compute $\overline{k^*}$, the mean value of $k^*$ using $|V_Z| = z$.
We continue this procedure until $[0.9(\overline{k^*}),1.1(\overline{k^*})]$ falls within the $90\%$ confidence interval.

Figure \ref{fig:zero57} shows the simulation results for solving \full and \xval on synthetic graphs modeled by IEEE bus $57$. Results for other topologies considered here 
(i.e., $14$, $30$ and $118$) followed the same trend and are thus omitted. Due to the exponential running time of {\tt optimal} and {\tt xvoptimal}, we present here only results of our 
greedy algorithms. 

As expected, increasing the number of zero-injection nodes, for both {\tt greedy} and {\tt xvgreedy}, reduces the number of PMUs required for 
full observability. 
More zero-injection nodes allow O2 to be applied more frequently (Figure \ref{fig:o2}), thereby increasing the number of observed nodes without using more PMUs.
In fact, we found the relationship between $|V_Z|$ to the greedy estimate of $k^*$  to be linear.

The gap in $k^*$ between {\tt greedy} and {\tt xvgreedy} decreases as $z$ grows. {\tt greedy} and {\tt xvgreedy} observe a similar
number of nodes via O2 across all $z$ values: the mean absolute difference in the number of nodes observed by O2 between the two algorithms is only $1.66$ nodes (equivalently, less
than $3\%$ of observed nodes).  
Thus, as $z$ grows the number of nodes observed by O2 accounts for an increasing proportion of all observed nodes (Figure \ref{fig:o2}), causing the gap between {\tt greedy} and {\tt xvgreedy} to shrink.
%This statistic is confirmed visually by referring to Figure \ref{fig:o2} where the error bars (e.g., the $90\%$ confidence interval) for each data point are overlapping.  Thus, the number of nodes observed by O1 
%-- {\tt greedy} observes more nodes via O1 than {\tt xvgreedy} -- accounts for the difference between the two greedy algorithms. 
%We conclude that becuase the number of O2 observed nodes increases super-linearly with $z$ the gap between {\tt greedy} and {\tt xvgreedy} shrinks.


%\begin{figure*}[t]
%  \begin{center}
%    \subfigure[Graphs based on IEEE Bus $30$]{\label{fig:xvbus30}\includegraphics[scale=0.45]{figs/zero30.pdf}}
%    \subfigure[Graphs based on IEEE Bus $57$]{\label{fig:xvbus57}\includegraphics[scale=0.45]{figs/zero57.pdf}}
%    \subfigure[Graphs based on IEEE Bus $118$]{\label{fig:xvbus118}\includegraphics[scale=0.45]{figs/zero118.pdf}}
%  \end{center}
%	\caption{Simulation Results for \full and \xval greedy approximations, in which we vary the number of zero-injection nodes and determine the minimum number of PMUs needed to observe all nodes. 
%	{\tt xvgreedy} refers to the greedy algorithm with cross-validation and {\tt greedy} denotes the greedy algorithm without cross-validation. The $90\%$ confidence interval is shown. }
%  \label{fig:zero}
%\end{figure*}

\begin{figure*}[t]
  \begin{center}
    \subfigure[Simulation 2]{\label{fig:zero57}\includegraphics[scale=0.45]{figs/zero57.pdf}}
    \subfigure[Simulation 2]{\label{fig:o2}\includegraphics[scale=0.45]{figs/o2-b57.pdf}}
    \subfigure[Simulation 3]{\label{fig:single57}\includegraphics[scale=0.45]{figs/bus-single57.pdf}}
  \end{center}
	\caption{Figures (a) and (b) correspond to Simulation 2 using synthetic graphs based on IEEE bus $57$. The $90\%$ confidence interval is shown in Figure (a) and (b). 
	Figure (a) considers the number of PMUs needed for full observability as a function of $|V_Z|$ and Figure (b) shows the number of nodes observed by O2 for different $|V_Z|$ values. 
	Figure (c) displays the number of observed nodes as a function of the number of PMUs, using  IEEE bus $57$. }
	%\caption{Figures (a) and (b) correspond to Simulation 2 using synthetic graphs based on IEEE bus $57$. The $90\%$ confidence interval is shown in Figure (a) and (b). 
	%Figure (a) considers the number of PMUs needed for full observability as a function of $|V_Z|$ and Figure (b) shows the number of nodes observed by O2 for different $|V_Z|$ values. 
	 %Figure (c) displays the number of observed nodes as a function of the number of PMUs, using  IEEE bus $57$. }
  \label{fig:sim23}
\end{figure*}


\subsection{Simulation 3: Synthetic vs Actual IEEE Graphs}
\label{subsec:ieee}
In this section, we compare our results with the performance over the original IEEE systems. 
We assign nodes to $V_Z$ and $V_I$ as specified in the IEEE database files. Our results indicate that the trends we observed over the synthetic graphs apply as well to real topologies.


%When repeating Simulation 1, we find the same trends using the IEEE topologies. %as with the synthetic graphs.
Figure \ref{fig:single57} shows the number of observed nodes for the {\tt greedy},  {\tt xvgreedy}, {\tt optimal},  and {\tt xvoptimal} algorithms %when we vary the number of PMUs
for IEEE bus system $57$. {\tt greedy} and {\tt xvgreedy} observe nearly as many nodes as the corresponding optimal solution.
%For both with and without cross-validation, the greedy algorithm observes nearly as many nodes as the {\tt optimal} solution.
In many cases, greedy yields the optimal placement. %These results are consistent with our findings for IEEE bus system $14$, $30$, and $118$.
%Similarly, when repeating Simulation 2 using the actual IEEE bus systems, we observe the same trends described in Section \ref{subsec:zero}.
Similarly, as with the synthetic graphs, the number of PMUs required to observe all nodes decreases linearly as $|V_Z|$ increases.
{\footnote {\small The same trends are observed using IEEE bus systems $14$, $30$, and $118$.}}

\begin{table}
\begin{center}
\begin{tabular}{|l|l|l|l|l|}
\hline
 &  {\tt greedy} & {\tt xvgreedy}  & {\tt optimal} & {\tt xvoptimal}  \\
\hline \hline
Simulation 1  & $4\%$  & $4.6\%$ & $6\%$ & $7.6\%$  \\
\hline
Simulation 2 & $9.1\%$ & $16.1\%$ & N/A  & N/A  \\
\hline
\end{tabular}
\end{center}
\caption{Mean absolute difference between the computed values from synthetic graphs and IEEE graphs, normalized by the result for the synthetic graph.}
%\caption{Mean absolute difference between the computed values of each synthetic graph data point and corresponding IEEE graph data point.}
\label{tab:diff}
\end{table}

Finally, we consider whether node degree distribution is an appropriate feature for generating topologies similar to their IEEE counterparts.
%evaluate the appropriateness of node degree distribution as a feature for generating synthetic topologies.
To do so, we compare the actual number of observed nodes for synthetic graphs to those over IEEE graphs.  Specifically, we take the mean absolute difference between these two values,
and normalized by the result for the synthetic graph.
%For example, let $n_{G',k}$ be the output of {\tt greedy} for solving \maxinc for synthetic graph $G'$ and $k$, and let $n_{G,k}$ be the same for the corresponding IEEE graph.
For example, let $n_{k}$ be the mean number of observed nodes using {\tt greedy} over all synthetic graphs with input $k$, and let $n_{G,k}$ 
be the output of {\tt greedy} for IEEE graph $G$ and $k$.
We compute $n_{d,k}=(|n_{k}-n_{G,k}|)/n_{k}$.  Finally, we calculate the mean over all $n_{d,k}$.
This process is done for each algorithm we evaluate.
The resulting statistics can be found in Table \ref{tab:diff}.  The small average difference between the synthetic and 
actual IEEE topologies suggests that the node degree distribution of the IEEE graph is an effective feature for generating similar synthetic graphs.
%{\footnote {\small The average difference is larger for Simulation 2 because the $k^*$ value is sensitive to the assignment of zero-injection nodes.  Recall that the zero
%injection nodes are assigned uniformely at random and for the IEEE topology we only have a single  SHOULD AVG OVER MANY ZERO INJECTION NODE ASSIGNMENTS}}


\section{Related Work}
\label{sec:related}

The PMUP problem -- find the minimum number and placement of PMUs to allow a bus system to be fully observable -- is well-studied \cite{Baldwin93,Brueni05,Haynes02, Mili90, Xu04}. 
Although similar, the \maxinc problem differs from the PMUP problem: \maxinc considers the more general case in which a constant number of PMUs are given and the task is to 
place the PMUs such that the maximum number of nodes are observed.
Haynes et al. \cite{Haynes02} and Brueni and Heath \cite{Brueni05} both prove PMUP is NP-Complete.  We leverage these proofs in our NP-Completeness proofs.  

The power systems literature generally ignores the fact that PMUP is NP-Complete because, in practice, power system graphs are small enough to allow for an exact solution to be found.
Xu and Abur \cite{Xu04,Xu05} use integer programming to find the optimal PMU placement when a subset of buses are zero injection. O2 can only be applied to zero injection buses.  As a result,
the PMUP problem is simplified when only some buses are zero injection.
Baldwin et al. \cite{Baldwin93} and Mili et al. \cite{Mili90} use simulated annealing to determine PMU placement. 
%The work of Xu and Abur \cite{Xu04} and Phadke et al. are representive of the power systems approach to the problem: formulate the problem as integer 
%program and use an integer programming solver to find the optimal PMU placement.  

Aazami and Stilp \cite{Aazami07} investigate approximation algorithms for the PMUP problem.  They derive a hardness approximation threshold of $2^{\log^{1 -\epsilon}n}$ for PMUP.  
Also they prove that in the worst case, the same greedy algorithm presented in Section \ref{sec:approx} does no better $\Theta(n)$ of the optimal solution.  
%We leverage this approximation result in proving the approximation ratios of our heuristic-based algorithms.

Chen and Abur \cite{Abur06} and Vanfretti et al. \cite{Vanfretti10} both study the problem of bad PMU data. Chen and Abur \cite{Abur06} formulate their problem differently than \xval and \xvalparts.  
They consider graphs that are already fully observable and then add PMUs to the system to make all existing PMU measurements non-critical 
(a critical measurement is one in which the removal of a PMU makes the system
no longer fully observable). Vanfretti et al. \cite{Vanfretti10} define the cross-validation rules used in this paper.  They also derive a
lower bound on the number of PMUs needed to ensure all PMUs are cross-validated and the system is fully observable. 



%\input{weak}

\section{Thesis Summary}
\label{sec:thesis-summary}

This thesis examined component failures in communication networks and algorithms to make networks robust to these failures.  
Three separate but related problems were considered: node (i.e., switch or router) failure in traditional networks such as the Internet or wireless sensor networks,
the failure of critical sensors that measure voltage and current throughout the smart grid, and link failures in a smart grid communication network.

Chapter \ref{ch:rollback} considered scenarios where a malicious node injects and spreads false routing state throughout a network of routers.
We presented and evaluated three new algorithms -- \seconds, \purges, and \cpr -- for recovery in such scenarios. %from false state in distance vector routing 
Among these algorithms, we found that \cpr -- a checkpoint-rollback based algorithm -- yielded the lowest message overhead and convergence time over topologies
with fixed link weights but at the cost of storage overhead at the routers.
%However, \cpr required that routing table copies are stored at each router while the other two algorithms did not. %and synhcronization using logical clocks.
For topologies where link weights could change, \purge performed best because \purge globally invalidated false routing state, helping \purge avoid the problems that 
plagued \cpr and \seconds: updating large amounts of stale state (\cprs) and the \infinity problem (\seconds).
%Unlike \cprs, \purge had no stale state to update because \purge does not rollback in time.  
%The \infinity problem resulted in high message overhead for \seconds, while \purge eliminated the \infinity problem by globally purging false state before finding new least cost paths.


Next, in Chapter \ref{ch:pmu} we studied PMUs -- critical sensors being deployed in electric power grids worldwide that provide voltage and current measurements to power grid operators -- and 
a set of placement problems that considered detecting PMU measurement errors.  We formulated four PMU placement problems that 
considered two constraints: place PMUs ``near'' each other to allow for measurement error detection and use the minimal number of PMUs to infer the state 
of the maximum number of system buses and transmission lines. Each PMU placement problem was proved to be NP-Complete. As a first step, we proposed and evaluated  
a simple greedy approximation algorithm to each placement problem.  Using simulations based on topologies generated from real portions of the North American electric power grid, we found 
our greedy algorithms consistently reached close-to-optimal performance (on average within $97\%$ of optimal).  
Additionally, our simulations showed that requiring PMUs to placed near each other (in order to detect measurement errors) resulted in only a small decrease in system observability (on average
only $5\%$ fewer buses were observed with this additional constraint), which made for a strong case for imposing this requirement.
%Additionally, results showed that imposing a requirement that PMUs be placed near each other (in order to detect measurement errors) resulted in a small marginal decrease

In our final technical chapter, we designed algorithms that provide fast recovery from link failures in a smart grid communication network. 
We proposed, designed, and evaluated solutions to all three aspects of link failure recovery: link failure detection, algorithms that pre-computed backup multicast trees, and
fast backup tree installation.  Because these algorithms required making changes to network switches, these algorithms used OpenFlow to access and modify the forwarding plane of switches. 


As an alternative to slower algorithms based on end-to-end measurements, we presented \pcnts.  \pcnt used OpenFlow primitives to detect and report link failures inside the network.  
Next, a new problem was formulated, \mcs, that considered computing backup trees that reuse edges of already installed multicast trees as a means to reduce control plane signaling.
\mc was proved to be at least NP-hard so we designed an approximation algorithm for \mcs. Lastly, we presented two algorithms, \pre and \posts, that installed backup trees 
at OpenFlow controlled switches.  As an optimization to \pre and \posts, we designed \merges, an algorithm that consolidated forwarding rules at switches where multiple trees have common children.

These algorithms were evaluated using Mininet simulations and considered communication networks that mirrored the structure of actual portions of the North American power grid.
\pcnt packet loss estimates were accurate when monitoring even a small number of flows over short time window: after sampling only $75$ packets, the $95\%$ confidence interval of \pcnt loss estimates 
were within $15\%$ of the true loss probability. 
\pre had a $10x$ decrease in control messages compared with \post because \pre required only a single control message to install each backup tree since all other rules were pre-installed,
whereas \post had to signal multiple switches to install each backup tree. 
However, \pres's pre-installed forwarding rules accounted for a significant portion of scarce OpenFlow switch table capacity, especially in cases with many multicast groups (up to $35\%$ of
flow table capacity of a standard OpenFlow switch). Fortunately, \merge reduced the amount of pre-installed forwarding state by a factor of $2-2.5$, to acceptable levels.


\section{Future Work}
\label{sec:thesis-future}

Our research in Chapter \ref{ch:rollback} %on recovery from false routing state injected into a network of routers 
only considered a single instance of false state, which was our best guess of the worst possible false routing state (we assumed that the compromised
node falsely claimed the minimum distance to all nodes).  As future work, we are interested in exploring how our algorithms (i.e, \seconds, \purges, and \cprs) 
respond to other possible false state values. Some interesting values include false state that maximizes the effect of the \infinity problem and false state that contaminates a bottleneck link.


%finding the worst possible false state a compromised node can inject. Some options include the minimum distance to all nodes (e.g., our choice for false state used in this paper), 
%state that maximizes the effect of the count-to-∞ problem, and false state that contaminates a bottleneck link.

There are several topics for future work from Chapter \ref{ch:pmu} on PMU placement. The success of the greedy PMU placement algorithms suggests that bus systems have special topological characteristics,
and investigating these properties could provide interesting insight to power grid topologies. 
%As additional item for future work, we would like to evaluate our greedy approximations 
%using the IEEE bus systems used to evaluate our greedy approximations are based on portions of the North American power grid from the 1960s
Because our brute-force optimal algorithm could only produce data points for small inputs, much could be learned by implementing  
the integer programming approach proposed by Xu and Abur \cite{Xu04} to solve \fulls.  This would provide valuable data points to measure the relative performance of {\tt greedy}.


From Chapter \ref{ch:reliable-mcast}, several problems remain unaddressed. One problem of interest is using optimization criteria different from \mcs's objective function 
to compute backup trees and then evaluate \pres, \posts, and \merge performance using these backup trees.  
For example, backup trees may be computed with the goal of protecting against the worst-case impact of a subsequent link failure
by minimizing the maximum number of multicast trees using a single link. %These backup tree help protect against the worst-case impact of a subsequent link failure. 
It is unknown how effective our installation algorithms would be given these types of backup trees. % when given backup trees other than those computed using \steiners.
Measurements using real OpenFlow hardware switches would strengthen our \pcnt processing time and backup tree installation time results, which both suffered from inaccuracies due to Mininet's
performance fidelity issues.  
%Lastly, the complexity of the problem \merge addresses is an open-question: find the minimum number of forwarding rules for a set of multicast trees. 
Lastly, the problem \merge addresses -- find the minimum number of forwarding rules for a set of multicast trees -- has unknown complexity.
We conjectured that this problem is NP-hard in Section \ref{subsubsec:merge-discuss}.




\section{Acknowledgments}
We thank Luigi Vanfretti, David Bertagnolli, and  Dan Brancaccio for helpful discussions about power systems, PMUs, and PMU errors.  
We also thank David Brent for his wisdom. 

\bibliographystyle{plain}
\bibliography{eenergy}

\end{document}


% (1) give decsription of PMUP problem 
%
%
%

%shortening ideas
%	1. merge approx algs and simulations
%	2. merge plots
%	3. 
%
%

