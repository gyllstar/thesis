\section{Conclusions and Future Work}
\label{sec:conclude}

In this work, we formulated four PMU placement problems and proved that each one is NPC. 
Consequently, future work should focus on developing approximation algorithms for these problems.  As a first step, we presented two simple greedy algorithms:
{\tt xvgreedy} which considers cross-validation and {\tt greedy} which does not.  Both algorithms iteratively add PMUs to the node which observes the maximum of number of nodes. 

Using simulations, we found that our greedy algorithms consistently reached close-to-optimal performance.  Our simulations also
showed that the number of PMUs needed to observe all graph nodes decreases linearly as the number of zero-injection nodes increase. Finally, we found
that cross-validation had a limited effect on observability: for a fixed number of PMUs, {\tt xvgreedy} and {\tt xvoptimal} observed only $5\%$ fewer nodes than
{\tt greedy} and {\tt optimal}, respectively. As a result, we believe imposing the cross-validation requirement on PMU placements is advised, as the benefits they provide come at a low marginal cost.

There are several topics for future work. The success of the greedy algorithms suggests that bus systems have special topological characteristics, 
and we plan to investigate their properties. Additionally, we intend to implement the integer programming approach proposed by Xu and Abur \cite{Xu04} to solve \fulls.  This would
provide valuable data points to measure the relative performance of {\tt greedy}.
%allowing allowing for a comparison with {\tt greedy} to solve \fulls.

