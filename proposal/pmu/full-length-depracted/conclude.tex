\section{Conclusions and Future Work}
\label{sec:conclude}

In conclusion, we have formulated three PMU placement problems: \maxincs, \xvals, and \xvalparts. \maxinc aims to observe the maximum number of nodes given a constant number of PMUs.
The observability rules from Brueni and Heath \cite{Brueni05} were used to determine when the voltage phasor of non-PMU nodes can estimated.
%which nodes  become observed based on a PMU placement. 
Like \maxincs, \xval and \xvalpart 
have the goal of observing the maximum number of nodes but also require that PMUs be placed near each other so their measurements can be validated.  We called this 
cross-validation, a term which we formalized. Cross-validation is possible when two or more PMUs measure the current on the same transmission line or the voltage of the same (adjacent) bus.

We proved that \maxincs, \xvals, and \xvalpart are all NP-Complete.  For this reason, we presented a simple greedy algorithm which iteratively adds a PMU to the node which 
observes the maximum of number of nodes.  In a simulation study, we compared our greedy algorithm with a brute-force optimal algorithm over several
IEEE bus systems and synthetic graphs.  For all three placement problems, our greedy algorithm, on average, yielded a PMU placement within $96\%$ of optimal.

As future work, we plan to derive theoretical bounds on worst case performance of each greedy algorithm.
Also, it would be interesting to relax our assumption that all buses are zero injection and evaluate our greedy algorithms over graphs with a 
mixture of zero and non-zero injection buses. 
%As future work, we plan to derive theoretical bounds on worst case performance .  In particular, proving the performance ratio of each greedy algorithm would 
%strengthen our analysis. Also, it would be interesting to relax our assumption that all buses are zero injection and evaluate our greedy algorithms over graphs with a 
%mixture of zero and non-zero injection buses. 

