\section{Introduction}
\label{sec:intro}

% protection and control during abnormal conditions + help monitoring the system state`

Significant investments have been made to deploy phasor measurement unit (PMUs) on electric power grids worldwide. A new generation of PMUs provide synchronized voltage and current measurements at a sampling rate orders of magnitude higher than the status quo: $10$ to $60$ samples per second rather than one 
sample every $1$ to $4$ seconds. Consequently, PMUs have the potential to enable
an entirely new set of applications for the power grid:  protection and control during abnormal conditions, real-time distributed control, postmortem analysis of system faults using time synchronized data, advanced state estimators for system monitoring, and the reliable integration of renewable
energy resources \cite{Naspi10}. 

An electric power system consists of a set of buses  -- an electric substation, power generation center, or aggregation of loads -- and transmission lines connecting those buses.
The state of a power system is defined by the voltage phasor -- the magnitude and phase angle -- of all system buses and the current phasor of all transmission lines.  
PMUs placed on buses provide real-time measurements of these system variables. % PMUs are placed on buses to measure the bus voltage phasor and the current phasor on all connected transmission lines. 
%PMUs provide real-time measurements of these system variables. PMUs are placed on buses to measure the bus voltage phasor and the current phasor on all connected transmission lines. 
However, because PMUs are expensive, they cannot be deployed on all system buses \cite{Baldwin93,LaRee10}. Fortunately, all system variables can be observed even when PMUs are placed at only a subset of system buses, by estimating the voltage phasor of buses without PMUs and the current phasors of unmeasured transmission lines using Ohm's and Kirchhoff's laws 
\cite{Baldwin93,Brueni05}. 
%system variables to be computed: the voltage phasor of buses without PMUs and current phasors of unmeasured transmission lines can be estimated using Ohm's and Kirchhoff's laws 
%\cite{Baldwin93,Brueni05}. 

In this work, we study two sets of PMU placement problems. The first problem -- \maxinc -- aims to observe the maximum number of system buses given a fixed number of PMUs.
A bus is said to be {\em observed} if there is a PMU placed at it or if its voltage phasor can be estimated using Ohm's or Kirchhoff's Law.  
\maxinc is closely related to another well-studied PMU placement problem: finding the minimum number of PMUs that can result in the observation of all system
buses \cite{Baldwin93,Brueni05,Haynes02, Mili90, Xu04}.  However, the \maxinc problem formulation presented here is unique.

The second set of placement problems aim to identify PMU errors, which have been recorded in PMUs used in practice \cite{Vanfretti10}. 
Informally, the goal of the first such placement problem 
(\xvalparts) is to observe, given a set of PMUs, the maximum number of
buses while deploying PMUs ``near'' each other to enable them to cross-validate each other's measurements.  Next, 
we formulate a placement problem similar to \xvalparts, called \xvals. In \xvals, the goal is to find the minimum number of PMUs such that all system buses are observed under 
the constraint that each PMU is placed near enough to at least one other PMU to allow its PMU measurements to be cross-validated.

%In this work, we study two sets of PMU placement problems. The first problem -- called \maxinc -- aims to observe the maximum number of system buses given a constant number of PMUs.
%An observed bus is one with a PMU or a bus in which Ohm's Law or Kirchhoff's Current Law can be used to estimated the voltage phasor.  In the second placement problem (\xvalparts), the goals 
%(informally) are twofold: given a constant number of PMUs, observe the maximum number of nodes and deploy PMUs ``near'' each other to enable PMUs to validate each other's measurements.  Finally,
%we formulate a third placement problem simlar to \xvalparts, called \xvals. In \xvals, the goal is to find the minimum number of PMUs such that all PMUs are placed near each other 
%-- to allow for PMU measurements to be validated -- and all system buses are observed.

%\maxinc is closely related to another PMU placement problem -- find the minimum number of PMUs that can result in the observation of all system buses -- 
%which is well studied \cite{Baldwin93,Brueni05,Haynes02, Mili90, Xu04}.  However, the \maxinc problem formulation is unique.
%We build on this related wor
%The \xvalpart formulation is motivated by PMU errors, in particular PMU phase angle errors, recorded in PMUs used by real utilities \cite{Vanfretti10}.  
%These errors must be identified and corrected, especially as PMUs become critical infrastructure in the power grid. 




We make the following contributions in this paper:
\begin{itemize}
	
	\item  We formally define graph-theoretic rules for cross-validating PMU placements. %, in which nearby PMUs can verify each other's measurements.  
	PMUs are cross-validated when PMU data from two buses can be used to each, independently, compute the voltage phasor of a non-PMU bus (Section \ref{subsec:xval}).
	%Informally, a PMU is cross-validated when PMU data from two buses can compute the voltage phasor of a non-PMU bus (Section \ref{subsec:xval}).
	%a PMU is cross-validated when the voltage phasor of a non-PMU bus is measured by two PMUs
	
	\item  We define three related PMU placement problems. The first problem seeks to observe the maximum number of nodes given a fixed number of PMUs.  
	The second two problems consider placing the minimum number of PMUs such that all PMUs are cross-validated and the maximum number of nodes is observed (Section \ref{sec:prelim}).
	%The second two placement problems consider placing the minimum number of PMUs such that the voltage phasor of a non-PMU bus is measured by two PMUs -- this ensures PMU data is cross-validated --
	%and the maximum number of nodes are observed (Section \ref{sec:prelim}).
	
	\item We prove that all three PMU placement problems are NP-Complete. This represents our most important contribution (Section \ref{sec:prelim}). 

	\item We present greedy approximation algorithms for each PMU placement problem (Section \ref{sec:approx}) and through simulations, we evaluate these algorithms over real
	IEEE bus systems and synthetic graphs (Section \ref{sec:simulations}).
	
\end{itemize}
