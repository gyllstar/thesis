\section{Introduction}
\label{sec:intro-pmu}

This chapter considers placing electric power grid sensors, called phasor measurement units (PMUs), to enable measurement error detection.
%\xx{move paragraph to Smart Grid Overview Chapterr}
Significant investments have been made to deploy PMUs on electric power grids worldwide. PMUs provide \emph{synchronized} voltage and current measurements at a sampling rate orders 
of magnitude higher than the status quo: $10$ to $60$ samples per second rather than one sample every $1$ to $4$ seconds.  This allows system operators to directly measure the state of the electric power grid in real-time, rather than 
relying on imprecise state estimation. Consequently, PMUs have the potential to enable
an entirely new set of applications for the power grid:  protection and control during abnormal conditions, real-time distributed control, postmortem analysis of system faults,
advanced state estimators for system monitoring, and the reliable integration of renewable energy resources \cite{Naspi10}.

%\xx{move paragraph to Smart Grid Overview Chapter}
An electric power system consists of a set of buses  -- electric substations, power generation centers, or aggregation points of electrical loads -- and transmission lines connecting those buses.
The state of a power system is defined by the voltage phasor -- the magnitude and phase angle of electrical sine waves -- of all system buses and the current phasor of all transmission lines.
PMUs placed on buses provide real-time measurements of these system variables.
However, because PMUs are expensive, they cannot be deployed on all system buses \cite{Baldwin93}\cite{LaRee10}. Fortunately, the voltage phasor at a system bus can, at times, 
be determined (termed {\it observed} in this paper) even when a PMU is not placed at that bus, by applying Ohm's and Kirchhoff's laws
on the measurements taken by a PMU placed at some nearby system bus \cite{Baldwin93}\cite{Brueni05}. Specifically, with correct placement of enough PMUs at a subset of system buses, the entire system state can be determined. 

In this chapter, we study two sets of PMU placement problems.  The first problem set consists of \full and \maxincs, and considers maximizing the observability of the network via PMU placement. \full considers the minimum number of PMUs needed 
to observe all system buses, while \maxinc considers the maximum number of buses that can be observed with a given number of PMUs. 
A bus is said to be {\em observed} if there is a PMU placed at it or if
its voltage phasor can be calculated using Ohm's or Kirchhoff's Law.  Although \full is well studied \cite{Baldwin93,Brueni05,Haynes02,Mili90,Xu04}, existing work considers only networks consisting solely of zero-injection buses, 
an unrealistic assumption in practice,
while we generalize the problem formulation to include mixtures of zero and  non-zero-injection buses. Additionally, our approach for analyzing \full provides the foundation with which to present the other three new (but related) PMU placement problems.

The second set of placement problems considers PMU placements that support PMU error detection. PMU measurement errors have been recorded in actual systems \cite{Vanfretti10}. 
One method of detecting these errors is to deploy PMUs ``near'' each other, thus enabling them to {\em cross-validate} each-other's measurements. 
{\xvals} aims to minimize the number of PMUs needed to observe all buses while insuring PMU cross-validation, and {\xvalparts} computes the maximum number of observed buses for a given number of PMUs, while insuring PMU cross-validation.


We make the following contributions in this chapter: 
\begin{itemize}
    
	\item We formulate two PMU placement problems, which (broadly) aim at maximizing observed buses while minimizing the number of PMUs used. Our formulation extends previously studied systems by 
	considering both zero and non-zero-injection buses.

    \item We formally define graph-theoretic rules for PMU cross-validation. Using these rules, we formulate two additional PMU placement problems that seek to maximize 
	the number of observed buses while minimizing the number of PMUs used under the condition that the PMUs are cross-validated. 

    \item We prove that all four PMU placement problems are NP-Complete. This represents our most important contribution.

	\item Given the proven complexity of these problems, we evaluate heuristic approaches for solving these problems. For each problem, we describe a greedy algorithm, and prove that each greedy
	algorithm has polynomial running time.

	\item Using simulations, we evaluate the performance of our greedy approximation algorithms over synthetic and actual
	IEEE bus systems. We find that the greedy algorithms yield a PMU placement that is, on average, within $97\%$ optimal. Additionally, we find that 
	the cross-validation constraints have limited effects on observability: on average our greedy algorithm that places PMUs according to the cross-validation rules observes 
	only $5.7\%$ fewer nodes than the same algorithm that does not consider cross-validation.

\end{itemize}

The rest of this chapter is organized as follows. In Section \ref{sec:prelim} we introduce our modeling assumptions, notation, and observability and cross-validation rules. In Section \ref{sec:problem-analysis} we formulate and prove the complexity of our four PMU placement problems. Section \ref{sec:approx} presents the approximation algorithms for each problem and Section \ref{sec:simulations} considers our simulation-based evaluation. We conclude with a review of related work (Section \ref{sec:related-pmu}) 
and concluding remarks (Section \ref{sec:pmu-conclude}).
